\subsection{随机变量}
\paragraph{}
有些试验的样本空间不是数字类型,因此可通过将样本空间映射到实数上,研究样本的概率分布情况。比如:硬币的$\{H,T\} \to \{0, 1\}$
\paragraph{}
\textbf{定义\;}设随机试验的样本空间为$S=\{e\}$。$X = X(e)$是定义在样本空间$S$上的实值单值函数,称$X=X(e)$为随机变量。

\paragraph{}
实值单值函数,比如:$y=f(x)$,实值即$x$的定义域是实数集合$R$;单值函数即$x$对应的$y$是唯一的。

\subsection{离散型随机变量及其\textbf{分布律}}
\subsubsection{离散型随机变量}
\paragraph{}
有些随机变量,它全部可能取到的值是有限个或可列无限多个,这种随机变量称为\textbf{离散型随机变量}。

\paragraph{}
设离散型随机变量$X$所有可能取的值为$x_k(k=1,2,\cdots)$,$X$取各个可能值的概率,即事件$\{X = x_k\}$的概率,为:
\begin{equation}
  \label{离散型随机变量的分布律}
  P\{X = x_k\} = p_k, k = 1, 2, \cdots.
\end{equation}
由概率的定义,$p_k$满足如下两个条件:

\label{离散型随机变量的条件}
\begin{enumerate}
  \item $p_k \geq 0, k = 1, 2, \cdots$
  \item $\displaystyle \sum_{k=1}^\infty p_k = 1$
\end{enumerate}
我们称 \eqref{离散型随机变量的分布律} 式为离散型随机变量$X$的\textbf{分布律},分布律也可以用表格的形式表示:

\begin{figure}[H]
\centering
  \begin{tabular}{c|ccccc}
    $X$ & $x_1$ & $x_2$ & $\cdots$ & $x_n$ & $\cdots$ \\
    \hline
    \\ [-1em]
    $p_k$ & $p_1$ & $p_2$ & $\cdots$ & $p_n$ & $\cdots$ \\
  \end{tabular}
\end{figure}

\subsubsection{(0-1)分布}
\paragraph{}
设随机变量$X$只可能取$0$与$1$两个值,它的分布律是:

\begin{equation}
  P\{X=k\}=p^k(1-p)^{1-k},k=0,1(0<p<1)
\end{equation}

则称$X$服从以$p$为参数的(0-1)分布或两点分布,以下以表格形式表示:

\begin{figure}[H]
\centering
  \begin{tabular}{c|cc}
    $X$ & $0$ & $1$ \\
    \hline
    \\ [-1em]
    $p_k$ & $1-p$ & $p$ \\
  \end{tabular}
\end{figure}

\subsubsection{伯努利试验、二项分布}
\paragraph{}
设试验$E$只有两个可能结果:$A$及$\overline{A}$,则称$E$为\textbf{伯努利(Bernoulli)试验}。设$P(A) = p(0<p<1)$,此时$P(\overline{A})=1-p$。将$E$独立重复地进行$n$次,则称这一串重复地独立试验为\textbf{$n$重伯努利试验}。

\paragraph{}
“重复”是指在每次试验中$P(A)=p$保持不变;“独立”是指各次试验的结果互不影响。若以$C_i$记第$i$次试验的结果,$C_i$为$A$或$\overline{A}, i = 1,2,\cdots,n$,“独立”是指:
\begin{equation}
  P(C_1C_2\cdots C_n) = P(C_1)P(C_2)\cdots P(C_n).
\end{equation}

\paragraph{}
以$X$表示$n$重伯努利试验中事件$A$发生的次数,$X$是一个随机变量。$X$所有可能取的值为$0,1,2,\cdots,n$。由于各次试验是相互独立的,因此事件$A$在指定的$k(0 \leq k \leq n)$次试验中发生,在其它$n-k$次试验中$A$不发生,前$k$次试验中$A$发生而后$n-k$次试验中$A$不发生的概率为:

\begin{equation}
  % \bigcdot 额外配置的命令
  \underbrace{p \bigcdot p \bigcdot \cdots \bigcdot p}_{k \text{个}} \bigcdot \underbrace{(1-p) \bigcdot (1-p) \bigcdot \cdots \bigcdot (1-p)}_{n-k \text{个}} = p^k(1-p)^{n-k}.
\end{equation}

这种指定的方式共有${{n}\choose{k}}$种,它们是两两互不相容的,故在$n$次试验中$A$发生$k$次的概率为${{n}\choose{k}}p^k(1-p)^{n-k}$,记$q = 1 - p$,即有:

\begin{equation}
  P\{X=k\} = {{n}\choose{k}}p^kq^{n-k}, k=0,1,2,\cdots,n.
\end{equation}
显然
\begin{gather}
  P\{X=k\} \geq 0, k = 0,1,2,\cdots,n; \\
  \sum_{k=0}^nP\{X=k\} = \sum_{k=0}^n {{n}\choose{k}}p^kq^{n-k}=(p+q)^n = 1.
\end{gather}
即$P\{X=k\}$满足 \hyperref[离散型随机变量的条件]{\color{blue} \ref*{离散型随机变量的条件}} 的两个条件,注意到${{n}\choose{k}}p^kq^{n-k}$刚好是二项式$(p+q)^n$的展开式中出现$p^k$的那一项,我们称随机变量$X$服从参数为$n,p$的\textbf{二项分布},并记为$X \thicksim b(n,p)$。

\paragraph{}
特别地,当$n=1$时,二项分布变成(0-1)分布。

\subsubsection{泊松分布}
\paragraph{}
设随机变量$X$所有可能取的值为$0,1,2,\cdots$,而取各个值的概率为

\begin{equation}
  P\{X=k\} = \frac{\lambda^ke^{-\lambda}}{k!}, k = 0,1,2,\cdots,
\end{equation}

其中$\lambda > 0$是常数。则称$X$服从参数为$\lambda$的\textbf{泊松分布},记为$X\thicksim \pi(\lambda)$

\paragraph{}
\textbf{泊松定理\;}设$\lambda>0$是一个常数,$n$是任意正整数,设$np_n=\lambda$,则对于任一固定的非负整数$k$,有

\begin{equation}
  \lim_{n \to \infty} {n \choose k}p_n^k(1-p)^{n-k}=\frac{\lambda^ke^{-\lambda}}{k!}.
\end{equation}

\subsection{随机变量的\textbf{分布函数}}
\paragraph{}
非离散型随机变量$X$,其取值不能一一列举出来,因此不能用分布律来描述它。实际中,研究误差落在某个区间,而不是具体某个样本,由于

\begin{equation}
  P\{x_1<X\leq x_2\} = P\{X\leq x_2\} - P\{X \leq x_1\},
\end{equation}

我们只需知道$P\{X\leq x_2\}$和$P\{X \leq x_1\}$,即可知道落在区间$(x_1,x_2]$的概率了。

\paragraph{}
\textbf{定义\;}设$X$是一个随机变量,$x$是任意实数,函数

\begin{equation}
  F(x) = P\{X \leq x\}, -\infty < x < \infty
\end{equation}

称为$X$的\textbf{分布函数}。

\paragraph{}
对于任意实数$x_1, x_2 (x_1 < x_2)$,有

\begin{align}
  \begin{split}
    P\{x_1 < X \leq x_2\} =&\; P\{X \leq x_2\} - P\{X \leq x_1\} \\
    =&\; F(x_2) - F(x_1)
  \end{split}
\end{align}

\paragraph{}
分布函数$F(x)$具有以下的基本性质:

\begin{enumerate}
  \item $F(x)$是一个不减函数,由$F(x_2) - F(x_1) = P\{x_1 < X \leq x_2\} \geq 0$可证明,也称为累计分布函数
  \item $0 \leq F(x) \leq 1$,且
  \begin{align}
    \begin{split}
      F(-\infty) =&\; \lim_{x \to -\infty}F(x) = 0, \\
      F(\infty) =&\; \lim_{x \to \infty}F(x) = 1. \\
    \end{split}
  \end{align}
  \item $F(x+0) = F(x)$,即$F(x)$是右连续的。
\end{enumerate}

\subsection{连续型随机变量及其概率密度}
\paragraph{}
如果对于随机变量$X$的分布函数$F(x)$,存在非负函数$f(x)$,使对于任意实数$x$有

\begin{equation}
  \label{连续型随机变量的分布函数}
  F(x) = \int_{-\infty}^{x}f(t)dt,
\end{equation}
则称$X$为\textbf{连续型随机变量},其中函数$f(x)$称为$X$的\textbf{概率密度函数},简称\textbf{概率密度}。

\paragraph{}
概率密度$f(x)$具有以下性质:
\begin{enumerate}
  \item $f(x) \geq 0$
  \item $\int_{-\infty}^{\infty}f(x)dx=1$
  \item 对于任意实数$x_1, x_2(x_1 \leq x_2),$
  \begin{equation}
    P\{x_1 < X \leq x_2\} = F(x_2) - F(x_1) = \int_{x_1}^{x_2}f(x)dx
  \end{equation}
  \item 若$f(x)$在点$x$处连续,则有$F'(x) = f(x)$
\end{enumerate}

\begin{figure}[h]
\centering
  %------- 第1行 -------
  \begin{subfigure}[t]{0.48\linewidth}
    \centering
      % 1.05*e^(-(x+1)^2) + 0.8*e^(-(x-1)^2) + 0.1
\begin{tikzpicture}[scale = 0.9]
  \begin{axis}[clip=false,xmin=-4.5,xmax=4.5,ymin=-0.15,ymax=1.6,ticks=none,axis lines=middle,smooth,xlabel={$x$}, ylabel={$f(x)$}]
    \addplot[draw=blue,domain=-4.5:4,samples=200] {1.05*e^(-(x+1)^2) + 0.8*e^(-(x-1)^2) + 0.1};
    \addplot+[draw=none,mark=none,domain=-4.5:4,samples=100,%
              pattern=north east lines]%
              {1.05*e^(-(x+1)^2) + 0.8*e^(-(x-1)^2) + 0.1}
              \closedcycle;
    \node[fill=white] at (-1,0.5) {$1$};
    \node[below left] at (0,0) {$O$};
  \end{axis}
\end{tikzpicture}

  \end{subfigure}
  \begin{subfigure}[t]{0.48\linewidth}
    \centering
      \input{figure/2_cdf_section.tex}
  \end{subfigure}
  \caption{概率密度性质}
  \label{概率密度性质}
\end{figure}

\paragraph{}
由\textbf{性质4}在$f(x)$的连续点$x$处有

\begin{align}
  \label{f(x)连续性质}
  \begin{split}
    f(x) =&\; \lim_{\Delta x \to 0^+}\frac{F(x+\Delta x) - F(x)}{\Delta x} \\
    =&\; \lim_{\Delta x \to 0^+}\frac{P\{x < X \leq x + \Delta x\}}{\Delta x}.
  \end{split}
\end{align}

\paragraph{}
由 \eqref{f(x)连续性质} 式知道,若不计高阶无穷小,有

\begin{equation}
  P\{x < X \leq x + \Delta x\} \approx f(x)\Delta x.
\end{equation}

这表示$X$落在小区间$(x, x+\Delta x]$上的概率近视地等于$f(x)\Delta x$.

\paragraph{}
对于连续型随机变量$X$来说,它取任一指定实数值$a$的概率均为$0$,即$P\{X=a\}=0$,因此可以不必区分该区间是开区间或闭区间:

\begin{equation}
  P\{a < X \leq b\} = P\{a \leq X \leq b\} = P\{a < X < b\}.
\end{equation}

\paragraph{}
下面介绍三种重要的连续型随机变量。

\subsubsection{均匀分布}
\paragraph{}
若连续型随机变量$X$具有概率密度

\begin{equation}
  f(x) = \left\{ \begin{array}{ll}
    \frac{1}{b-a}, & a<x<b, \\
    0, & \textbf{\small 其它},
  \end{array}\right.
\end{equation}

则称$X$在区间$(a,b)$上服从\textbf{均匀分布}。记为$X \sim U(a,b)$

\paragraph{}
对于任一长度$l$的子区间$(c, c+l), a \leq c < c+l \leq b$,有

\begin{equation}
  P\{c < X \leq c+l\} = \int_c^{c+l}f(x)dx = \int_c^{c+l}\frac{1}{b-a}dx=\frac{l}{b-a}.
\end{equation}

\paragraph{}
由 \eqref{连续型随机变量的分布函数} 式得$X$的分布函数为:

\begin{equation}
  F(x) = \left\{ \begin{array}{ll}
    0, & x < a, \\
    \frac{x-a}{b-a}, & a \leq x < b, \\
    1, & x \geq b.
  \end{array} \right.
\end{equation}

\paragraph{}
$f(x)$及$F(x)$的图形分别为:

\begin{figure}[h]
\centering
  %------- 第1行 -------
  \begin{subfigure}[t]{0.48\linewidth}
    \centering
      \input{figure/2_uniform.tex}
  \end{subfigure}
  \begin{subfigure}[t]{0.48\linewidth}
    \centering
      \input{figure/2_uniform_cumulative.tex}
  \end{subfigure}
  \caption{均匀分布的概率密度及其分布函数}
  \label{均匀分布的概率密度及其分布函数}
\end{figure}

\subsubsection{指数分布}
\paragraph{}

\subsubsection{正态分布}
\paragraph{}

\subsection{随机变量的函数的分布}
\paragraph{}
