\subsection{二阶行列式}
\subsubsection{二元线性方程组}
\paragraph{}
用消元法解二元线性方程组
\begin{align}
\label{消元法解二元线性方程组}
\begin{split}
  \left\{ \begin{array}{l}
    a_{11}x_1 + a_{12}x_2 = b_1, \\
    a_{21}x_1 + a_{22}x_2 = b_2.
  \end{array} \right.
\end{split}
\end{align}

\paragraph{}
为消去未知数$x_2$,以$a_{22}$与$a_{12}$分别乘上列两方程的两端,然后两个方程相减,得
\begin{equation*}
  (a_{11}a_{22} - a_{12}a_{21})x_1 = b_1a_{22} - a_{12}b_2;
\end{equation*}
类似地,消去$x_1$,得
\begin{equation*}
  (a_{11}a_{22} - a_{12}a_{21})x_2 = a_{11}b_2 - b_1a_{21}.
\end{equation*}

\paragraph{}
当$a_{11}a_{22} - a_{12}a_{21} \neq 0$时,求得方程组\eqref{消元法解二元线性方程组}的解为
\begin{equation}
  \label{二元线性方程组的解}
  x_1 = \frac{ b_1a_{22} - a_{12}b_2 }{ a_{11}a_{22} - a_{12}a_{21} }, \; x_2 = \frac{ a_{11}b_2 - b_1a_{21} }{ a_{11}a_{22} - a_{12}a_{21} }.
\end{equation}

\subsubsection{二阶行列式}
\paragraph{}
其中分母$a_{11}a_{22} - a_{12}a_{21}$ 是由方程组\eqref{消元法解二元线性方程组}的四个系数确定的,把这四个数按它们在方程组\eqref{消元法解二元线性方程组}中的位置,排成二行二列的数表
\begin{equation}
\label{二阶数表}
\begin{array}{cc}
a_{11} & a_{12} \\ a_{21} & a_{22},
\end{array}
\end{equation}
表达式$a_{11}a_{22} - a_{12}a_{21}$称为数表\eqref{二阶数表}所确定的\textbf{二阶行列式},并记作
\begin{equation}
  \left|
  \begin{array}{cc}
  a_{11} & a_{12} \\ a_{21} & a_{22}
  \end{array} \right|.
\end{equation}

\subsubsection{对角线法则}
\paragraph{}
二阶行列式的定义,可以用\textbf{对角线法则}来记忆。$a_{11}$到$a_{22}$的实线称为\textbf{主对角线},$a_{12}$到$a_{21}$的虚线称为\textbf{副对角线}。于是二阶行列式便是:主对角线上的两元素之积减去副对角线上的两元素之积所得的差。
\begin{figure}[H]
\centering
  \input{figure/1/2d_diagonal‘s_rule}
  \caption{对角线法则}
  \label{图:二阶对角线法则}
\end{figure}

\subsubsection{二元线性方程组的解}
\paragraph{}
利用二阶行列式的概念和对角线法则,\eqref{二元线性方程组的解}式中$x_1,x_2$的分子也可写成二阶行列式,即
\begin{equation*}
  b_1a_{22} - a_{12}b_2 = \left|\begin{array}{cc} b_1 & a_{12} \\ b_2 & a_{22}\end{array}\right|, \;
  a_{11}b_2 - b_1a_{21} = \left|\begin{array}{cc} a_{11} & b_1 \\ a_{21} & b_2\end{array}\right|,
\end{equation*}

\paragraph{}
若记
\begin{equation*}
  D = \left|\begin{array}{cc} a_{11} & a_{12} \\ a_{21} & a_{22} \end{array}\right|, \;
  D_1 = \left|\begin{array}{cc} b_1 & a_{12} \\ b_2 & a_{22} \end{array}\right|, \;
  D_2 = \left|\begin{array}{cc} a_{11} & b_1 \\ a_{21} & b_2 \end{array}\right|, \;
\end{equation*}
那么\eqref{二元线性方程组的解}式可写成
\begin{equation*}
  x_1 = \frac{D_1}{D} = \frac{\left|\begin{array}{cc} b_1 & a_{12} \\ b_2 & a_{22} \end{array}\right|}{\left|\begin{array}{cc} a_{11} & a_{12} \\ a_{21} & a_{22} \end{array}\right|}, \;
  x_2 = \frac{D_2}{D} = \frac{\left|\begin{array}{cc} a_{11} & b_1 \\ a_{21} & b_2 \end{array}\right|}{\left|\begin{array}{cc} a_{11} & a_{12} \\ a_{21} & a_{22} \end{array}\right|}.
\end{equation*}

\paragraph{}
注意,这里的分母$D$是由方程组\eqref{消元法解二元线性方程组}的系数所确定的二阶行列式(称系数行列式),$x_1$的分子$D_1$是常数项$b_1,b_2$替换$D$中$x_1$的系数$a_{11}, a_{21}$(第$1$列)所得的二阶行列式;$x_2$的分子$D_2$是用常数项$b_1, b_2$替换$D$中$x_2$的系数$a_{12}, a_{22}$(第$2$列)所得的二阶行列式。

\subsection{三阶行列式}
\subsubsection{定义}
\paragraph{}
\textbf{定义~~}设有$9$个数排成$3$行$3$列的数表
\begin{equation}
  \label{三阶数表}
  \begin{array}{ccc}
    a_{11} & a_{12} & a_{13} \\
    a_{21} & a_{22} & a_{23} \\
    a_{31} & a_{32} & a_{33},
  \end{array}
\end{equation}
记
\begin{align}
\centering
  \begin{split}
    \label{三阶行列式}
    &\;\left|\begin{array}{ccc}
      a_{11} & a_{12} & a_{13} \\
      a_{21} & a_{22} & a_{23} \\
      a_{31} & a_{32} & a_{33}
    \end{array}\right| \\
    =&\; a_{11}a_{22}a_{33} + a_{12}a_{23}a_{31} + a_{13}a_{21}a_{32} - \\
    &\; a_{11}a_{23}a_{32} - a_{12}a_{21}a_{33} - a_{13}a_{22}a_{31},
  \end{split}
\end{align}
\eqref{三阶行列式}称为数表\eqref{三阶数表}所确定的三阶行列式。

\subsubsection{对角线法则}
\begin{figure}[H]
\centering
  \input{figure/1/3d_diagonal‘s_rule}
  \caption{对角线法则}
  \label{图:三阶对角线法则}
\end{figure}

\paragraph{}
对角线法则只适用于二阶与三阶行列式,下面先介绍全排列及其逆序数,然后由此引出$n$阶行列式的概念。

\subsection{全排列及其逆序数}
\subsubsection{全排列}
\paragraph{}
把$n$个不同的元素排成一列,叫做这$n$个元素的\textbf{全排列}(简称\textbf{排列})。

\paragraph{}
从$n$个元素中任取一个放在第一个位置上,有$n$种取法;又从剩下的$n-1$个元素中任取一个放在第二个位置上,有$n-1$种取法;依此类推,于是:
\begin{equation*}
  P_n = n \bigcdot (n-1) \bigcdot \cdots \bigcdot 3 \bigcdot 2 \bigcdot 1 = n!.
\end{equation*}

\subsubsection{逆序数}
\paragraph{}
概念:
\begin{enumerate}
  \item \textbf{标准次序}:$n$个不同的自然数,按由小到大的顺序排序
  \item \textbf{逆序}:当某两个元素的先后次序与标准次序不同时,就说有$1$个\uwave{逆序}
  \item \textbf{排列的逆序数}:一个排列中所有逆序的总数
  \item \textbf{奇/偶排列}:逆序数为奇/偶数
\end{enumerate}

\paragraph{}
下面讨论计算排列的逆序数的方法:

\paragraph{}
一般性,设$n$个元素为$1$至$n$这$n$个自然数,并规定由小到大为标准次序,设:
\begin{equation*}
  p_1p_2\cdots p_n
\end{equation*}
为这$n$个自然数的一个排列,考虑元素$p_i(i=1,2,\cdots,n)$,如果比$p_i$大的且排在$p_i$前面的元素有$t_i$个,就说$p_i$这个元素的逆序数是$t_i$。全体元素的逆序数之总和
\begin{equation*}
  t = t_1 + t_2 + \cdots + t_n = \sum_{t=1}^n t_i,
\end{equation*}
即是这个排列的逆序数。

\subsection{$n$阶行列式的定义}
\subsubsection{三阶行列式的结构}
\paragraph{}
先研究三阶行列式的结构,然后推广到$n$阶行列式的定义。
\begin{align}
\centering
  \begin{split}
    \label{三阶行列式的结构}
    &\;\left|\begin{array}{ccc}
      a_{11} & a_{12} & a_{13} \\
      a_{21} & a_{22} & a_{23} \\
      a_{31} & a_{32} & a_{33}
    \end{array}\right| \\
    =&\; a_{11}a_{22}a_{33} + a_{12}a_{23}a_{31} + a_{13}a_{21}a_{32} - \\
    &\; a_{11}a_{23}a_{32} - a_{12}a_{21}a_{33} - a_{13}a_{22}a_{31},
  \end{split}
\end{align}

\begin{enumerate}[label=\alph*)]
  \item 第\eqref{三阶行列式的结构}式右边的每一项都恰是三个元素的乘积,这三个元素位于不同的行、不同的列。因此,第\eqref{三阶行列式的结构}式右边的任一项除符号外,可以写成$a_{1p_1}a_{2p_2}a_{3p_3}$,这里第一个下标(行标)排成标准次序$123$,而第二个下标(列标)排成$p_1p_2p_3$,它是$1,2,3$的某个排列。这样的排列共有$6$种。
  \item 各项的正负号与列标的奇偶排列对照。
  \\带正号的三项列标排列是:$123, \; 231, \; 312$;
  \\带负号的三项列标排列是:$132, \; 213, \; 321$。
  \\经计算可知前三个排列都是偶排列,而后三个排列都是奇排列。因此各项所带的正负号可以表示为$(-1)^t$,其中t为列标排列的逆序数。
\end{enumerate}

\paragraph{}
总之,三阶行列式可以写成
\begin{equation*}
  \left|\begin{array}{ccc}
    a_{11} & a_{12} & a_{13} \\
    a_{21} & a_{22} & a_{23} \\
    a_{31} & a_{32} & a_{33}
  \end{array}\right| = \sum(-1)^ta_{1p_1}a_{2p_2}a_{3p_3},
\end{equation*}
其中$t$为排列$p_1p_2p_3$的逆序数,$\sum$表示对$1,2,3$个数的所有排列$p_1p_2p_3$取和。

\subsubsection{$n$阶行列式的定义}
\paragraph{}
从上面的三阶行列式的研究,可以把行列式推广到一般情形。

\paragraph{}
\textbf{定义~~}设有$n^2$个数,排成$n$行$n$列的数表
\begin{equation*}
\begin{array}{cccc}
  a_{11} & a_{12} & \cdots & a_{1n} \\
  a_{21} & a_{22} & \cdots & a_{2n} \\
  \vdots & \vdots & \vdots & \vdots \\
  a_{n1} & a_{n2} & \cdots & a_{nn}
\end{array}
\end{equation*}
作出表中位于不同行不同列的$n$个数的乘积,并冠以符号$(-1)^t$,得到形如
\begin{equation}
  \label{n阶行列式的通项}
  (-1)^ta_{1p_1}a_{2p_2}\cdots a_{np_n}
\end{equation}
的项,其中$p_1p_2\cdots p_n$为自然数$1,2,\cdots,n$的一个排列,$t$为这个排列的逆序数。由于这样的排列共有$n!$个,因而形如\eqref{n阶行列式的通项}式的项共有$n!$项。所有这$n!$项的代数和
\begin{equation*}
  \sum(-1)^ta_{1p_1}a_{2p_2}\cdots a_{np_n}
\end{equation*}
称为\textbf{$n$阶行列式},记作
\begin{equation*}
  D = \left|\begin{array}{cccc}
    a_{11} & a_{12} & \cdots & a_{1n} \\
    a_{21} & a_{22} & \cdots & a_{2n} \\
    \vdots & \vdots & \vdots & \vdots \\
    a_{n1} & a_{n2} & \cdots & a_{nn}
  \end{array}\right|,
\end{equation*}
简记作$det(a_{ij})$,其中数$a_{ij}$为行列式$D$的$(i,j)$元。

\subsubsection{特殊的$n$阶行列式}
\paragraph{}
\textbf{对角行列式}
\begin{equation*}
  \left|\begin{array}{cccc}
    \lambda_1 & & & \\
    & \lambda_2 & & \\
    & & \ddots & \\
    & & & \lambda_n
  \end{array}\right| = \lambda_1\lambda_2\cdots \lambda_n,
\end{equation*}

\begin{equation*}
  \left|\begin{array}{cccc}
    & & & \lambda_1 \\
    & & \lambda_2 & \\
    & \iddots & & \\
    \lambda_n & & &
  \end{array}\right| = (-1)^{\frac{n(n-1)}{2}}\lambda_1\lambda_2\cdots \lambda_n,
\end{equation*}

\paragraph{}
主对角线以下(上)的元素都为$0$的行列式叫做\textbf{上(下)三角形行列式},它的值与对角行列式一样。例如,下三角形行列式:
\begin{equation*}
  \left|\begin{array}{cccc}
    a_{11} & & & 0 \\
    a_{21} & a_{22} & & \\
    \vdots & \vdots & \ddots & \\
    a_{n1} & a_{n2} & \cdots & a_{nn}
  \end{array}\right| = a_{11}a_{22}\cdots a_{nn}.
\end{equation*}

\subsection{对换}
\paragraph{}
为了研究$n$阶行列式的性质,先来讨论对换以及它与排列的奇偶性的关系。

\paragraph{}
在排列中,将任意两个元素对调,其余的元素不动,这种作出新排列的手续叫做\textbf{对换}。将相邻两个元素对换,叫做\textbf{相邻对换}。

\paragraph{}
\textbf{定理1~~}一个排列中的任意两个元素对换,排列改变奇偶性。

\paragraph{}
\textbf{推论~~}奇排列变成标准排列的对换次数为奇数,偶排列变成标准排列的对换次数为偶数。

\paragraph{}
\textbf{定理2~~}$n$阶行列式也可定义为
\begin{equation*}
  D = \sum(-1)^ta_{p_11}a_{p_22}\cdots a_{p_nn},
\end{equation*}
其中$t$为行标排列$p_1p_2\cdots p_n$的逆序数。

\subsection{行列式的性质}
\paragraph{}
记
\begin{equation*}
  D = \left|\begin{array}{cccc}
    a_{11} & a_{12} & \cdots & a_{1n} \\
    a_{21} & a_{22} & \cdots & a_{2n} \\
    \vdots & \vdots & \vdots & \vdots \\
    a_{n1} & a_{n2} & \cdots & a_{nn}
  \end{array} \right|, \;
  D^T = \left|\begin{array}{cccc}
    a_{11} & a_{21} & \cdots & a_{n1} \\
    a_{12} & a_{22} & \cdots & a_{n2} \\
    \vdots & \vdots & \vdots & \vdots \\
    a_{1n} & a_{2n} & \cdots & a_{nn}
  \end{array} \right|,
\end{equation*}
行列式$D^T$称为行列式$D$的\textbf{转置行列式}。

\paragraph{}
\textbf{性质1~~}行列式与它的转置行列式相等。

\paragraph{}
\textbf{性质2~~}互换行列式的两行(列),行列式变号。

\paragraph{}
\textbf{推论~~}如果行列式有两行(列)完全相同,则此行列式等于零。

\paragraph{}
\textbf{性质3~~}行列式的某一行(列)中所有的元素都乘以同一数$k$,等于用数$k$乘此行列式。

\paragraph{}
第$i$行(列)乘以$k$,记作$r_i \times k$($c_i \times k$)。

\paragraph{}
\textbf{推论~~}行列式中某一行(列)的所有元素的公因子可以提到行列式记号的外面。

\paragraph{}
第$i$行(列)提出公因子$k$,记作$r_i \div k$($c_i \div k$)。

\paragraph{}
\textbf{性质4~~}行列式中如果有两行(列)元素成比例,则此行列式等于零。

\paragraph{}
\textbf{性质5~~}若行列式的某一列(行)的元素都是两数之和,例如第$i$列的元素都是两数之和:
\begin{equation*}
\left|\begin{array}{cccccc}
  a_{11} & a_{12} & \cdots & (a_{1i}+a'_{1i}) & \cdots & a_{1n} \\
  a_{21} & a_{22} & \cdots & (a_{2i}+a'_{2i}) & \cdots & a_{2n} \\
  \vdots & \vdots & \vdots & \vdots & \vdots & \vdots \\
  a_{n1} & a_{n2} & \cdots & (a_{ni}+a'_{ni}) & \cdots & a_{n n}
\end{array}\right|
\end{equation*}
则$D$等于下列两个行列式之和:
\begin{align*}
  D =&\; \left|\begin{array}{cccccc}
    a_{11} & a_{12} & \cdots & a_{1i} & \cdots & a_{1n} \\
    a_{21} & a_{22} & \cdots & a_{2i} & \cdots & a_{2n} \\
    \vdots & \vdots & \vdots & \vdots & \vdots & \vdots \\
    a_{n1} & a_{n2} & \cdots & a_{ni} & \cdots & a_{n n}
  \end{array}\right| \\
  &\; + \left|\begin{array}{cccccc}
    a_{11} & a_{12} & \cdots & a'_{1i} & \cdots & a_{1n} \\
    a_{21} & a_{22} & \cdots & a'_{2i} & \cdots & a_{2n} \\
    \vdots & \vdots & \vdots & \vdots & \vdots & \vdots \\
    a_{n1} & a_{n2} & \cdots & a'_{ni} & \cdots & a_{n n}
  \end{array}\right|
\end{align*}

\paragraph{}
\textbf{性质6~~}把行列式的某一列(行)的各元素乘以同一数然后加到另一列(行)对应的元素上去,行列式不变。
\paragraph{}
例如以数$k$乘第$j$列加到第$j$列上(记作$c_i+kc_j$),有
\begin{align*}
  &\;\left|\begin{array}{ccccccc}
    a_{11} & \cdots & a_{1i} & \cdots & a_{1j} & \cdots & a_{1n} \\
    a_{21} & \cdots & a_{2i} & \cdots & a_{2j} & \cdots & a_{2n} \\
    \vdots & \vdots & \vdots & \vdots & \vdots & \vdots & \vdots \\
    a_{n1} & \cdots & a_{ni} & \cdots & a_{nj} & \cdots & a_{nn}
  \end{array}\right| \\
  \stackrel{c_i+kc_j}{=\joinrel=\joinrel=\joinrel=\joinrel=} &\;\left|\begin{array}{ccccccc}
    a_{11} & \cdots & (a_{1i} + ka_{1j}) & \cdots & a_{1j} & \cdots & a_{1n} \\
    a_{21} & \cdots & (a_{2i} + ka_{2j}) & \cdots & a_{2j} & \cdots & a_{2n} \\
    \vdots & \vdots & \vdots & \vdots & \vdots & \vdots & \vdots \\
    a_{n1} & \cdots & (a_{ni} + ka_{nj}) & \cdots & a_{nj} & \cdots & a_{nn}
  \end{array}\right| \; (i \neq j).
\end{align*}

\paragraph{}
此外还要注意运算$r_i + r_j$与$r_j + r_i$的区别,记号$r_i + kr_j$不能写作$kr_j + r_i$(这里不能套用加法的交换率)。

\subsection{行列式按行(列)展开}

\subsection{克拉默法则}
