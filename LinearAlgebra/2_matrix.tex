\subsection{矩阵的概念}
\subsubsection{概念}
\paragraph{}
\textbf{定义~~}由$m\times n$个数$a_{ij} (i=1,2,\cdots,m; \; j =1,2,\cdots,n)$排成的$m$行$n$列的数表
\begin{equation*}
  \left|\begin{array}{cccc}
    a_{11} & a_{12} & \cdots & a_{1n} \\
    a_{21} & a_{22} & \cdots & a_{2n} \\
    \vdots & \vdots &  & \vdots \\
    a_{m1} & a_{m2} & \cdots & a_{mn}
  \end{array} \right|
\end{equation*}
称为\textbf{$m$行$n$列矩阵},简称\textbf{$m \times n$ 矩阵},记作
\begin{equation}
  A = \left[\begin{array}{cccc}
  a_{11} & a_{12} & \cdots & a_{1n} \\
  a_{21} & a_{22} & \cdots & a_{2n} \\
  \vdots & \vdots &  & \vdots \\
  a_{m1} & a_{m2} & \cdots & a_{mn}
\end{array} \right]
\end{equation}
也记作$A_{m\times n}$

\paragraph{}
行数和列数都等于$n$的矩阵称为\textbf{$n$阶矩阵}或\textbf{$n$阶方阵}。$n$阶矩阵$A$也记作$A_n$。

\paragraph{}
只有一行的矩阵
\begin{equation*}
  A = (a_1a_2\cdots a_n)
\end{equation*}
称为\textbf{行矩阵},又称\textbf{行向量}。为了避免元素间的混淆,也记作
\begin{equation*}
  A = (a_1,a_2,\cdots,a_n).
\end{equation*}

\paragraph{}
只有一列的矩阵
\begin{equation*}
  B = \left[\begin{array}{c}
    b_1 \\
    b_2 \\
    \cdots \\
    b_m
  \end{array}\right]
\end{equation*}
称为\textbf{列矩阵},又称\textbf{列向量}。

\paragraph{}
两个矩阵的行数相等、列数也相等时,就称它们是\textbf{同型矩阵}。如果$A=(a_{ij})$与$B=(b_{ij})$是同型矩阵,并且它们的对应元素相等,即
\begin{equation*}
  a_{ij}=b_{ij} \;(i=1,2,\cdots,m; \; j=1,2,\cdots,n),
\end{equation*}
那么就称矩阵$A$与矩阵$B$\textbf{相等},记作
\begin{equation*}
  A=B.
\end{equation*}

\subsubsection{类型}
\paragraph{}
元素都是零的矩阵称为\textbf{零矩阵},记作$O$。注意不同型的零矩阵是不同的。

\paragraph{}
主对角线上的元素都是$1$,其余都为$0$的$n$阶矩阵,叫做$n$阶\textbf{单位矩阵},记作$E$
\begin{equation*}
  E = \left[\begin{array}{cccc}
    1 & 0 & \cdots & 0 \\
    0 & 1 & \cdots & 0 \\
    \vdots & \vdots & & \vdots \\
    0 & 0 & \cdots & 1
  \end{array}\right]
\end{equation*}

\paragraph{}
不在主对角线上的元素都是$0$,称为\textbf{对角矩阵},记作$A = diag(\lambda_1,\lambda_2,\cdots,\lambda_n)$
\begin{equation*}
  A = \left[\begin{array}{cccc}
    \lambda_1 & 0 & \cdots & 0 \\
    0 & \lambda_2 & \cdots & 0 \\
    \vdots & \vdots & & \vdots \\
    0 & 0 & \cdots & \lambda_n
  \end{array}\right]
\end{equation*}

\paragraph{}
\textbf{由于矩阵和线性变换之间存在一一对应的关系,因此可以利用矩阵来研究线性变换,也可以利用线性变换来解析矩阵的含义。}

\subsection{矩阵的运算}
\paragraph{}

\subsection{逆矩阵}
\paragraph{}

\subsection{矩阵的分块法}
\paragraph{}
