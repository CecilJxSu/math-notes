\subsection{矩阵的概念}
\subsubsection{概念}
\paragraph{}
\textbf{定义~~}由$m\times n$个数$a_{ij} (i=1,2,\cdots,m; \; j =1,2,\cdots,n)$排成的$m$行$n$列的数表
\begin{equation*}
  \left|\begin{array}{cccc}
    a_{11} & a_{12} & \cdots & a_{1n} \\
    a_{21} & a_{22} & \cdots & a_{2n} \\
    \vdots & \vdots &  & \vdots \\
    a_{m1} & a_{m2} & \cdots & a_{mn}
  \end{array} \right|
\end{equation*}
称为\textbf{$m$行$n$列矩阵},简称\textbf{$m \times n$ 矩阵},记作
\begin{equation}
  A = \left[\begin{array}{cccc}
  a_{11} & a_{12} & \cdots & a_{1n} \\
  a_{21} & a_{22} & \cdots & a_{2n} \\
  \vdots & \vdots &  & \vdots \\
  a_{m1} & a_{m2} & \cdots & a_{mn}
\end{array} \right]
\end{equation}
也记作$A_{m\times n}$

\paragraph{}
行数和列数都等于$n$的矩阵称为\textbf{$n$阶矩阵}或\textbf{$n$阶方阵}。$n$阶矩阵$A$也记作$A_n$。

\paragraph{}
只有一行的矩阵
\begin{equation*}
  A = (a_1a_2\cdots a_n)
\end{equation*}
称为\textbf{行矩阵},又称\textbf{行向量}。为了避免元素间的混淆,也记作
\begin{equation*}
  A = (a_1,a_2,\cdots,a_n).
\end{equation*}

\paragraph{}
只有一列的矩阵
\begin{equation*}
  B = \left[\begin{array}{c}
    b_1 \\
    b_2 \\
    \cdots \\
    b_m
  \end{array}\right]
\end{equation*}
称为\textbf{列矩阵},又称\textbf{列向量}。

\paragraph{}
两个矩阵的行数相等、列数也相等时,就称它们是\textbf{同型矩阵}。如果$A=(a_{ij})$与$B=(b_{ij})$是同型矩阵,并且它们的对应元素相等,即
\begin{equation*}
  a_{ij}=b_{ij} \;(i=1,2,\cdots,m; \; j=1,2,\cdots,n),
\end{equation*}
那么就称矩阵$A$与矩阵$B$\textbf{相等},记作
\begin{equation*}
  A=B.
\end{equation*}

\subsubsection{类型}
\paragraph{}
元素都是零的矩阵称为\textbf{零矩阵},记作$O$。注意不同型的零矩阵是不同的。

\paragraph{}
主对角线上的元素都是$1$,其余都为$0$的$n$阶矩阵,叫做$n$阶\textbf{单位矩阵},记作$E$
\begin{equation*}
  E = \left[\begin{array}{cccc}
    1 & 0 & \cdots & 0 \\
    0 & 1 & \cdots & 0 \\
    \vdots & \vdots & & \vdots \\
    0 & 0 & \cdots & 1
  \end{array}\right]
\end{equation*}

\paragraph{}
不在主对角线上的元素都是$0$,称为\textbf{对角矩阵},记作$A = diag(\lambda_1,\lambda_2,\cdots,\lambda_n)$
\begin{equation*}
  A = \left[\begin{array}{cccc}
    \lambda_1 & 0 & \cdots & 0 \\
    0 & \lambda_2 & \cdots & 0 \\
    \vdots & \vdots & & \vdots \\
    0 & 0 & \cdots & \lambda_n
  \end{array}\right]
\end{equation*}

\paragraph{}
\textbf{由于矩阵和线性变换之间存在一一对应的关系,因此可以利用矩阵来研究线性变换,也可以利用线性变换来解析矩阵的含义。}

\subsection{矩阵的运算}
\subsubsection{矩阵的加法}
\paragraph{}
\textbf{定义~~}设有两个$m\times n$矩阵$A=(a_{ij})$和$B=(b_{ij})$,那么\textbf{矩阵$A$与$B$的和}记作$A+B$,规定为
\begin{equation*}
  A+B=\left[\begin{array}{cccc}
    a_{11} + b_{11} & a_{12}+b_{12} & \cdots & a_{1n} + b_{1n} \\
    a_{21} + b_{21} & a_{22}+b_{22} & \cdots & a_{2n} + b_{2n} \\
    \vdots & \vdots &  & \vdots \\
    a_{m1} + b_{m1} & a_{m2}+b_{m2} & \cdots & a_{mn} + b_{mn}
  \end{array}\right]
\end{equation*}

\paragraph{}
运算规则(设$A, B, C$都是$m\times n$矩阵):
\begin{enumerate}
  \item $A+B = B+A$;
  \item $(A+B)+C = A+(B+C)$。
\end{enumerate}

\paragraph{}
设矩阵$A=(a_{ij})$,记
\begin{equation*}
  -A = (-a_{ij}),
\end{equation*}
$-A$称为矩阵$A$的\textbf{负矩阵},显然有
\begin{equation*}
  A + (-A) = O,
\end{equation*}
由此规定矩阵的减法为
\begin{equation*}
  A-B=A+(-B).
\end{equation*}

\subsubsection{数与矩阵相乘}
\paragraph{}
\textbf{定义~~}数$\lambda$与矩阵$A$的乘积记作$\lambda A$或$A\lambda$,规定为
\begin{equation*}
\lambda A = A\lambda = \left[\begin{array}{cccc}
  \lambda a_{11} & \lambda a_{12} & \cdots & \lambda a_{1n} \\
  \lambda a_{21} & \lambda a_{22} & \cdots & \lambda a_{2n} \\
  \vdots & \vdots &  & \vdots \\
  \lambda a_{m1} & \lambda a_{m2} & \cdots & \lambda a_{mn}
\end{array}\right]
\end{equation*}

\paragraph{}
数乘矩阵满足下列运算法则(设$A,B$为$m\times n$矩阵,$\lambda, \mu$为数):
\begin{enumerate}
  \item $(\lambda\mu)A=\lambda(\mu A)$;
  \item $(\lambda + \mu)A = \lambda A + \mu A$;
  \item $\lambda(A+B)=\lambda A + \lambda B$。
\end{enumerate}

\paragraph{}
矩阵加法和数乘矩阵合起来,统称为矩阵的\textbf{线性运算}。

\subsubsection{矩阵与矩阵相乘:线性变换}
\paragraph{}
设有两个线性变换
\begin{equation}
  \label{线性变换1}
  \left\{\begin{array}{c}
    y_1 = a_{11}x_1 + a_{12}x_2 + a_{13}x_3, \\
    y_2 = a_{21}x_1 + a_{22}x_2 + a_{23}x_3,
  \end{array} \right.
\end{equation}
\begin{equation}
  \label{线性变换2}
  \left\{\begin{array}{c}
    x_1 = b_{11}t_1 + b_{12}t_2, \\
    x_2 = b_{21}t_1 + b_{22}t_2, \\
    x_3 = b_{31}t_1 + b_{32}t_2,
  \end{array} \right.
\end{equation}
若想求出从$t_1,t_2$到$y_1,y_2$的线性变换,可将\eqref{线性变换2}代入\eqref{线性变换1},便得
\begin{equation}
  \label{线性变换3}
  \left\{\begin{array}{c}
    y_1 = (a_{11}b_{11} + a_{12}b_{21} + a_{13}b_{31})t_1 + (a_{11}b_{12} + a_{12}b_{22} + a_{13}b_{32})t_2, \\
    y_2 = (a_{21}b_{11} + a_{22}b_{21} + a_{23}b_{31})t_1 + (a_{21}b_{12} + a_{22}b_{22} + a_{23}b_{32})t_2,
  \end{array} \right.
\end{equation}

\paragraph{}
线性变换\eqref{线性变换3}可看成是先作线性变换\eqref{线性变换2}再作线性变换\eqref{线性变换1}的结果。我们把线性变换\eqref{线性变换3}叫做\textbf{线性变换\eqref{线性变换1}与\eqref{线性变换2}的乘积},相应的把\eqref{线性变换3}所对应的矩阵定义为\eqref{线性变换1}与\eqref{线性变换2}所对应的矩阵的乘积,即
\begin{align*}
& \left[\begin{array}{ccc}
  a_{11} & a_{12} & a_{13} \\
  a_{21} & a_{22} & a_{23}
\end{array}\right]\left[\begin{array}{cc}
  b_{11} & b_{12} \\
  b_{21} & b_{22} \\
  b_{31} & b_{32}
\end{array}\right] \\
& = \left[\begin{array}{cc}
  a_{11}b_{11} + a_{12}b_{21} + a_{13}b_{31} & a_{11}b_{12} + a_{12}b_{22} + a_{13}b_{32} \\
  a_{21}b_{11} + a_{22}b_{21} + a_{23}b_{31} & a_{21}b_{12} + a_{22}b_{22} + a_{23}b_{32}
\end{array}\right]
\end{align*}

\subsubsection{矩阵与矩阵相乘:定义}
\paragraph{}
\textbf{定义~~}设$A=(a_{ij})$是一个$m\times s$矩阵,$B=(b_{ij})$是一个$s\times n$矩阵,那么规定\textbf{矩阵$A$与矩阵$B$的乘积}是一个$m\times n$矩阵$C=(c_{ij})$,其中
\begin{align}
\begin{split}
  c_{ij} = a_{i1}b_{1j} + a_{i2}b_{2j} + \cdots + a_{is}b_{sj} = \sum_{k=1}^sa_{ik}b_{kj} &\\
  (i = 1,2,\cdots,m; \; j = 1,2,\cdots, n),&
\end{split}
\end{align}
并把此乘积记作
\begin{equation*}
  C=AB.
\end{equation*}

\paragraph{}
一般情况下,$AB\neq BA$。若$AB = BA$,则称方阵$A$与$B$是\textbf{可交换}的。

\paragraph{}
若有两个矩阵$A,B$满足$AB=O$,不能得出$A=O$或$B=O$的结论。若$A\neq O$而$A(X-Y)=O$,也不能得出$X=Y$的结论。

\paragraph{}
矩阵乘法不满足交换律,但满足结合律和分配律:
\begin{enumerate}
  \item $(AB)C = A(BC)$;
  \item $\lambda(AB) = (\lambda A)B = A(\lambda B)$(其中$\lambda$为数);
  \item $A(B+C) = AB + AC,\\(B+C)A = BA+CA$。
\end{enumerate}

\paragraph{}
对于单位矩阵$E$,容易验证
\begin{equation*}
  E_mA_{m\times n} = A_{m\times n}, \; A_{m\times n}E_n = A_{m\times n},
\end{equation*}
或简写成
\begin{equation*}
  EA=AE=A.
\end{equation*}
可见单位矩阵$E$在矩阵乘法中的作用类似于数$1$。

\paragraph{}
矩阵
\begin{equation*}
  \lambda E = \left[\begin{array}{cccc}
    \lambda & & & \\
    & \lambda & & \\
    & & \ddots & \\
    & & & \lambda
  \end{array}\right]
\end{equation*}
称为\textbf{纯量阵}。由$(\lambda E)A=\lambda A, \; A(\lambda E) = \lambda A$,可知纯量阵$\lambda E$与矩阵$A$的乘积等于数$\lambda$与$A$的乘积。并且当$A$为$n$阶方阵时,有
\begin{equation*}
  (\lambda E_n)A_n = \lambda A_n = A_n(\lambda E_n),
\end{equation*}
表明纯量阵$\lambda E$与任何同阶方阵都是可交换的。

\subsubsection{矩阵的幂}
\paragraph{}
有了矩阵的乘法,就可以定义\textbf{矩阵的幂}。设$A$是$n$阶方阵,定义
\begin{equation*}
  A^1 = A, \; A^2 = A^1A^1, \; \cdots, \;A^{k+1}=A^kA^1,
\end{equation*}
其中$k$为正整数,这就是说$A^k$就是$k$个$A$连乘。

\subsubsection{矩阵的转置}
\paragraph{}
\textbf{定义~~}把矩阵$A$的行换成同序数的列得到一个新矩阵,叫做$A$的\textbf{转置矩阵},记作$A^T$。

\paragraph{}
例如:
\begin{align*}
  A &= \left[\begin{array}{ccc}
    1 & 2 & 0 \\
    3 & -1 & 1
  \end{array}\right] \\
  A^T &= \left[\begin{array}{cc}
    1 & 3 \\
    2 & -1 \\
    0 & 1
  \end{array}\right].
\end{align*}

\paragraph{}
矩阵的转置也是一种运算,满足下述运算规律:
\begin{enumerate}
  \item $(A^T)^T = A$;
  \item $(A+B)^T = A^T + B^T$;
  \item $(\lambda A)^T = \lambda A^T$;
  \item $(AB)^T = B^TA^T$
\end{enumerate}

\paragraph{}
设$A$为$n$阶方阵,如果满足$A^T=A$,即
\begin{equation*}
  a_{ij}=a_{ji} \; (i,j=1,2,\cdots,n),
\end{equation*}
那么$A$称为\textbf{对称矩阵},简称\textbf{对称阵}。对称阵的特点是:它的运算以对角线为对称轴对应相等。

\subsubsection{方阵的行列式}\label{方阵的行列式}
\paragraph{}
\textbf{定义~~}由$n$阶方阵$A$的元素所构成的行列式(各元素的位置不变),称为\textbf{方阵$A$的行列式},记作$|A|$或$detA$。

\paragraph{}
由$A$确定$|A|$的这个运算满足下述规律(设$A,B$为$n$阶方阵,$\lambda$为数):
\begin{enumerate}
  \item $|A^T|=|A|$(行列式性质$1$);
  \item $|\lambda A| = \lambda^n|A|$;
  \item $|AB|=|A||B|$。
\end{enumerate}

\paragraph{}
行列式$|A|$的各个元素的代数余子式$A_{ij}$所构成的如下的矩阵
\begin{equation*}
  A^* = \left[\begin{array}{cccc}
    A_{11} & A_{21} & \cdots & A_{n1} \\
    A_{12} & A_{22} & \cdots & A_{n2} \\
    \vdots & \vdots &  & \vdots \\
    A_{1n} & A_{2n} & \cdots & A_{nn}
  \end{array}\right]
\end{equation*}
称为矩阵$A$的\textbf{伴随矩阵},简称\textbf{伴随阵}。

\paragraph{}
试证
\begin{equation*}
  AA^* = A^*A = |A|E.
\end{equation*}

\paragraph{}
\textbf{证~~}设$A=(a_{ij})$,记$AA^*=(b_{ij})$,则
\begin{equation*}
  b_{ij} = a_{i1}A_{j1} + a_{i2}A_{j2} + \cdots + a_{in}A_{jn} = |A|\delta_{ij},
\end{equation*}
故
\begin{equation*}
  AA^* = (|A|\delta_{ij}) = |A|(\delta_{ij}) = |A|E.
\end{equation*}
类似有
\begin{equation*}
  A^*A = (\sum_{k=1}^nA_{ki}a_{kj}) = (|A|\delta_{ij}) = |A|(\delta_{ij}) = |A|E.
\end{equation*}

\subsubsection{共轭矩阵}
\paragraph{}
当$A=(a_{ij})$为复矩阵时,用$\overline{a}_{ij}$表示$a_{ij}$的共轭复数,记
\begin{equation*}
  \overline{A} = (\overline{a}_{ij}),
\end{equation*}
$\overline{A}$称为$A$的\textbf{共轭矩阵}。

\paragraph{}
共轭矩阵满足下述运算规律(设$A,B$为复矩阵,$\lambda$为复数):
\begin{enumerate}
  \item $\overline{A+B}=\overline{A} + \overline{B}$;
  \item $\overline{\lambda A}=\overline{\lambda}\overline{A}$;
  \item $\overline{AB} = \overline{A}\overline{B}$。
\end{enumerate}

\subsection{逆矩阵}
\subsubsection{线性变换的逆变换}
\paragraph{}
设给定一个线性变换
\begin{equation}
  \left\{\begin{array}{l}
    y_1 = a_{11}x_1 + a_{12}x_2 + \cdots + a_{1n}x_n, \\
    y_2 = a_{21}x_1 + a_{22}x_2 + \cdots + a_{2n}x_n, \\
    \cdots\cdots\cdots\cdots \\
    y_n = a_{n1}x_1 + a_{n2}x_2 + \cdots + a_{nn}x_n,
  \end{array} \right.
\end{equation}
它的系数矩阵是一个$n$阶矩阵$A$,若记
\begin{equation*}
  X = \left[\begin{array}{c}
    x_1 \\
    x_2 \\
    \vdots \\
    x_n
  \end{array}\right],\;
  Y = \left[\begin{array}{c}
    y_1 \\
    y_2 \\
    \vdots \\
    y_n
  \end{array}\right],
\end{equation*}
则线性变换可记作
\begin{equation}
  \label{Y=AX}
  Y=AX.
\end{equation}

\paragraph{}
以$A$的伴随阵$A^*$左乘上式两端,并利用\secref{方阵的行列式}的结果,可得
\begin{equation*}
  A^*Y = A^*AX, \;\text{即}\; A^*Y = |A|X,
\end{equation*}
当$|A|\neq 0$时,可解出
\begin{equation*}
  X = \frac{1}{|A|}A^*Y,
\end{equation*}
记$B=\frac{1}{|A|}A^*$,上式可记作
\begin{equation}
  \label{X=BY}
  X = BY.
\end{equation}

\paragraph{}
\eqref{X=BY}式表示一个从$Y$到$X$的线性变换,称为线性变换\eqref{Y=AX}的\textbf{逆变换}。

\paragraph{}
研究方阵$A$与逆变换所对应的方阵$B$之间的关系。用\eqref{X=BY}代入\eqref{Y=AX},可得
\begin{equation*}
  Y = A(BY) = (AB)Y,
\end{equation*}
可见$AB$为恒等变换所对应的矩阵,故$AB=E$。用\eqref{Y=AX}代入\eqref{X=BY}得
\begin{equation*}
  X=B(AX)=(BA)X,
\end{equation*}
知有$BA=E$。于是有
\begin{equation*}
  AB=BA=E.
\end{equation*}

\subsubsection{定义}
\paragraph{}
\textbf{定义~~}对于$n$阶矩阵$A$,如果有一个$n$阶矩阵$B$,使
\begin{equation*}
  AB=BA=E,
\end{equation*}
则说矩阵$A$是\textbf{可逆的},并把矩阵$B$称为$A$的\textbf{逆矩阵},简称\textbf{逆阵}。

\paragraph{}
如果矩阵$A$是可逆的,那么$A$的逆阵是唯一的。

\subsubsection{定理}
\paragraph{}
\textbf{定理1~~}若矩阵$A$可逆,则$|A|\neq 0$。

\paragraph{}
\textbf{定理2~~}若$|A|\neq 0$,则矩阵$A$可逆,且
\begin{equation}
  A^{-1} = \frac{1}{|A|}A^*,
\end{equation}
其中$A^*$为矩阵$A$的伴随阵。

\paragraph{}
当$|A|=0$时,$A$称为\textbf{奇异矩阵},否则称\textbf{非奇异矩阵}。由上面两定理可知:\textbf{$A$是可逆矩阵的充分必要条件是$|A|\neq 0$,即可逆矩阵就是非奇异矩阵}。

\subsubsection{推论}
\paragraph{}
\textbf{推论~~}若$AB=E$(或$BA=E$),则$B=A^{-1}$。

\paragraph{}
方阵的逆阵满足下述运算规律:
\begin{enumerate}
  \item 若$A$可逆,则$A^{-1}$亦可逆,且$(A^{-1})^{-1} = A$。
  \item 若$A$可逆,数$\lambda\neq 0$,则$\lambda A$可逆,且$(\lambda A)^{-1} = \frac{1}{\lambda}A^{-1}$。
  \item 若$A,B$为同阶矩阵且均可逆,则$AB$亦可逆,且
  \begin{equation*}
    (AB)^{-1} = B^{-1}A^{-1}。
  \end{equation*}
  \item 若$A$可逆,则$A^T$亦可逆,且$(A^T)^{-1}=(A^{-1})^T$。
  \item 若$A$可逆,
  \begin{equation*}
    A^0 = E, \; A^{-k} = (A^{-1})^k,
  \end{equation*}
  其中$k$为正整数。这样,当$A$可逆,$\lambda, \mu$为整数时,有
  \begin{equation*}
    A^\lambda A^\mu = A^{\lambda + \mu}, \; (A^\lambda)^\mu = A^{\lambda\mu}.
  \end{equation*}
\end{enumerate}

\subsection{矩阵的分块法}
\paragraph{}
