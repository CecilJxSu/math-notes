\paragraph{}
\begin{enumerate}
  \item 引入矩阵的初等变换,建立矩阵的秩的概念,以及秩的性质。
  \item 利用矩阵的秩,讨论线性方程组无解、有唯一解或有无穷多解的充分必要条件。
  \item 介绍用初等变换解线性方程组的方法。
\end{enumerate}

\subsection{矩阵的初等变换}
\subsubsection{初等变换的引例}
\paragraph{}
消元法解线性方程组例子

\paragraph{}
\textbf{引例~~}求解线性方程组
\begin{equation*}
  \left\{\begin{array}{c}
            2x_1 - \enspace x_2 - \enspace x_3 + \enspace x_4 = 2, \; \circled{1} \\
    \enspace x_1 + \enspace x_2 -         2x_3 + \enspace x_4 = 4, \; \circled{2} \\
            4x_1 -         6x_2 +         2x_3 -         2x_4 = 4, \; \circled{3} \\
            3x_1 +         6x_2 -         9x_3 +         7x_4 = 9, \; \circled{4}
  \end{array} \right. \tag{$A$}
\end{equation*}

\paragraph{}
\textbf{解~~}
\begin{alignat*}{3}
  (A)\, & \xrightarrow[\circled{\tiny 3} \,\div\, 2]{\circled{\tiny 1} \,\leftrightarrow\, \circled{\tiny 2}} & \;
  \left\{\begin{array}{ll}
    \enspace x_1 + \enspace x_2 -         2x_3 + \enspace x_4 = 4, & \circled{1} \\
            2x_1 - \enspace x_2 - \enspace x_3 + \enspace x_4 = 2, & \circled{2} \\
            2x_1 -         3x_2 + \enspace x_3 - \enspace x_4 = 2, & \circled{3} \\
            3x_1 +         6x_2 -         9x_3 +         7x_4 = 9, & \circled{4}
  \end{array} \right. \tag{$A_1$} \\
  % ----------------------------------------------------------------------------
  & \xrightarrow[\circled{\tiny 4} \,-\, 3\circled{\tiny 1}]{\substack{\circled{\tiny 2} \,-\, \circled{\tiny 3} \\ \circled{\tiny 3} \,-\, 2\circled{\tiny 1}}} & \;
  \left\{\begin{array}{rl}
     x_1 + x_2 - 2x_3 + \enspace x_4 = \hspace{0.8em} 4, & \circled{1} \\
          2x_2 - 2x_3 +         2x_4 = \hspace{0.8em} 0, & \circled{2} \\
         -5x_2 + 5x_3 -         3x_4 =               -6, & \circled{3} \\
          3x_2 - 3x_3 +         4x_4 =               -3, & \circled{4}
  \end{array} \right. \tag{$A_2$} \\
  % ----------------------------------------------------------------------------
  & \xrightarrow[\substack{\circled{\tiny 3} \,+\, 5\circled{\tiny 2} \\ \circled{\tiny 4} \,-\, 3\circled{\tiny 2}}]{\circled{\tiny 2} \,\times\, \frac{1}{2}} & \;
  \left\{\begin{array}{rl}
     x_1 + x_2 -         2x_3 + \enspace x_4 = \hspace{0.8em} 4, & \circled{1} \\
           x_2 - \enspace x_3 + \enspace x_4 = \hspace{0.8em} 0, & \circled{2} \\
                                        2x_4 = -6, & \circled{3} \\
                                         x_4 = -3, & \circled{4}
  \end{array} \right. \tag{$A_3$} \\
  % ----------------------------------------------------------------------------
  & \xrightarrow[\circled{\tiny 4} \,-\, 2\circled{\tiny 3}]{\circled{\tiny 3} \,\leftrightarrow\, \circled{\tiny 4}} & \;
  \left\{\begin{array}{rrrl}
     x_1 + & x_2 - 2x_3 + & x_4 = \hspace{0.8em} 4, & \circled{1} \\\cdashline{1-1}[1pt/1pt]
     \multicolumn{1}{l;{1pt/1pt}}{} & x_2 - \enspace x_3 + & x_4 = \hspace{0.8em} 0, & \circled{2} \\\cdashline{2-2}[1pt/1pt]
     \multicolumn{2}{l;{1pt/1pt}}{} & x_4 = -3, & \circled{3} \\\cdashline{3-4}[1pt/1pt]
     & & 0 = \hspace{0.8em} 0, & \circled{4}
  \end{array} \right. \tag{$A_4$}
\end{alignat*}

\subsubsection{初等变换的定义}
\paragraph{}
\textbf{定义~~}下面三种变换称为矩阵的\textbf{初等行变换}:
\begin{enumerate}
  \item 对调两行(对调$i,j$两行,记作$r_i \leftrightarrow r_j$);
  \item 以数$k \neq 0$乘某一行中的所有元素(第$i$行乘$k$,记作$r_i \times k$);
  \item 把某一行所有元素的$k$倍加到另一行对应的元素上去(第$j$行的$k$倍,加到第$i$行上,记作$r_i + kr_j$)。
\end{enumerate}
相同地,可得到\textbf{初等列变换}的定义。

\paragraph{}
矩阵$A$经过有限次初等变换变成矩阵$B$,就称\textbf{矩阵$A$与$B$等价},记作$A\sim B$。仅有行/列变换,可记作$A\overset{r}{\sim}B$或$A\overset{c}{\sim}B$

\paragraph{}
矩阵之间的等价关系具有下列性质:
\begin{enumerate}
  \item \textbf{反身性~~$A\sim A$};
  \item \textbf{对称性~~若$A\sim B$,则$B\sim A$};
  \item \textbf{传递性~~若$A\sim B, B\sim C$,则$A\sim C$}。
\end{enumerate}

\subsubsection{行阶梯形矩阵}
\paragraph{}
\textbf{行阶梯形矩阵~~}可画出一条阶梯线,线的下方全为$0$,每个台阶只有一行,台阶数即是非零行的行数。阶梯线的竖线后面的第一个元素为非零元。例如,
\begin{equation*}
\left[\begin{array}{rrrrr}
  {\color{red!50}1} & 1 & -2 & 1 & 4 \\\cdashline{1-1}[1pt/1pt]
  \multicolumn{1}{r;{1pt/1pt}}{{\color{gray!50}0}} & {\color{red!50}1} & -1 & 1 & 0 \\\cdashline{2-3}[1pt/1pt]
  {\color{gray!50}0} & {\color{gray!50}0} & \multicolumn{1}{r;{1pt/1pt}}{{\color{gray!50}0}} & {\color{red!50}1} & -3 \\\cdashline{4-5}[1pt/1pt]
  {\color{gray!50}0} & {\color{gray!50}0} & {\color{gray!50}0} & 0 & {\color{gray!50}0}
\end{array}\right]
\end{equation*}

\paragraph{}
\textbf{行最简形矩阵~~}非零行的第一个非零元为$1$,且这些非零元所在的列的其它元素都为$0$。例如,
\begin{equation*}
  \left[\begin{array}{rrrrr}
    {\color{red!50}1} & {\color{blue!50}0} & -1 & {\color{blue!50}0} & 4 \\\cdashline{1-1}[1pt/1pt]
    \multicolumn{1}{r;{1pt/1pt}}{{\color{gray!50}0}} & {\color{red!50}1} & -1 & {\color{blue!50}0} & 3 \\\cdashline{2-3}[1pt/1pt]
    {\color{gray!50}0} & {\color{gray!50}0} & \multicolumn{1}{r;{1pt/1pt}}{{\color{gray!50}0}} & {\color{red!50}1} & -3 \\\cdashline{4-5}[1pt/1pt]
    {\color{gray!50}0} & {\color{gray!50}0} & {\color{gray!50}0} & {\color{gray!50}0} & {\color{gray!50}0}
  \end{array}\right]
\end{equation*}

\paragraph{}
对于任何矩阵$A_{m\times n}$,总可经过有限次初等行变换把它变为行阶梯形矩阵和行最简形矩阵(归纳法可证)。

\paragraph{}
\textbf{标准形~~}矩阵的左上角是一个单位矩阵,其余元素全为$0$。例如,
\begin{equation*}
  \left[{\color{gray!50} \begin{array}{ccccc}
    {\color{red!50}1} & 0 & 0 & 0 & 0 \\
    0 & {\color{red!50}1} & 0 & 0 & 0 \\
    0 & 0 & {\color{red!50}1} & 0 & 0 \\
    0 & 0 & 0 & 0 & 0
  \end{array}}\right]
\end{equation*}

\paragraph{}
对于$m\times n$矩阵$A$,总可经过初等变换把它化为标准形
\begin{equation*}
  F = \left[\begin{array}{cc}
    E_r & O \\
    O & O
  \end{array}\right]_{m\times n}
\end{equation*}

\subsubsection{初等变换的性质}\label{sec:初等变换的性质}
\paragraph{}
\textbf{定理1~~}设$A$与$B$为$m\times n$矩阵,那么:
\begin{enumerate}
  \item $A \overset{r}{\sim} B$的充分必要条件是存在$m$阶可逆矩阵$P$,使$PA = B$;
  \item $A \overset{c}{\sim} B$的充分必要条件是存在$n$阶可逆矩阵$Q$,使$AQ = B$;
  \item $A \sim B$的充分必要条件是存在$m$阶可逆矩阵$P$,以及$n$阶可逆矩阵$Q$,使$PAQ = B$。
\end{enumerate}

\paragraph{}
证明参考\secref{sec:定理1的证明}

\subsubsection{初等矩阵的定义}
\paragraph{}
由单位阵$E$经过一次初等变换得到的矩阵称为\textbf{初等矩阵}。
\begin{enumerate}
  \item 单位阵中第$i,j$两行对调(或第$i,j$两列对调),得初等矩阵
  \begin{equation*}
    E(i,j) = \left[\begin{array}{ccccccccccc}
      1 & & & & & & & & & & \\
        & \ddots & & & & & & & & & \\
        &        & 1 & & & & & & & & \\
        &        &   & 0 & & \cdots & &  1 & & & \\
        &        &   &   & 1 & & & & & & \\
        &        &   & \vdots  &   & \ddots & & \vdots & & & \\
        &        &   &   &   &        & 1 & & & & \\
        &        &   & 1 &   &    \cdots    &  & 0 &  & & \\
        &        &   &   &   &        &   &   & 1 &  & \\
        &        &   &   &   &        &   &   &   & \ddots & \\
        &        &   &   &   &        &   &   &   &        & 1
    \end{array}\right]\begin{array}{c}
      \\
      \\
      \\
      \leftarrow\text{第$i$行} \\
      \\
      \\
      \\
      \leftarrow\text{第$j$行} \\
      \\
      \\
      \\
    \end{array}
  \end{equation*}
  矩阵$A$的第$i,j$行交换,相当于用$m$阶初等矩阵$E_m(i,j)$左乘矩阵$A_{m\times n}$。\\
  矩阵$A$的第$i,j$列交换,相当于用$n$阶初等矩阵$E_n(i,j)$右乘矩阵$A_{m\times n}$。
  \item 以数$k\neq 0$乘单位阵的第$i$行(或第$i$)列,得初等矩阵
  \begin{equation*}
    E(i(k)) = \left[\begin{array}{ccccccc}
      1 & & & & & & \\
        & \ddots & & & & & \\
        &        & 1 & & & & \\
        &        &   & k & & & \\
        &        &   &   & 1 & & \\
        &        &   &   &   & \ddots & \\
        &        &   &   &   &        & 1 \\
    \end{array}\right]\begin{array}{c}
        \\
        \\
        \\
        \leftarrow\text{第$i$行}\\
        \\
        \\
        \\
    \end{array}
  \end{equation*}
  矩阵$A$的第$i$行乘以$k$倍,相当于用$m$阶初等矩阵$E_m(i(k))$左乘矩阵$A_{m\times n}$。\\
  矩阵$A$的第$i$列乘以$k$倍,相当于用$n$阶初等矩阵$E_n(i(k))$右乘矩阵$A_{m\times n}$。
  \item 以$k$乘$E$的第$j$行加到第$i$行上,或者对列进行类似操作,得初等矩阵
  \begin{equation*}
    E(i+j(k)) = \left[\begin{array}{ccccccc}
      1 & & & & & & \\
        & \ddots & & & & & \\
        &        & 1 & \cdots & k & & \\
        &        &   & \ddots & \vdots  & & \\
        &        &   &        & 1 & & \\
        &        &   &        &   & \ddots & \\
        &        &   &        &   &        & 1 \\
    \end{array}\right]\begin{array}{c}
      \\
      \\
      \leftarrow\text{第$i$行}\\
      \\
      \leftarrow\text{第$j$行}\\
      \\
      \\
    \end{array}
  \end{equation*}
  以$E_m(i+j(k))$左乘矩阵$A$,相当于把$A$的第$j$行乘$k$加到第$i$行上。相同地,列也具有类似性质。
\end{enumerate}

\paragraph{}
归纳以上的讨论,可得

\paragraph{}
\textbf{性质1~~}设$A$是一个$m\times n$矩阵,对$A$施行一次初等行变换,相当于在$A$的左边乘以相应的$m$阶初等矩阵;对$A$施行一次初等列变换,相当于在$A$的右边乘以相应的$n$阶初等矩阵。

\paragraph{}
\textbf{性质2~~}方阵$A$可逆的充分必要条件是存在有限个初等矩阵$P_1,P_2,\cdots,P_l$,使$A = P_1P_2\cdots P_l$。

\subsubsection{定理$1$的证明}\label{sec:定理1的证明}
\paragraph{}
\secref{sec:初等变换的性质}的证明:

\paragraph{}
$(1.)$依据$A\overset{r}{\sim}B$的定义和初等矩阵的性质,有
\begin{align*}
  A \overset{r}{\sim}B \Leftrightarrow &\; A\text{经有限次初等行变换变成}B \\
  \Leftrightarrow &\;\text{存在有限个$m$阶初等矩阵}P_1,P_2,\cdots,P_l\text{,使}P_l\cdots P_2P_1A = B \\
  \Leftrightarrow &\;\text{存在$m$阶可逆矩阵$P$,使}PA=B.
\end{align*}
类似可证明$(2.)$和$(3.)$。

\paragraph{}
\textbf{推论~~}方阵$A$可逆的充分必要条件是$A\overset{r}{\sim}E$。

\subsection{矩阵的秩}
\paragraph{}

\subsection{线性方程组的解}
\paragraph{}
