\subsection{函数的和、差、积、商的求导法则}
\paragraph{}
\textbf{定理1~~}如果函数 $u = u(x)$ 及  $v = v(x)$ 都在点 $x$ 具有导数,那么它们的和、差、积、商(除分母为零的点外)都在点 $x$ 具有导数,且

\begin{align}
\left[u(x) \pm  v(x)\right]' &= u'(x) \pm v'(x); \\
\left[u(x)v(x)\right]' &= u'(x)v(x) + u(x)v'(x); \\
\left[\frac{u(x)}{v(x)}\right]' &= \frac{u'(x)v(x) - u(x)v'(x)}{v^2(x)} \; (v(x) \neq 0).
\end{align}

\paragraph{}
定理 1 中的法则(1)、(2)可推广到任意有限个可导函数的情形,即

\begin{align}
(u + v - w)' &= u' + v' - w'; \\
(uvw)' &= u'vw + uv'w + uvw'; \\
(Cu)' &= Cu' \text{~~(C 为常数)}.
\end{align}

\subsection{反函数的求导法则}
\paragraph{}
\textbf{定理2~~}如果函数 $x = f(y)$ 在区间 $I_y$ 内单调、可导且 $f'(y) \neq 0$,则它的反函数 $y = f^{-1}(x)$ 在区间 $I_x = \{x|x = f(x), y \in I_y\}$ 内也可导,且

\begin{equation}
[f^{-1}(x)]' = \frac{1}{f'(y)} \text{~~或~~} \frac{dy}{dx} = \frac{1}{\frac{dx}{dy}}.
\end{equation}

\subsection{复合函数的求导法则}
\paragraph{}
\textbf{定理3~~}如果 $u = g(x)$ 在点 $x$ 可导,而 $y = f(u)$ 在点 $u = g(x)$ 可导,则复合函数 $y = f[g(x)]$ 在点 $x$ 可导,且其导数为

\begin{equation}
\frac{dy}{dx} = f'(u) \cdot g'(x) \text{~~或~~} \frac{dy}{dx} = \frac{dy}{du} \cdot \frac{du}{dx}.
\end{equation}

\subsection{基本求导法则与导数公式}
\subsubsection{常数和基本初等函数的导数公式}

\begin{equation}
  \begin{aligned}[c]
    (C)' &= 0; \\
    (\sin x)' &= \cos x; \\
    (\tan x)' &= \sec^2 x; \\
    (\sec x)' &= \sec x \tan x; \\
    (a^x)' &= a^x\ln a; \\
    (\log_a x)' &= \frac{1}{x \ln a}; \\
    (\arcsin x)' &= \frac{1}{\sqrt{1 - x^2}}; \\
    (\arctan x)' &= \frac{1}{1 + x^2};
  \end{aligned}
  \qquad \qquad
  \begin{aligned}[c]
    (x^\mu)' &= \mu x^{\mu - 1}; \\
    (\cos x)' &= -\sin x; \\
    (\cot x)' &= - \csc^2 x; \\
    (\csc x)' &= -\csc x \cot x; \\
    (e^x)' &= e^x; \\
    (\ln x)' &= \frac{1}{x}; \\
    (\arccos x)' &= - \frac{1}{\sqrt{1 - x^2}}; \\
    (\arccot x)' &= - \frac{1}{1 + x^2}.
  \end{aligned}
\end{equation}

\subsubsection{函数的和、差、积、商的求导法则}
\paragraph{}
设 $u = u(x), v = v(x)$ 都可导,则

\begin{equation}
\begin{aligned}[c]
  (u \pm v)' &= u' \pm v'; \\
  (uv)' & = u'v + uv';
\end{aligned}
\qquad \qquad
\begin{aligned}[c]
  (Cu)' = Cu' \text{~~(C 是常数)}; \\
  \left(\frac{u}{v}\right)' = \frac{u'v - uv'}{v^2}(v \neq 0)
\end{aligned}
\end{equation}

\subsubsection{反函数的求导法则}
\paragraph{}
设 $x = f(y)$ 在区间 $I_y$ 内单调、可导且 $f'(y) \neq 0$,则它的反函数 $y = f^{-1}(x)$ 在区间 $I_x = f(I_y)$ 内也可导,且

\begin{equation}
[f^{-1}(x)]' = \frac{1}{f'(y)} \text{~~或~~} \frac{dy}{dx} = \frac{1}{\frac{dx}{dy}}.
\end{equation}

\subsubsection{复合函数的求导法则}
\paragraph{}
设 $y = f(u)$,而 $u = g(x)$ 且 $f(u)$ 及 $g(x)$ 都可导,则复合函数 $y = f[g(x)]$ 的导数为

\begin{equation}
\frac{dy}{dx} = \frac{dy}{du} \cdot \frac{du}{dx} \text{~~或~~} y'(x) = f'(u) \cdot g'(x).
\end{equation}
