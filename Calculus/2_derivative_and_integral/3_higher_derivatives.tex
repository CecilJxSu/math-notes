\subsection{高阶导数定义}
\paragraph{}
一般的,函数 $y = f(x)$ 的导数 $y' = f'(x)$ 仍然是 $x$ 的函数。我们把 $y' = f'(x)$ 的\textbf{导数}叫做函数 $y = f(x)$ 的\uwave{二阶导数},记作 $y''$ 或 $\frac{d^2y}{dx^2}$,即

\begin{equation}
y'' = (y')' \text{~~或~~} \frac{d^2y}{dx^2} = \frac{d}{dx}(\frac{dy}{dx}).
\end{equation}

\paragraph{}
一般的,$(n - 1)$ 阶导数的导数叫做 \uwave{$n$ 阶导数},记作

\begin{gather}
y^{(n)} \text{~~或~~} \frac{d^ny}{dx^n}.
\end{gather}

\paragraph{}
如果函数 $f(x)$ 在点 $x$ 处具有 $n$ 阶导数,那么 $f(x)$ 在点 $x$ 的某一邻域内必定具有一切低于 $n$ 阶的导数。

\subsection{初等函数的 $n$ 阶导数}

\begin{align}
(e^x)^{(n)} &= e^x \\
(\sin x)^{(n)} &= \sin(x + n \cdot \frac{\pi}{2}) \\
(\cos x)^{(n)} &= \cos(x + n \cdot \frac{\pi}{2}) \\
[\ln(1 + x)]^{(n)} &= (-1)^{n-1}\frac{(n-1)!}{(1+x)^n} \\
(x^{\mu})^{(n)} &= \mu(\mu - 1)(\mu - 2) \cdots (\mu - n + 1)x^{\mu - n} \\
(u \pm v)^{(n)} &= u^{(n)} \pm v^{(n)} \\
(uv)^{(n)} &= u^{(n)}v + nu^{(n - 1)}v' + \frac{n(n-1)}{2!}u^{(n-2)}v'' + \cdots \\
 & \quad + \frac{n(n-1) \cdots (n - k + 1)}{k!} u^{(n - k)}v^{(k)} + \cdots + uv^{(n)} \\
 & = \sum_{k = 0}^{n} C_n^ku^{(n-k)}v^{(k)}. \textbf{~~(Leibniz 公式)}
\end{align}
