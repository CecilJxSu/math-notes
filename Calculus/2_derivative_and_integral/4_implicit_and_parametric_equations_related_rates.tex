\subsection{隐函数的导数}
\paragraph{}
\textbf{显函数:}等号左端是因变量的符号,而右端是含有自变量的式子,当自变量取定义域内任一值时,由这式子能确定对应的函数值。如:$y = \sin x$。

\paragraph{}
\textbf{隐函数:}如果变量 $x$ 和 $y$ 满足一个方程 $F(x,y) = 0$,在一定条件下,当 $x$ 取某区间内的任一值时,相应的总有满足这方程的\uwave{唯一的} $y$ 值存在,那么方程 $F(x,y) = 0$ 在该区间内确定了一个隐函数。如:$x + y^3 - 1 = 0$。

\paragraph{}
\textbf{隐函数的显化:}把一个隐函数化成显函数。如:$x + y^3 - 1 = 0 \; \to \; y = \sqrt[3]{1 - x}$。

\paragraph{}
\textbf{例子:}求椭圆 $\frac{x^2}{16} + \frac{y^2}{9} = 1$ 在点 $(2, \frac{3}{2}\sqrt{3})$ 处的切线方程。
\paragraph{}
\textbf{解~~}由导数的几何意义知道,所求切线的斜率为
\begin{equation*}
k = y'|_{x = 2}.
\end{equation*}
\paragraph{}
椭圆方程的两边分别对 $x$ 求导(\textbf{y 相当于复合函数,按复合函数求导}),有
\begin{equation*}
\frac{x}{8} + \frac{2}{9}y \cdot \frac{dy}{dx} = 0.
\end{equation*}
\paragraph{}
从而 $\frac{dy}{dx} = -\frac{9x}{16y}.$
\paragraph{}
当 $x = 2$ 时,$y = \frac{3}{2}\sqrt{3}$,代入上式得
\begin{equation*}
\frac{dy}{dx}|_{x = 2} = -\frac{\sqrt{3}}{4}.
\end{equation*}
\paragraph{}
于是所求的切线方程为
\begin{align*}
y - \frac{3}{2}\sqrt{3} &= -\frac{\sqrt{3}}{4}(x - 2), \text{~~即}\\
\sqrt{3}x + 4y - 8\sqrt{3} &= 0.
\end{align*}

\paragraph{}
\textbf{对数求导法~~}在某些场合,先在 $y = f(x)$ 的两边取对数,然后再求出 $y$ 的导数,如:$y = x^{\sin x} (x > 0)$ 的导数,可以先对两边取对数,即 $\ln y = \ln x^{\sin x} = \sin x \cdot \ln x$,然后才对两边求导。

\paragraph{}
对于一般形式的幂指函数

\begin{equation}
y = u^v (u > 0)
\end{equation}

\paragraph{}
如果 $u = (x), v = v(x)$ 都可导,则,可以表示为:$y = e^{v \ln u}.$

\subsection{由参数方程所确定的函数的导数}
\paragraph{}
研究物体抛射运动轨迹,水平和垂直方向的分解:
\begin{equation}
\left\{
  \begin{array}{l}
    x = v_1 t, \\
    y = v_2 t - \frac{1}{2}gt^2
  \end{array}
\right.
\end{equation}

\paragraph{}
消去参数 $t$,有:

\begin{equation}
y = \frac{v_2}{v_1}x - \frac{g}{2v_1^2}x^2.
\end{equation}

一般的,若\uwave{参数方程}

\begin{equation}
\left\{
\begin{array}{l}
  x = \varphi(t), \\
  y = \psi(t)
\end{array}
\right.
\end{equation}

\paragraph{}
确定 $y$ 与 $x$ 间的函数关系,则称此函数关系所表达的函数为由上面的参数方程确定的函数。

\paragraph{}
如果函数 $x = \varphi(t)$ 具有单调连续反函数 $t = \varphi^{-1}(x)$,且此反函数能与函数 $y = \psi(t)$ 构成复合函数,那么由参数方程所确定的函数可以看成是由函数 $y = \psi(t), t = \varphi^{-1}(x)$ 复合而成的函数 $y = \psi[\varphi^{-1}(x)]$。假设函数 $x = \varphi(t), y = \psi(t)$ 都可导,根据复合函数和反函数的求导法则,有

\begin{align}
\frac{dy}{dx} = \frac{dy}{dt} \cdot \frac{dt}{dx} &= \frac{dy}{dt} \cdot \frac{1}{\frac{dx}{dt}} = \frac{\psi'(t)}{\varphi'(t)} \text{~~即} \\
\frac{dy}{dx} &= \frac{\psi'(t)}{\varphi'(t)}. \\
\text{也可写成~~} \frac{dy}{dx} &= \frac{\frac{dy}{dt}}{\frac{dx}{dt}}.
\end{align}

\paragraph{}
如果 $x = \varphi(t), y = \psi(t)$ 二阶可导,则二阶导数公式为

\begin{equation}
\frac{d^2y}{dx^2} = \frac{\psi''(t)\varphi'(t) - \psi'(t)\varphi''(t)}{\varphi'^3(t)}
\end{equation}

\subsection{相关变化率}
\paragraph{}
设 $x = x(t)$ 及 $y = y(t)$ 都是可导函数,而变量 $x$ 与 $y$ 间存在某种关系,从而变化率 $\frac{dx}{dt}$ 与 $\frac{dy}{dt}$ 间也存在一定关系。这两个相互依赖的变化率称为\uwave{相关变化率}。

\paragraph{}
相关变化率问题研究这两个变化率之间的关系,以便从其中一个变化率求出另外一个变化率。
