\subsection{导数的引例}
\begin{enumerate}
  \item 速度问题:直线运动的速度
  \item 切线问题:曲线的切线
\end{enumerate}

\subsection{导数的定义}
\subsubsection{函数在一点处的导数与导函数}
\paragraph{}
\textbf{定义~~}设函数 $y = f(x)$ 在点 $x_0$ 的某个邻域内有定义,当自变量 $x$ 在 $x_0$ 处取得增量 $\Delta x$(点 $x_0 + \Delta x$ 仍在该邻域内)时,相应的函数取得增量 $\Delta y = f(x_0 + \Delta x) - f(x_0)$;如果 $\Delta y$ 与 $\Delta x$ 之比当 $\Delta x \to 0$ 时的极限存在,则称函数 $y = f(x)$ 在点 $x_0$ 处\textbf{可导},并称这个极限为函数 $y = f(x)$ 在点 $x_0$ 处的\textbf{导数},记为 $f'(x_0)$,即

\begin{equation}
f'(x_0) = \lim_{\Delta x \to 0} \frac{\Delta y}{\Delta x} = \lim_{\Delta x \to 0} \frac{f(x_0 + \Delta x) - f(x_0)}{\Delta x},
\end{equation}

\paragraph{}
也可记作 $y'|_{x = x_0}, \frac{dy}{dx}|_{x = x_0}$ 或 $\frac{df(x)}{dx}|_{x=x_0}$。

\paragraph{}
也可取不同的形式,常见的有:

\begin{gather}
f'(x_0) = \lim_{h \to 0} \frac{f(x_0 + h) - f(x_0)}{h} , \\
\text{和} \\
f'(x_0) = \lim_{x \to x_0}\frac{f(x) - f(x_0)}{x - x_0}
\end{gather}

\paragraph{}
极限不存在,就说函数 $y = f(x)$ 在点 $x_0$ 处不可导。由于 $\Delta x \to 0$ 时,比式 $\frac{\Delta y}{\Delta x} \to \infty$,函数 $y = f(x)$ 在点 $x_0$ 处导数无穷大。

\paragraph{}
如果函数 $y = f(x)$ 在开区间 $I$ 内的每点处都可导,就称函数 $f(x)$ 在开区间 $I$ 内可导,这就构成了一个新的函数,叫做原来函数 $y = f(x)$ 的导函数,记作 $y', f'(x), \frac{dy}{dx}$ 或 $\frac{df(x)}{dx}$。

\paragraph{}
把 $x_0$ 换成 $x$,即得导函数的定义式:

\begin{gather}
y' = \lim_{x \to 0}\frac{f(x + \Delta x) - f(x)}{\Delta x} , \\
\text{或} \\
f'(x) = \lim_{h \to 0} \frac{f(x + h) - f(x)}{h}.
\end{gather}

\paragraph{}
导函数 $f'(x)$ 简称导数,而 $f'(x_0)$ 是 $f(x)$ 在 $x_0$ 处的导数

\subsubsection{常见函数的导数公式}

\begin{align}
C' &= 0 \qquad \text{(C 为常数)} \\
(x^a)' &= ax^{a-1} \qquad \text{$(a \in Q)$} \\
(\sin x)' &= \cos x \\
(\cos x)' &= - \sin x \\
(e^x)' &= e^x \\
(a^x)' &= a^x \ln a \\
(\ln x)' &= \frac{1}{x} \\
(\log_a x)' &= \frac{1}{x}\log_a e
\end{align}

\subsubsection{单侧导数}
\paragraph{}
极限存在的充分必要条件是左、右极限都\textbf{存在且相等}:

\begin{gather}
\lim_{h \to 0^-}\frac{f(x_0 + h) - f(x_0)}{h} \qquad \text{(左导数)} \\
\lim_{h \to 0^+}\frac{f(x_0 + h) - f(x_0)}{h} \qquad \text{(右导数)}
\end{gather}

\subsection{导数的几何意义}
\paragraph{}
函数 $y = f(x)$ 在点 $x_0$ 处的导数 $f'(x_0)$ 在几何上表示曲线 $y = f'(x)$ 在点 $M(x_0, f(x_0))$ 处的切线的斜率,即:

\begin{equation}
f'(x_0) = \tan \alpha ,
\end{equation}

\paragraph{}
其中 $\alpha$ 是切线的倾角。

\begin{figure}[H]
  \centering
    % 导数的几何意义
\begin{tikzpicture}
  \begin{axis}[xmin=0, xmax=3,ymin=0,ymax=3, grid=none,
    xtick=\empty,ytick=\empty, font=\large, axis lines=middle,
    smooth, xlabel={$x$}, ylabel={$y$}]

    % 曲线
    \addplot[draw=red,domain=0.3:2.5] {(x - 1)^2 + 0.5};
    \node [above] at (axis cs:2,2.5) {$y = f(x_0)$};
    \draw[fill] (1.5,0.75) circle [radius=0.02];
    % 切线
    \addplot[draw=blue,domain=0.3:2.2] {x - 0.75};
    \node [right] at (axis cs:2.2,1.45) {$T$};

    % 辅助线和点
    \addplot[dashed, draw=cyan, mark=none] coordinates {(1.5, 0) (1.5, 0.75)};
    \node [above] at (axis cs:1.5,0.75) {$M$};

    % 倾角
    \draw (1,0) arc (0:45:0.25);
    \node [right] at (axis cs:1,0.1) {$\alpha$};
  \end{axis}
  % 原点
  \node [below left] at (0,0) {$O$};
  \node [below] at (3.45,0) {$x_0$};
\end{tikzpicture}

    \caption{几何意义}
    \label{derivative_geometric_meaning}
\end{figure}

\paragraph{}
过点 $M(x_0, y_0)$ 和切线斜率 $f'(x_0)$,根据直线的点斜式,得到\textbf{切线方程}:

\begin{equation}
y - y_0 = f'(x_0)(x - x_0)
\end{equation}

\paragraph{}
过点 $M(x_0, y_0)$ 和法线斜率 $-\frac{1}{f'(x_0)}, f'(x_0) \neq 0$,根据直线的点斜式,得到\textbf{法线方程}:

\begin{equation}
y - y_0 = - \frac{1}{f'(x_0)}(x - x_0), \; f'(x_0) \neq 0
\end{equation}

\subsection{函数可导性与连续性的关系}
\paragraph{}
设函数 $y = f(x)$ 在点 $x$ 处可导,即

\begin{equation}
\lim_{\Delta x \to 0} \frac{\Delta y}{\Delta x} = f'(x)
\end{equation}

\paragraph{}
存在。由具有极限的函数与无穷小的关系知道,

\begin{equation}
\frac{\Delta y}{\Delta x} = f'(x) + \alpha ,
\end{equation}

\paragraph{}
其中 $\alpha$ 为当 $\Delta x \to 0$ 时的无穷小。上式两边同乘以 $\Delta x$,得

\begin{equation}
\Delta y = f'(x) \Delta x + \alpha\Delta x.
\end{equation}

\paragraph{}
由此可见,当 $\Delta x \to 0$ 时,$\Delta y \to 0$。这就是说,函数 $y = f(x)$ 在点 $x$ 处是连续的。

\paragraph{}
所以,如果函数 $y = f(x)$ 在点 $x$ 处可导,则函数在该点必连续。

\paragraph{}
函数在某点的连续是函数在该点可导的必要条件,但不是充分条件(\textbf{可导即连续,但连续不一定可导})。

\paragraph{}
充分条件:条件 A $\to$ 结论 B,A 则为 B 充分条件

\paragraph{}
必要条件:结论 B $\to$ 条件 A,A 则为 B 必要条件
