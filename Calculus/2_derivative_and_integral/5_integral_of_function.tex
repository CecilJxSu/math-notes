\subsection{微分的定义}
\paragraph{}
先分析一个具体问题。一块正方形金属薄片受温度变化的影响,其边长由 $x_0$ 变到 $x_0 + \Delta x$,问此薄片的面积改变了多少?

\paragraph{}
设此薄片的边长为 $x$,面积为 $A$,则 $A$ 与 $x$ 存在函数关系:$A = x^2$。薄片受温度变化的影响时面积的改变量,可以看成是当自变量 $x$ 自 $x_0$ 取得增量 $\Delta x$ 时,函数 $A = x^2$ 相应的增量 $\Delta A$,即

\begin{equation}
\Delta A = (x_0 + \Delta x)^2 - x_0^2 = 2x_0\Delta x + (\Delta x)^2.
\end{equation}

\begin{figure}[H]
  \centering
    % 微分例子
\begin{tikzpicture}
  \draw (0,0) rectangle (5, 5);
  \draw[pattern=north east lines] (0,4) rectangle (5, 5);
  \draw[pattern=north west lines] (4,0) rectangle (5, 5);

  \node at (2,2) {$A = x_0^2$};
  \node[fill=white] at (2,4.5) {$x_0 \Delta x$};
  \node[fill=white, rotate=-90] at (4.5,2) {$x_0 \Delta x$};
  \node[fill=white] at (4.5,4.5) {\scriptsize $(\Delta x)^2$};

  \draw (0, 5.1) -- (0, 5.7);
  \draw (4, 5.1) -- (4, 5.7);
  \draw (5, 5.1) -- (5, 5.7);
  \draw [Latex-Latex] (0, 5.4) -- (4,5.4);
  \node[fill=white] at (2,5.4) {$x_0$};
  \draw [<->] (4, 5.4) -- (5, 5.4);
  \node[fill=white] at (4.5,5.4) {$\Delta x$};

  \draw (5.1, 5) -- (5.7, 5);
  \draw (5.1, 4) -- (5.7, 4);
  \draw (5.1, 0) -- (5.7, 0);
  \draw [<->] (5.4, 5) -- (5.4, 4);
  \node[fill=white,rotate=-90] at (5.4,4.5) {$\Delta x$};
  \draw [Latex-Latex] (5.4, 4) -- (5.4, 0);
  \node[fill=white,rotate=-90] at (5.4,2) {$x_0$};
\end{tikzpicture}

    \caption{金属薄片面积受温度变化}
    \label{金属薄片面积受温度变化}
\end{figure}

\paragraph{}
$\Delta A$ 的第一部分 $2x_0\Delta x$ 是 $\Delta x$ 的线性函数,图中斜线的两个矩形面积之和;第二部分 $(\Delta x)^2$ 在图中带有交叉斜线的小正方形的面积。当 $\Delta x \to 0$ 时,第二部分 $(\Delta x)^2$ 比 $\Delta x$ 高阶无穷小,即 $(\Delta x)^2 = o(\Delta x)$。因此,如果边长改变的很微小,即 $|\Delta x|$ 很小时,面积的改变量 $\Delta A$ 可近似地用第一部分来代替。

\paragraph{}
\textbf{定义~~}设函数 $y = f(x)$ 在某区间内有定义,$x_0$ 及 $x_0 + \Delta x$ 在这区间内,如果增量

\begin{equation}
\Delta y = f(x_0 + \Delta x) - f(x_0)
\end{equation}

\paragraph{}
可表示为

\begin{equation}
\Delta y = A\Delta x + o(\Delta x),
\end{equation}

\paragraph{}
其中 $A$ 是不依赖于 $\Delta x$ 的常数,那么称函数 $y = f(x)$ 在点 $x_0$ 是\uwave{可微}的,而 $A\Delta x$ 叫做函数 $y = f(x)$ 在点 $x_0$ 相应于自变量增量 $\Delta x$ 的\uwave{微分},记作 $dy$,即

\begin{equation}
dy = A\Delta x.
\end{equation}

\paragraph{}
函数 $f(x)$ 在点 $x_0$ 可微的充分必要条件是函数 $f(x)$ 在点 $x_0$ 可导,且当 $f(x)$ 在点 $x_0$ 可微时,其微分一定是

\begin{equation}
dy = f'(x_0)\Delta x.
\end{equation}

\paragraph{}
当 $f'(x_0) \neq 0$ 时,有

\begin{equation}
\lim_{\Delta x \to 0} \frac{\Delta y}{dy} = \lim_{\Delta x \to 0} \frac{\Delta y}{f'(x_0) \Delta x} = \frac{1}{f'(x_0)}\lim_{\Delta x \to 0}\frac{\Delta y}{\Delta x} = 1.
\end{equation}

\paragraph{}
从而,当 $\Delta x \to 0$ 时,$\Delta y$ 与 $dy$ 是等价无穷小,这时有

\begin{equation}
\Delta y = dy + o(dy).
\end{equation}
