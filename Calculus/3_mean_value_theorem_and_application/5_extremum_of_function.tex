\subsection{函数的极值及其求法}
\subsubsection{定义}
\paragraph{}
\textbf{定义\;}设函数$f(x)$在点$x_0$的某邻域$U(x_0)$内有定义,如果对于去心邻域$\mathring{U}(x_0)$内的任一$x$,有
\begin{equation}
  f(x) < f(x_0) \;\; \text{(或$f(x) > f(x_0)$)},
\end{equation}
那么就称$f(x_0)$是函数$f(x)$的一个\uwave{极大值}(或\uwave{极小值})。

\paragraph{}
函数的极大值与极小值统称为函数的\uwave{极值},使函数取得极值的点称为\uwave{极值点}。极值的概念是局部性的。

\subsubsection{定理}
\paragraph{}
\textbf{定理1(必要条件)\;}设函数$f(x)$在$x_0$处可导,且在$x_0$处取得极值,那么$f'(x_0)=0$。

\paragraph{}
反过来确不一定,比如:$f(x)=x^3$的导数$f'(x)=3x^2, f'(0) = 0$。

\paragraph{}
\textbf{定理2(第一充分条件)\;}设函数$f(x)$在$x_0$处连续,且在$x_0$的某去心邻域$\mathring{U}(x_0,\delta)$内可导。
\begin{enumerate}
  \item 若$x\in (x_0 - \delta, x_0)$时,$f'(x) > 0$,而$x \in (x_0,x_0+\delta)$时,$f'(x) < 0$,则$f(x)$在$x_0$处取得极大值;
  \item 若$x\in(x_0-\delta,x_0)$时,$f'(x)<0$,而$x\in(x_0,x_0+\delta)$时,$f'(x)>0$,则$f(x)$在$x_0$处取得极小值;
  \item 若$x\in\mathring{U}(x_0,\delta)$时,$f'(x)$的符号保持不变,则$f(x)$在$x_0$处没有极值。
\end{enumerate}

\paragraph{}
\textbf{定理3(第二充分条件)\;}设函数$f(x)$在$x_0$处具有二阶导数且$f'(x_0)=0, f''(x_0) \neq 0$,那么
\begin{enumerate}
  \item 当$f''(x_0)<0$时,函数$f(x)$在$x_0$处取得极大值;
  \item 当$f''(x_0)>0$时,函数$f(x)$在$x_0$处取得极小值。
\end{enumerate}

\subsection{最大值最小值问题}
\paragraph{}
在工农业生产、工程技术及科学实验中,常常会遇到这样一类问题:在一定条件下,怎样使“产品最多”、“用料最省”、“成本最低”、“效率最高”等问题,这类问题在数学上有时可归纳为求某一函数(通常称为\uwave{目标函数})的最大值或最小值。

\paragraph{}
最大值和最小值的方法:
\begin{enumerate}
  \item 求出$f(x)$在$(a,b)$内的驻点$x_1,x_2, \cdots, x_m$及不可导点$x'_1,x'_2,\cdots,x'_n$;
  \item 计算$f(x_i)(i=1,2,\cdots,m), \; f(x'_j)(j=1,2,\cdots,n)$及$f(a),f(b)$;
  \item 比较$2$中诸值的大小,其中最大的便是$f(x)$在$[a,b]$上的最大值,最小的便是$f(x)$在$[a,b]$上的最小值。
\end{enumerate}

\paragraph{}
应用例子:
\begin{enumerate}
  \item 折射定律
  \item 经济学的边际成本、边际收入和边际利润
\end{enumerate}
