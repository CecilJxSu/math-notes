% 切线法的 4 种不同情形
\begin{tikzpicture}[scale=0.8]
  \begin{axis}[clip=false,xmin=0, xmax=8,ymin=-4,ymax=4, grid=none,
    xtick=\empty, ytick=\empty, axis lines=middle,
    smooth, xlabel={$x$}, ylabel={$y$}]

    % y' = 3/x
    \addplot[draw=red,domain=1.5:6] {3*ln(x) - 3.5};

    % 实根
    \node [below] at (3.23,0) {$\xi$};

    % A
    \draw [dashed] (1.5,-2.28) -- (1.5,0);
    \draw [fill] (1.5,-2.28) circle [radius=0.05];
    \node [left] at (1.5,-2.28) {$A$};
    \node [above] at (1.5,0) {$a$};

    % 切线
    \addplot[domain=1.5:2.64] {2*x - 5.28};
    \node [above] at (2.64,0) {$x_1$};

    % B
    \draw [dashed] (6,1.88) -- (6,0);
    \draw [fill] (6,1.88) circle [radius=0.05];
    \node [right] at (6,1.88) {$B$};
    \node [below] at (6,0) {$b$};

    \node [below left] at (5,1.88) {$y=f(x)$};

    % 原点
    \node [left] at (0,0) {$O$};
  \end{axis}
\end{tikzpicture}
