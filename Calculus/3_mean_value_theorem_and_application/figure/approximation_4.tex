% 切线法的 4 种不同情形
\begin{tikzpicture}[scale=0.8]
  \begin{axis}[clip=false,xmin=0, xmax=8,ymin=-4,ymax=4, grid=none,
    xtick=\empty, ytick=\empty, axis lines=middle,
    smooth, xlabel={$x$}, ylabel={$y$}]

    % -0.6*(x-1)^0.8
    \addplot[draw=red,domain=1:6] {-0.33*(x-1)^1.8+2.5};

    % 实根
    \node [below left] at (4.248,0) {$\xi$};

    % A
    \draw [dashed] (1,2.5) -- (1,0);
    \draw [fill] (1,2.5) circle [radius=0.05];
    \node [left] at (1,2.5) {$A$};
    \node [below] at (1,0) {$a$};

    \node [below right] at (2.5,2.5) {$y=f(x)$};

    % B
    \draw [dashed] (6,-3.48) -- (6,0);
    \draw [fill] (6,-3.48) circle [radius=0.05];
    \node [right] at (6,-3.48) {$B$};
    \node [above] at (6,0) {$b$};

    % 切线
    \addplot[domain=4.41:6] {-2.17*x + 9.57};
    \node [above] at (4.41,0) {$x_1$};

    % 原点
    \node [left] at (0,0) {$O$};
  \end{axis}
\end{tikzpicture}
