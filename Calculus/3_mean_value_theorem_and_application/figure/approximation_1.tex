% 切线法的 4 种不同情形
\begin{tikzpicture}[scale=0.8]
  \begin{axis}[clip=false,xmin=0, xmax=8,ymin=-4,ymax=4, grid=none,
    xtick=\empty, ytick=\empty, axis lines=middle,
    smooth, xlabel={$x$}, ylabel={$y$}]

    % y' = 0.15*(x-1)^0.5*x + 0.1 * (x-1)^1.5
    \addplot[draw=red,domain=1:5.5] {0.1 * (x-1)^1.5 * x - 2};

    % 实根
    \node [above] at (3.95,0) {$\xi$};

    % A
    \draw [dashed] (1,-2) -- (1,0);
    \draw [fill] (1,-2) circle [radius=0.05];
    \node [left] at (1,-2) {$A$};
    \node [above] at (1,0) {$a$};

    % B
    \draw [dashed] (5.5,3.25) -- (5.5,0);
    \draw [fill] (5.5,3.25) circle [radius=0.05];
    \node [right] at (5.5,3.25) {$B$};
    \node [below] at (5.5,0) {$b$};

    \node [below left] at (5,3.25) {$y=f(x)$};

    % 切线
    \addplot[domain=4.3:5.5] {2.7*x - 11.6};
    \node [below] at (4.3,0) {$x_1$};

    % 原点
    \node [left] at (0,0) {$O$};
  \end{axis}
\end{tikzpicture}
