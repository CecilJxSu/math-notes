\paragraph{}
本章将应用导数来研究函数以及曲线的某些形态,并利用这些知识解决一些实际问题。为此,先介绍微分学的几个中值定理,它们是导数应用的理论基础。

\subsection{罗尔定理(Rolle's theorem)}
\paragraph{}
\textbf{费马引理\;}设函数$f(x)$在点$x_0$的某邻域$U(x_0)$内有定义,并且在$x_0$处可导,如果对任意的$x\in U(x_0)$,有
\begin{equation}
  f(x) \leq f(x_0) \;(\text{或} f(x) \geq f(x_0)),
\end{equation}
那么$f'(x_0) = 0$。

\begin{figure}[H]
  \centering
    \input{figure/Fermat‘s_theorem}
    \label{Fermat‘s_theorem}
    \caption{费马引理}
\end{figure}

\paragraph{}
通常称导数等于零的点为函数的\uwave{驻点}(或\uwave{稳定点},\uwave{临界点})。

\paragraph{}
\textbf{罗尔定理\;}如果函数$f(x)$满足
\begin{enumerate}
  \item 在闭区间$[a,b]$上连续;
  \item 在开区间$(a,b)$内可导;
  \item 在区间端点处的函数值相等,即$f(a)=f(b)$,
\end{enumerate}
那么在$(a,b)$内至少有一点$\xi(a < \xi < b)$,使得$f'(\xi) = 0$。

\subsection{拉格朗日中值定理(Lagrange's mean value theorem)}
\subsubsection{定理}
\paragraph{}
罗尔定理中$f(a)=f(b)$这个条件是相当特殊的,它使罗尔定理的应用受到限制。如果把该条件取消,但仍保留其余两个条件,并相应的改变结论,那么得到微分学中十分重要的拉格朗日中值定理。

\paragraph{}
\textbf{拉格朗日中值定理\;}如果函数$f(x)$满足
\begin{enumerate}
  \item 在闭区间$[a,b]$上连续;
  \item 在开区间$(a,b)$内可导,
\end{enumerate}
那么在$(a,b)$内至少有一点$\xi(a < \xi < b)$,使等式
\begin{equation}
  f(b) - f(a) = f'(\xi)(b-a)
\end{equation}
成立。

\begin{figure}[H]
  \centering
    \input{figure/Lagrange‘s_mean_value_theorem}
    \caption{拉格朗日中值定理}
    \label{Lagrange‘s mean value theorem}
\end{figure}

\paragraph{}
从图\figureref{Lagrange‘s mean value theorem}中看出,在罗尔定理中,由于$f(a)=f(b)$,弦$AB$是平行于$x$轴的,因此点$C$处的切线实际上也平行于弦$AB$。由此可见,罗尔定理是拉格朗日中值定理的特殊情形。

\subsubsection{证明}
\paragraph{}
\textbf{证明前的准备\;}使用罗尔定理来证明拉格朗日中值定理。

\paragraph{}
在拉格朗日中值定理中,函数$f(x)$不一定具备$f(a)=f(b)$这个条件,为此我们设想构造一个与$f(x)$有密切联系的函数$\varphi(x)$(称为\uwave{辅助函数}),使$\varphi(x)$满足条件$\varphi(a) = \varphi(b)$。然后应用罗尔定理。

\paragraph{}
从图\figureref{Lagrange‘s mean value theorem}中看到,有向线段$NM$的值是与$x$有关联的函数,把它表示为$\varphi(x)$,它与$f(x)$有密切的联系,且当$x=a$及$x=b$时,点$M$与点$N$重合,即有$\varphi(a)=\varphi(b)=0$。为求得函数$\varphi(x)$的表达式,设直线$AB$的方程为$y=L(x)$,则

\begin{equation*}
  L(x) = f(a) + \frac{f(b) - f(a)}{b - a}(x-a)
\end{equation*}

由于点$M$、$N$的纵坐标依次为$f(x)$及$L(x)$,故表示有向线段$NM$的值的函数

\begin{equation*}
  \varphi(x) = f(x) - L(x) = f(x) - f(a) - \frac{f(b) - f(a)}{b - a}(x - a).
\end{equation*}

\paragraph{}
\textbf{定理的证明\;}引进辅助函数

\begin{equation}
  \varphi(x) = f(x) - f(a) - \frac{f(b) - f(a)}{b - a}(x - a).
\end{equation}

容易验证函数$\varphi(x)$适合罗尔定理的条件:$\varphi(a) = \varphi(b) = 0$;$\varphi(x)$在闭区间$[a,b]$上连续,在开区间$(a,b)$内可导,且

\begin{equation}
  \varphi'(x) = f'(x) - \frac{f(b) - f(a)}{b - a}.
\end{equation}

根据罗尔定理,可知在$(a,b)$内至少有一点$\xi$,使$\varphi'(\xi) = 0$,即

\begin{equation}
  f'(\xi) - \frac{f(b) - f(a)}{b - a} = 0.
\end{equation}

由此得
\begin{equation}
  \frac{f(b) - f(a)}{b - a} = f'(\xi).
\end{equation}

即

\begin{equation}
  \label{拉格朗日中值公式}
  f(b) - f(a) = f'(\xi)(b - a).
\end{equation}

定理证毕。

\paragraph{}
公式\eqref{拉格朗日中值公式}对于$b<a$也成立,该式叫做\uwave{拉格朗日中值公式}

\subsubsection{推论}
\paragraph{}
拉格朗日中值定理在微分学中占有重要地位,有时也称该定理为\uwave{微分中值定理}。以及推导出一个重要推论,对后面积分学很有帮助。

\paragraph{}
如果函数$f(x)$在某一区间上是一个常数,那么$f(x)$在该区间上的导数恒为$0$。它的逆命题也成立:

\paragraph{}
\textbf{推论\;}如果函数$f(x)$在区间$I$上的导数恒为$0$,那么$f(x)$在区间$I$上是一个常数。

\subsection{柯西中值定理(Cauchy's mean value theorem)}
\paragraph{}
\begin{figure}[H]
  \centering
    \input{figure/Cauchy‘s_mean_value_theorem}
    \caption{柯西中值定理}
    \label{Cauchy‘s mean value theorem}
\end{figure}

\paragraph{}
设$AB$由参数方程
\begin{equation}
  \left\{
    \begin{array}{l}
      X = F(x), \\
      Y = f(x)
    \end{array}
    \;(a \leq x \leq b)
  \right.
\end{equation}
表示(图\figureref{Cauchy‘s mean value theorem}),其中$x$为参数,那么曲线上点$(X,Y)$处的切线的斜率为
\begin{equation}
  \frac{dY}{dX}=\frac{f'(x)}{F'(x)}
\end{equation}
弦$AB$的斜率为
\begin{equation}
  \frac{f(b)-f(a)}{F(b)-F(a)}
\end{equation}
假定点$C$对应于参数$x=\xi$,那么曲线上点$C$处的切线平行于弦$AB$,可表示为
\begin{equation}
  \frac{f(b)-f(a)}{F(b)-F(a)} = \frac{f'(x)}{F'(x)}
\end{equation}
因此

\paragraph{}
\textbf{柯西中值定理\;}如果函数$f(x)$及$F(x)$满足
\begin{enumerate}
  \item 在闭区间$[a,b]$上连续;
  \item 在开区间$(a,b)$内可导;
  \item 对任一$x \in (a,b), F'(x) \neq 0$,
\end{enumerate}
那么在$(a,b)$内至少有一点$\xi$,使等式
\begin{equation}
  \frac{f(b)-f(a)}{F(b)-F(a)} = \frac{f'(\xi)}{F'(\xi)}
\end{equation}
成立。
