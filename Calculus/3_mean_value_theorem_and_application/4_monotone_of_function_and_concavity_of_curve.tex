\subsection{函数单调性的判定法}
\paragraph{}
如果函数$y=f(x)$在$[a,b]$上单调增加(单调减少),那么它的图形是一条沿$x$轴正向上升(下降)的曲线。

\begin{figure}[H]
\centering
  %------- 第1行 -------
  \begin{subfigure}[t]{0.48\linewidth}
    \centering
      % 单调增加
\begin{tikzpicture}[scale=0.8]
  \begin{axis}[clip=false,xmin=0, xmax=6,ymin=0,ymax=4, grid=none,
    xtick=\empty, ytick=\empty, axis lines=middle,
    smooth, xlabel={$x$}, ylabel={$y$}]

    % 曲线
    \addplot[draw=red,domain=1.5:4.5] {(x - 1)^5 / (4 * x^3) + 1.5};

    % 端点辅助线
    \draw [dashed] (1.5,0) -- (1.5,1.5);
    \draw [dashed] (4.5,0) -- (4.5,2.94);

    % 端点标记
    \node [left] at (1.5,1.5) {$A$};
    \node [below] at (1.5,0) {$a$};
    \node [right] at (4.5,2.94) {$B$};
    \node [below] at (4.5,0) {$b$};

    % y = f(x)
    \node [left] at (4,2.44) {$y=f(x)$};

    % 切线, dx/dy = 0.444~
    \draw [fill] (3,1.79) circle [radius=0.03];
    \draw (1.3, 1.03) -- (3, 1.79) -- (4.7, 2.54);

    % 原点
    \node [below left] at (0,0) {$O$};
  \end{axis}
\end{tikzpicture}

    \subcaption{单调增加}
  \end{subfigure}
  \begin{subfigure}[t]{0.48\linewidth}
    \centering
      % 单调增加
\begin{tikzpicture}[scale=0.8]
  \begin{axis}[clip=false,xmin=0, xmax=6,ymin=0,ymax=4, grid=none,
    xtick=\empty, ytick=\empty, axis lines=middle,
    smooth, xlabel={$x$}, ylabel={$y$}]

    % 曲线
    \addplot[draw=red,domain=1.5:4.5] {(-x+1)^5 / (4*x^3) + 2.5};

    % 端点辅助线
    \draw [dashed] (1.5,0) -- (1.5,2.5);
    \draw [dashed] (4.5,0) -- (4.5,1.06);

    % 端点标记
    \node [left] at (1.5,2.5) {$A$};
    \node [below] at (1.5,0) {$a$};
    \node [right] at (4.5,1.06) {$B$};
    \node [below] at (4.5,0) {$b$};

    % y = f(x)
    \node [left] at (4,1.55) {$y=f(x)$};

    % 切线, dx/dy = -0.444~
    \draw [fill] (3,2.2) circle [radius=0.03];
    \draw (1.3,2.94) -- (3,2.2) -- (4.7,1.45);

    % 原点
    \node [below left] at (0,0) {$O$};
  \end{axis}
\end{tikzpicture}

    \subcaption{单调减少}
  \end{subfigure}
  \caption{函数单调性}
  \label{函数单调性}
\end{figure}

\paragraph{}
由拉格朗日中值定理和定义可证明:

\paragraph{}
\textbf{定理1\;}设函数$y=f(x)$在$[a,b]$上连续,在$(a,b)$内可导:
\begin{enumerate}
  \item 如果在$(a,b)$内$f'(x) > 0$,那么函数$y=f(x)$在$[a,b]$上单调增加;
  \item 如果在$(a,b)$内$f'(x) < 0$,那么函数$y=f(x)$在$[a,b]$上单调减少;
\end{enumerate}

\subsection{曲线的凹凸性与拐点}
\subsubsection{凹凸性}
\paragraph{}
\textbf{定义\;}设$f(x)$在区间$I$上连续,如果对$I$上任意两点$x_1,x_2$恒有
\begin{equation}
  f(\frac{x_1+x_2}{2}) < \frac{f(x_1) + f(x_2)}{2},
\end{equation}
那么称$f(x)$在$I$上的图形是\uwave{(向上)凹的(或凹弧)};如果恒有
\begin{equation}
  f(\frac{x_1+x_2}{2}) > \frac{f(x_1)+f(x_2)}{2},
\end{equation}
那么称$f(x)$在$I$上的图形是\uwave{(向上)凸的(或凸弧)}。

\paragraph{}

\begin{figure}[H]
\centering
  %------- 第1行 -------
  \begin{subfigure}[t]{0.48\linewidth}
    \centering
      % 函数凹弧
\begin{tikzpicture}[scale=0.8]
  \begin{axis}[clip=false,xmin=0, xmax=6,ymin=0,ymax=6, grid=none,
    xtick=\empty, ytick=\empty, axis lines=middle,
    smooth, xlabel={$x$}, ylabel={$y$}]

    % 曲线
    \addplot[draw=red,domain=1.3:4.7] {(x-3)^2/x + 2};
    % f'(x) = (2*(x-3)*x - (x-3)^2)) / x^2

    % 端点辅助线
    \draw [dashed] (1.8,0) -- (1.8,2.8);
    \draw [dashed] (4.2,0) -- (4.2,2.34);

    % 曲线端点的标记
    \node [below] at (1.8,0) {$x_1$};
    \node [below left] at (1.8,2.8) {$f(x_1)$};
    \node [below] at (4.2,0) {$x_2$};
    \node [below right] at (4.2,2.34) {$f(x_2)$};
    \node [below] at (3,0) {$\frac{x_1+x_2}{2}$};

    % 割线:y - 2.8 = -0.19 * (x - 1.8)
    \draw [fill] (3,2.57) circle [radius=0.03];
    \draw (1.8,2.8) -- (4.2,2.34);
    % 割线标记
    \node [above] at (3,2.7) {$\frac{f(x_1)+f(x_2)}{2}$};
    \draw [dashed] (3,2.57) -- (3,0);

    % 曲线上的 x=3,中点
    \node [fill=white,below, inner sep=1.5] at (3,1.8) {$f(\frac{x_1+x_2}{2})$};
    \draw [fill] (3,2) circle [radius=0.03];

    % 原点
    \node [below left] at (0,0) {$O$};
  \end{axis}
\end{tikzpicture}

    \subcaption{凹弧}
  \end{subfigure}
  \begin{subfigure}[t]{0.48\linewidth}
    \centering
      % 函数凹弧一阶、二阶导函数
\begin{tikzpicture}[scale=0.8]
  \begin{axis}[clip=false,xmin=0, xmax=6,ymin=-5,ymax=10, grid=none,
    xtick=\empty, ytick=\empty, axis lines=middle,
    smooth, xlabel={$x$}, ylabel={$y$}]

    % 一阶导函数
    \addplot[draw=red,domain=1.3:4.7] {1 - 9 * x^(-2)};
    % 二阶导函数
    \addplot[draw=blue,domain=1.3:4.7] {18 * x^(-3)};

    \node [left] at (1.5,-3) {$f'(x)$};
    \node [left] at (1.5,5.33) {$f''(x)$};

    % 原点
    \node [below left] at (0,0) {$O$};
  \end{axis}
\end{tikzpicture}

    \subcaption{一阶和二阶导函数}
  \end{subfigure}
  %------- 第2行 -------
  \begin{subfigure}[t]{0.48\linewidth}
    \centering
      % 函数凸弧弧
\begin{tikzpicture}[scale=0.8]
  \begin{axis}[clip=false,xmin=0, xmax=6,ymin=0,ymax=6, grid=none,
    xtick=\empty, ytick=\empty, axis lines=middle,
    smooth, xlabel={$x$}, ylabel={$y$}]

    % 曲线
    \addplot[draw=red,domain=1.3:4.7] {-(x-3)^2/x + 3.5};
    % f'(x) = -(2*(x-3)*x - (x-3)^2)) / x^2

    % 端点辅助线
    \draw [dashed] (1.8,0) -- (1.8,2.7);
    \draw [dashed] (4.2,0) -- (4.2,3.16);

    % 曲线端点的标记
    \node [below] at (1.8,0) {$x_1$};
    \node [above left] at (1.8,2.7) {$f(x_1)$};
    \node [below] at (4.2,0) {$x_2$};
    \node [above right] at (4.2,3.16) {$f(x_2)$};
    \node [below] at (3,0) {$\frac{x_1+x_2}{2}$};

    % 割线:y - 2.7 = 0.19 * (x - 1.8)
    \draw [fill] (3,2.93) circle [radius=0.03];
    \draw (1.8,2.7) -- (4.2,3.16);

    % 曲线上的 x=3,中点
    \node [above] at (3,3.6) {$f(\frac{x_1+x_2}{2})$};
    \draw [fill] (3,3.5) circle [radius=0.03];

    % 割线标记
    \node [fill=white,below,inner sep=1.5] at (3,2.7) {$\frac{f(x_1)+f(x_2)}{2}$};
    \draw [dashed] (3,3.5) -- (3,0);

    % 原点
    \node [below left] at (0,0) {$O$};
  \end{axis}
\end{tikzpicture}

    \subcaption{凸弧}
  \end{subfigure}
  \begin{subfigure}[t]{0.48\linewidth}
    \centering
      % 函数凸弧一阶、二阶导函数
\begin{tikzpicture}[scale=0.8]
  \begin{axis}[clip=false,xmin=0, xmax=6,ymin=-9,ymax=6, grid=none,
    xtick=\empty, ytick=\empty, axis lines=middle,
    smooth, xlabel={$x$}, ylabel={$y$}]

    % 一阶导函数
    \addplot[draw=red,domain=1.3:4.7] {-1 + 9 * x^(-2)};
    % 二阶导函数
    \addplot[draw=blue,domain=1.3:4.7] {-18 * x^(-3)};

    \node [left] at (1.5,3) {$f'(x)$};
    \node [left] at (1.5,-5.33) {$f''(x)$};

    % 原点
    \node [below left] at (0,0) {$O$};
  \end{axis}
\end{tikzpicture}

    \subcaption{一阶和二阶导函数}
  \end{subfigure}
  \caption{函数凹凸性}
  \label{函数凹凸性}
\end{figure}

\paragraph{}
由拉格朗日中值定理和定义可证明:

\paragraph{}
\textbf{定理2\;}设$f(x)$在$[a,b]$上连续,在$(a,b)$内具有一阶和二阶导数,那么
\begin{enumerate}
  \item 若在$(a,b)$内$f''(x) > 0$,则$f(x)$在$[a,b]$上的图形是凹的;
  \item 若在$(a,b)$内$f''(x) < 0$,则$f(x)$在$[a,b]$上的图形是凸的。
\end{enumerate}

\subsubsection{拐点}
\paragraph{}
设$y=f(x)$在区间$I$上连续,$x_0$是$I$的内点。如果曲线$y=f(x)$在经过点$(x_0,f(x_0))$时,曲线的凹凸性改变了,那么就称点$(x_0,f(x_0))$为这曲线的\uwave{拐点}。

\paragraph{}
由$f''(x)$的符号可以判定曲线的凹凸性,因此,如果$f''(x)$在$x_0$的左右两侧邻近异号,那么点$(x_0,f(x_0))$就是曲线的一个拐点。

\paragraph{}
除此之外,$f(x)$的二阶导数不存在的点,也有可能是$f''(x)$的符号发生变化的分界点。因此,找拐点的步骤:
\begin{enumerate}
  \item 求$f''(x)$;
  \item 令$f''(x)=0$,解出这方程在区间$I$内的实根,并求出在区间$I$内$f''(x)$不存在的点;
  \item 对于$2$中求出的每一个实根或二阶导数不存在的点$x_0$,检查$f''(x)$在$x_0$左、右两侧邻近的符号,当两侧的符号相反时,点$(x_0,f(x_0))$是拐点,当两侧的符号相同时,点$(x_0,f(x_0))$不是拐点。
\end{enumerate}
