\subsection{方程的近似解}
\paragraph{}
求解高次代数方程或其它类型的方程的问题。求得这类方程的实根的精确值,比较困难,因此,可以寻求方程的近似解。

\begin{enumerate}
  \item 确定根的大致范围,确定一个区间$[a,b]$,使所求的根位于这区间内。这一步工作称为\uwave{根的隔离},区间$[a,b]$称为所求实根的\uwave{隔离区间}。
  \item 以根的隔离区间的端点作为根的初始近似值,逐步逼近精确值,或在精确度范围内的值。
\end{enumerate}

\subsubsection{二分法}
\paragraph{}
设$f(x)$在区间$[a,b]$上连续,$f(a)\bigcdot f(b) < 0$,且方程$f(x)=0$在$(a,b)$内仅有一个实根$\xi$,于是$[a,b]$即是这个根的一个隔离区间。

\paragraph{}
取$[a,b]$的中点$\displaystyle \xi_1 = \frac{a+b}{2}$,计算$f(\xi_1)$,然后不断缩小隔离区间的范围,不断计算中点$\xi_1$直到$f(\xi_1)=0$或在精确度之内$f(\xi_1) - 0 < \varepsilon$

\subsubsection{切线法}
\paragraph{}
用曲线弧一端的切线来代替曲线弧,从而求出方程实根的近似值。这种方法叫做\uwave{切线法}。端点的选取根据下面$4$种情况。
\begin{figure}[H]
\centering
  %------- 第1行 -------
  \begin{subfigure}[t]{0.45\linewidth}
    \centering
      % 切线法的 4 种不同情形
\begin{tikzpicture}[scale=0.8]
  \begin{axis}[clip=false,xmin=0, xmax=8,ymin=-4,ymax=4, grid=none,
    xtick=\empty, ytick=\empty, axis lines=middle,
    smooth, xlabel={$x$}, ylabel={$y$}]

    % y' = 0.15*(x-1)^0.5*x + 0.1 * (x-1)^1.5
    \addplot[draw=red,domain=1:5.5] {0.1 * (x-1)^1.5 * x - 2};

    % 实根
    \node [above] at (3.95,0) {$\xi$};

    % A
    \draw [dashed] (1,-2) -- (1,0);
    \draw [fill] (1,-2) circle [radius=0.05];
    \node [left] at (1,-2) {$A$};
    \node [above] at (1,0) {$a$};

    % B
    \draw [dashed] (5.5,3.25) -- (5.5,0);
    \draw [fill] (5.5,3.25) circle [radius=0.05];
    \node [right] at (5.5,3.25) {$B$};
    \node [below] at (5.5,0) {$b$};

    \node [below left] at (5,3.25) {$y=f(x)$};

    % 切线
    \addplot[domain=4.3:5.5] {2.7*x - 11.6};
    \node [below] at (4.3,0) {$x_1$};

    % 原点
    \node [left] at (0,0) {$O$};
  \end{axis}
\end{tikzpicture}

      \subcaption{$f(a)<0,f(b)>0$\newline$f'(x)>0,f''(x)>0$}
  \end{subfigure}
  \begin{subfigure}[t]{0.45\linewidth}
    \centering
      % 切线法的 4 种不同情形
\begin{tikzpicture}[scale=0.8]
  \begin{axis}[clip=false,xmin=0, xmax=8,ymin=-4,ymax=4, grid=none,
    xtick=\empty, ytick=\empty, axis lines=middle,
    smooth, xlabel={$x$}, ylabel={$y$}]

    % y' = -20*(x+1)^-2
    \addplot[draw=red,domain=1.5:6] {20/(x+1) - 5};

    % 实根
    \node [above] at (3,0) {$\xi$};

    % A
    \draw [dashed] (1.5,3) -- (1.5,0);
    \draw [fill] (1.5,3) circle [radius=0.05];
    \node [left] at (1.5,3) {$A$};
    \node [below] at (1.5,0) {$a$};

    \node [below right] at (1.8,3) {$y=f(x)$};

    % 切线
    \addplot[domain=1.5:2.44] {-3.2*x + 7.8};
    \node [below] at (2.44,0) {$x_1$};

    % B
    \draw [dashed] (6,-2.14) -- (6,0);
    \draw [fill] (6,-2.14) circle [radius=0.05];
    \node [right] at (6,-2.14) {$B$};
    \node [above] at (6,0) {$b$};

    % 原点
    \node [left] at (0,0) {$O$};
  \end{axis}
\end{tikzpicture}

      \subcaption{$f(a)>0,f(b)<0$\newline$f'(x)<0,f''(x)>0$}
  \end{subfigure}
  %------- 第2行 -------
  \begin{subfigure}[t]{0.45\linewidth}
    \centering
      % 切线法的 4 种不同情形
\begin{tikzpicture}[scale=0.8]
  \begin{axis}[clip=false,xmin=0, xmax=8,ymin=-4,ymax=4, grid=none,
    xtick=\empty, ytick=\empty, axis lines=middle,
    smooth, xlabel={$x$}, ylabel={$y$}]

    % y' = 3/x
    \addplot[draw=red,domain=1.5:6] {3*ln(x) - 3.5};

    % 实根
    \node [below] at (3.23,0) {$\xi$};

    % A
    \draw [dashed] (1.5,-2.28) -- (1.5,0);
    \draw [fill] (1.5,-2.28) circle [radius=0.05];
    \node [left] at (1.5,-2.28) {$A$};
    \node [above] at (1.5,0) {$a$};

    % 切线
    \addplot[domain=1.5:2.64] {2*x - 5.28};
    \node [above] at (2.64,0) {$x_1$};

    % B
    \draw [dashed] (6,1.88) -- (6,0);
    \draw [fill] (6,1.88) circle [radius=0.05];
    \node [right] at (6,1.88) {$B$};
    \node [below] at (6,0) {$b$};

    \node [below left] at (5,1.88) {$y=f(x)$};

    % 原点
    \node [left] at (0,0) {$O$};
  \end{axis}
\end{tikzpicture}

      \subcaption{$f(a)<0,f(b)>0$\newline$f'(x)>0,f''(x)<0$}
  \end{subfigure}
  \begin{subfigure}[t]{0.45\linewidth}
    \centering
      % 切线法的 4 种不同情形
\begin{tikzpicture}[scale=0.8]
  \begin{axis}[clip=false,xmin=0, xmax=8,ymin=-4,ymax=4, grid=none,
    xtick=\empty, ytick=\empty, axis lines=middle,
    smooth, xlabel={$x$}, ylabel={$y$}]

    % -0.6*(x-1)^0.8
    \addplot[draw=red,domain=1:6] {-0.33*(x-1)^1.8+2.5};

    % 实根
    \node [below left] at (4.248,0) {$\xi$};

    % A
    \draw [dashed] (1,2.5) -- (1,0);
    \draw [fill] (1,2.5) circle [radius=0.05];
    \node [left] at (1,2.5) {$A$};
    \node [below] at (1,0) {$a$};

    \node [below right] at (2.5,2.5) {$y=f(x)$};

    % B
    \draw [dashed] (6,-3.48) -- (6,0);
    \draw [fill] (6,-3.48) circle [radius=0.05];
    \node [right] at (6,-3.48) {$B$};
    \node [above] at (6,0) {$b$};

    % 切线
    \addplot[domain=4.41:6] {-2.17*x + 9.57};
    \node [above] at (4.41,0) {$x_1$};

    % 原点
    \node [left] at (0,0) {$O$};
  \end{axis}
\end{tikzpicture}

      \subcaption{$f(a)>0,f(b)<0$\newline$f'(x)<0,f''(x)<0$}
  \end{subfigure}

  \caption{切线的$4$种不同情形}
  \label{切线的4种不同情形}
\end{figure}

\paragraph{}
假设选取端点为$a$的情形,令$x_0 = a$,在端点$(x_0,f(x_0))$作切线,这切线的方程为
\begin{equation*}
  y - f(x_0) = f'(x_0)(x-x_0)
\end{equation*}
令$y=0$,可以解出$x_1$为
\begin{equation*}
  x_1 = x_0 - \frac{f(x_0)}{f'(x_0)}
\end{equation*}

\paragraph{}
然后继续在点$(x_1,f(x_1))$作切线,直到逼近$\xi$,一般的,在点$(x_{n-1},f(x_{n-1}))$作切线,得根的近似值
\begin{equation}
  x_n = x_{n-1} - \frac{f(x_{n-1})}{f'(x_{n-1})}.
\end{equation}
