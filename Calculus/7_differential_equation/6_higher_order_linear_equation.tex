% \subsection{二阶线性微分方程举例}
\subsection{线性微分方程的解的结构}
\paragraph{}
二阶齐次线性方程
\begin{equation}
  \label{二阶齐次线性方程}
  y'' + P(x)y' + Q(x)y = 0.
\end{equation}

\paragraph{}
\textbf{定理1~~} 如果函数$y_1(x)$与$y_2(x)$是方程\eqref{二阶齐次线性方程}的两个解,那么
\begin{equation}
  y = C_1y_1(x) + C_2y_2(x)
\end{equation}
也是\eqref{二阶齐次线性方程}的解,其中$C_1, C_2$是任意常数。

\paragraph{}
\textbf{定理2~~}如果$y_1(x)$与$y_2(x)$是方程\eqref{二阶齐次线性方程}的两个线性无关的特解,那么
\begin{equation*}
  y = C_1y_1(x) + C_2y_2(x) ~~ (C_1, C_2\text{是任意常数})
\end{equation*}
就是方程\eqref{二阶齐次线性方程}的通解。

\paragraph{}
\textbf{推论~~}如果$y_1(x), y_2(x), \cdots, y_n(x)$是$n$阶齐次线性方程
\begin{equation*}
  y^{(n)}+a_1(x)y^{(n-1)}+\cdots+a_{n-1}(x)y' + a_n(x)y = 0
\end{equation*}
的$n$个线性无关的解,那么,此方程的通解为
\begin{equation*}
  y=C_1y_1(x) + C_2y_2(x) + \cdots + C_ny_n(x),
\end{equation*}
其中$C_1,C_2,\cdots,C_n$为任意常数。

\paragraph{}
\textbf{定理3~~}设$y^*(x)$是二阶非齐次线性方程
\begin{equation}
  \label{定理3,二阶非齐次线性方程}
  y''+P(x)y'+Q(x)y = f(x)
\end{equation}
的一个特解。$Y(x)$是与\eqref{定理3,二阶非齐次线性方程}对应的齐次方程\eqref{二阶齐次线性方程}的通解,那么
\begin{equation}
  y = Y(x) + y^*(x)
\end{equation}
是二阶非齐次线性微分方程\eqref{定理3,二阶非齐次线性方程}的通解。

\paragraph{}
\textbf{定理4~~}设非齐次线性方程\eqref{定理3,二阶非齐次线性方程}的右端$f(x)$是两个函数之和,即
\begin{equation}
  y'' + P(x)y' + Q(x)y = f_1(x) + f_2(x),
\end{equation}
而$y_1^*(x)$与$y_2^*(x)$分别是方程
\begin{equation*}
  y'' + P(x)y' + Q(x)y = f_1(x)
\end{equation*}
与
\begin{equation*}
  y'' + P(x)y' + Q(x)y = f_2(x)
\end{equation*}
的特解,那么$y_1^*(x)+y_2^*(x)$就是原方程的特解。
\paragraph{}
这一定理通常称为线性微分方程的解的\uwave{叠加原理}。

\subsection{常数变易法}
\paragraph{}
