\documentclass{article}

\usepackage{geometry}
\usepackage{xeCJK}
\usepackage{amsmath}
\usepackage{tikz}
\usepackage{pgfplots}
\usepackage{pst-func}
% figure[H] float
\usepackage{float}
% subfigure
\usepackage{subcaption}
\usepackage{amssymb}
\usepackage{hyperref}
\usepackage{setspace}
% 字体底部样式:下划线、波浪线等
\usepackage{ulem}
% 修改公式编号
\usepackage{chngcntr}

% make cdot thicker,比 cdot 更粗的圆点
\makeatletter
\newcommand*\bigcdot{\mathpalette\bigcdot@{.5}}
\newcommand*\bigcdot@[2]{\mathbin{\vcenter{\hbox{\scalebox{#2}{$\m@th#1\bullet$}}}}}
\makeatother

% 设置行间距 1.5 倍
\renewcommand{\baselinestretch}{1.5}
% 自定义图片的标题:Figure -> 图
\renewcommand{\figurename}{图}
% 自定义表格的标题:Table -> 表
\renewcommand{\tablename}{表}

% 设置页大小和页边距,或者scale=0.8
\geometry{a4paper,left=3.18cm,right=3.18cm,top=2.54cm,bottom=2.54cm}
% 兼容
\pgfplotsset{compat=1.16}
% 中文默认没有斜体和粗体格式,开启伪斜体和指定黑体;
\setCJKmainfont[AutoFakeSlant, BoldFont=SimHei]{SimSun}
% area of hatch,面积阴影部分, 箭头类型
\usetikzlibrary{patterns, arrows.meta, positioning}

\hypersetup{
    colorlinks,
    citecolor=black,
    filecolor=black,
    linkcolor=black,
    urlcolor=black
}
% 定义amsmath arccot, ch, sh
\DeclareMathOperator{\arccot}{arccot}
\DeclareMathOperator{\ch}{ch}
\DeclareMathOperator{\sh}{sh}
% 每个章节后,重置公式编号
\counterwithin*{equation}{section}

% 修改公式标签引用颜色
\def\eqref#1{{\color{blue}\hypersetup{linkcolor=blue} (\ref{#1}) }}
% 修改图片标签引用的颜色
\def\figureref#1{{\color{blue}\hypersetup{linkcolor=blue} (\ref{#1}) }}
% 修改跳转标签引用的颜色
\def\linkref[#1]#2{\hyperref[#1]{\color{blue}\ #2\ }}
% 引用 section 的章节号
\def\secref#1{\hyperref[#1]{\color{blue}[\ref{#1}节]}}

% 缩进
\usepackage{indentfirst}
\makeatletter
\def\paragraph{
    \parindent2em
}
\makeatother

\begin{document}
  \tableofcontents
  \newpage

  \part{微分方程}
  \section{微分方程的基本概念}
    \paragraph{}
函数是客观事物的内部联系在数量方面的反映,利用函数关系又可以对客观事物的规律性进行研究。因此如何寻求函数关系,在实践中具有重要意义。

\paragraph{}
在许多问题中,往往不能直接找出所需要的函数关系,但是根据问题所提供的情况,有时可以列出含有要找的函数及其导数的关系式。这样的关系式就是所谓\uwave{微分方程}。

\paragraph{}
微分方程建立后,对它进行研究,找出未知函数来,这就是\uwave{解微分方程}。

\subsection{例子}
\subsubsection{例1}
\paragraph{}
\textbf{例1\;}一曲线通过点$(1,2)$,且在该曲线上任一点$M(x,y)$处的切线的斜率为$2x$,求这曲线的方程。

\paragraph{}
\textbf{解\;}设所求曲线的方程为$y=\varphi(x)$。根据导数的几何意义,可知未知函数$y=\varphi(x)$应满足关系式

\begin{equation}
  \label{微分方程概念例1公式1}
  \frac{dy}{dx} = 2x.
\end{equation}

此外,未知函数$y=\varphi(x)$还应满足下列条件:

\begin{equation}
  \label{微分方程概念例1公式2}
  x = 1 \text{时}, y = 2.
\end{equation}

\paragraph{}
把\eqref{微分方程概念例1公式1}式两端积分,得

\begin{equation}
  \label{微分方程概念例1公式3}
  y = \int 2xdx \; \text{即} \; y=x^2 + C.
\end{equation}

其中$C$是任意常数。

\paragraph{}
把条件“$x=1$时$, y=2$”代入\eqref{微分方程概念例1公式3}式,得

\begin{equation}
  \label{微分方程概念例1公式4}
  2 = 1^2 + C,
\end{equation}

由此得出$C=1$。把$C=1$代入\eqref{微分方程概念例1公式3}式,即得所求曲线方程

\begin{equation}
  \label{微分方程概念例1公式5}
  y = x^2 + 1.
\end{equation}

\subsubsection{例2}
\paragraph{}
\textbf{例2\;}列车在平直线路上以$20m/s$(相当于$72 km/h$)的速度行驶;当制动时列车获得加速度$-0.4m/s^2$。问开始制动后多少时间列车才能停住以及列车在这段时间里行驶了多少路程?

\paragraph{}
\textbf{解\;}设列车在开始制动后$t s$时行驶了$s m$。根据题意,反映制动阶段列车运动规律的函数$s=s(t)$应满足关系式
\begin{equation}
  \label{微分方程概念例2公式1}
  \frac{d^2s}{dt^2} = -0.4.
\end{equation}
此外,未知函数$s=s(t)$还应满足下列条件:
\begin{equation}
  \label{微分方程概念例2公式2}
  t = 0 \text{时}, s = 0, v = \frac{ds}{dt} = 20.
\end{equation}

\paragraph{}
把\eqref{微分方程概念例2公式1}式两端积分一次,得
\begin{equation}
  \label{微分方程概念例2公式3}
  v = \frac{ds}{dt} = -0.4t + C_1;
\end{equation}
再积分一次,得
\begin{equation}
  \label{微分方程概念例2公式4}
  s = -0.2t^2 + C_1t + C_2,
\end{equation}
这里$C_1, C_2$都是任意常数。

\paragraph{}
把条件“$t = 0$时$,v=20$”代入\eqref{微分方程概念例2公式3}式,得
\begin{equation}
  \label{微分方程概念例2公式5}
  20 = C_1;
\end{equation}

\paragraph{}
把条件“$t = 0$时$,s=0$”代入\eqref{微分方程概念例2公式4}式,得
\begin{equation}
  \label{微分方程概念例2公式6}
  0 = C_2;
\end{equation}
把$C_1,C_2$的值代入\eqref{微分方程概念例2公式3}及\eqref{微分方程概念例2公式4},得
\begin{align}
    \label{微分方程概念例2公式7}
    v \;=&\; -0.4t + 20, \\
    \label{微分方程概念例2公式8}
    s \;=&\; -0.2t^2 + 20t.
\end{align}

\paragraph{}
在\eqref{微分方程概念例2公式7}式中令$v=0$,得到列车从开始制动到完全停住所需的时间
\begin{equation}
  \label{微分方程概念例2公式9}
  t = \frac{20}{0.4} = 50(s).
\end{equation}
再把$t=50$代入\eqref{微分方程概念例2公式8}式,得到列车在制动阶段行驶的路程
\begin{equation}
\label{微分方程概念例2公式10}
s = -0.2\times50^2+20\times50=500(m).
\end{equation}

\subsection{基本概念}
\subsubsection{微分方程和它的阶}
\paragraph{}
上述两个例子中的关系式\eqref{微分方程概念例1公式1}和\eqref{微分方程概念例2公式1}都含有未知函数的导数,它们都是微分方程。

\paragraph{}
一般的,凡表示未知函数、未知函数的导数与自变量之间的关系的方程,叫做\uwave{微分方程},有时也简称\uwave{方程}。

\paragraph{}
微分方程中所出现的未知函数的最高阶导数的阶数,叫做\uwave{微分方程的阶}。例如,方程\eqref{微分方程概念例1公式1}是一阶微分方程;\eqref{微分方程概念例2公式1}是二阶微分方程。又如,方程
\begin{equation}
x^3y''' + x^2y'' - 4xy' = 3x^2,
\end{equation}
是三阶微分方程;方程
\begin{equation}
y^{(4)} - 4y''' + 10y'' - 12y' + 5y = \sin{2x},
\end{equation}
是四阶微分方程。

\subsubsection{$n$阶微分方程}
\paragraph{}
一般的,$n$阶微分方程的形式是

\begin{equation}
\label{n阶微分方程的形式}
F(x,y,y',\cdots,y^{(n)})=0.
\end{equation}

这里必须指出,在方程\eqref{n阶微分方程的形式}中,$y^{(n)}$是必须出现的,而$x,y,y',\cdots,y^{(n-1)}$等变量则可以不出现。例如$n$阶微分方程

\begin{equation}
y^{(n)} + 1 = 0,
\end{equation}

中,除$y^{(n)}$外,其它变量都是没有出现。

\paragraph{}
如果能从方程\eqref{n阶微分方程的形式}中解出最高阶导数,则可得微分方程
\begin{equation}
  y^{(n)} = f(x,y,y',\cdots,y^{(n-1)}).
\end{equation}

\subsubsection{微分方程的解}
\paragraph{}
在研究某些实际问题时,首先要建立微分方程,然后找出满足微分方程的函数(解微分方程),就是说,找出这样的函数,把这函数代入微分方程能使该方程成为恒等式。这个函数就叫做该\uwave{微分方程的解}。

\paragraph{}
确切地说,设函数$y=\varphi(x)$在区间$I$上有$n$阶连续导数,如果在区间$I$上,
\begin{equation}
  F[x,\varphi(x),\varphi'(x),\cdots,\varphi^{(n)}(x) \equiv 0],
\end{equation}
那么函数$y=\varphi(x)$就叫做微分方程\eqref{n阶微分方程的形式}在区间$I$上的解;

\paragraph{}
如果微分方程的解中含有任意常数,且任意常数的个数与微分方程的阶数相同,这样的解叫做\uwave{微分方程的通解}。

\paragraph{}
例如,函数\eqref{微分方程概念例1公式3}是方程\eqref{微分方程概念例1公式1}的解,它含有一个任意常数,而方程\eqref{微分方程概念例1公式1}是一阶的,所以函数\eqref{微分方程概念例1公式3}是方程\eqref{微分方程概念例1公式1}的通解。

\paragraph{}
又如,函数\eqref{微分方程概念例2公式4}是方程\eqref{微分方程概念例2公式1}的解,它含有两个任意常数,而方程\eqref{微分方程概念例2公式1}是二阶的,所以函数\eqref{微分方程概念例2公式4}是方程\eqref{微分方程概念例2公式1}的通解。

\subsubsection{初始条件和特解}
\paragraph{}
由于通解中含有任意常数,所以它还不能完全确定地反映某一客观事物的规律性。要完全确定地反映客观事物的规律性,必须确定这些常数的值。

\paragraph{}
设微分方程中的未知函数为$y=\varphi(x)$,如果微分方程是一阶的,通常用来确定任意常数的条件是

\begin{equation}
  x = x_0 \text{时}, y = y_0,
\end{equation}
或写成
\begin{equation}
  y|_{x=x_0} = y_0,
\end{equation}

其中$x_0, y_0$都是给定的值;如果微分方程是二阶的,通常用来确定任意常数的条件是
\begin{equation}
  x=x_0 \text{时}, y=y_0, y'=y'_0,
\end{equation}
或写成
\begin{equation}
  y|_{x=x_0} = y_0, y'|_{x=x_0} = y'_0,
\end{equation}
其中$x_0, y_0$和$y'_0$都是给定的值。上述这种条件叫做\uwave{初始条件}。

\paragraph{}
确定了通解中的任意常数以后,就得到\uwave{微分方程的特解}。例如\eqref{微分方程概念例1公式5}式是方程\eqref{微分方程概念例1公式1}满足条件\eqref{微分方程概念例1公式2}的特解;\eqref{微分方程概念例2公式8}式是方程\eqref{微分方程概念例2公式1}满足条件\eqref{微分方程概念例2公式2}的特解

\subsubsection{初值问题和积分曲线}
\paragraph{}
求微分方程$y'=f(x,y)$满足初始条件$y|_{x=x_0}=y_0$的特解这样一个问题,叫做一阶微分方程的\uwave{初值问题},记作

\begin{align}
  \label{一阶初值问题}
  \left\{\begin{array}{l}
    y'=f(x,y), \\
    y|_{x=x_0} = y_0.
  \end{array} \right.
\end{align}

\paragraph{}
微分方程的解的图形是一条曲线,叫做\uwave{微分方程的积分曲线}。初值问题\eqref{一阶初值问题}的几何意义,就是求微分方程通过点$(x_0,y_0)$的那条积分曲线。二阶微分方程的初值问题

\begin{align}
  \label{二阶初值问题}
  \left\{\begin{array}{l}
    y''=f(x,y,y'), \\
    y|_{x=x_0} = y_0, y'|_{x=x_0} = y'_0
  \end{array} \right.
\end{align}
的几何意义,是求微分方程通过点$(x_0,y_0)$且在该点处的切线斜率为$y'_0$的那条积分曲线。

  \section{可分离变量的微分方程}
    \paragraph{}
一阶微分方程$y'=f(x,y)$有时也写成如下的对称形式:
\begin{equation}
  \label{微分方程的对称形式}
  P(x,y)dx + Q(x,y)dy = 0.
\end{equation}

在方程\eqref{微分方程的对称形式}中,变量$x$与$y$对称,它既可看作是以$x$为自变量、$y$为因变量的方程
\begin{equation}
  \frac{dy}{dx} = -\frac{P(x,y)}{Q(x,y)}, \; (Q(x,y) \neq 0)
\end{equation}
也可看作是以$y$为自变量、$x$为因变量的方程
\begin{equation}
  \frac{dx}{dy} = -\frac{Q(x,y)}{P(x,y)}, \; (P(x,y) \neq 0)
\end{equation}

\paragraph{}
在第一节的例1中,一阶微分方程$\displaystyle \frac{dy}{dx}=2x$,或$dy=2xdx$,可以把两端都积分就得到方程的通解$y=x^2+C$。但是并不是所有的一阶微分方程都能这样求解。例如,对于一阶微分方程

\begin{equation}
  \label{二阶微分方程例子1}
  \frac{dy}{dx}=2xy^2
\end{equation}
就不能像上面那样用直接对两端积分的方法求出它的通解。原因是方程\eqref{二阶微分方程例子1}的右端含有与$x$存在函数关系的变量$y$,积分
\begin{equation}
  \int{2xy^2dx}
\end{equation}
求不出来。

\paragraph{}
为了解决这个困难,在方程\eqref{二阶微分方程例子1}的两端同时乘以$\displaystyle \frac{dx}{y^2}$,使方程\eqref{二阶微分方程例子1}变为
\begin{equation}
  \frac{dy}{y^2}=2xdx,
\end{equation}
这样,变量$x$与$y$已分离在等式的两端,然后两端积分得
\begin{align}
  -\frac{1}{y} \;=&\; x^2 + C, \\
  y \;=&\; -\frac{1}{x^2+C}
\end{align}
其中$C$是任意常数。

\paragraph{}
一般的,如果一个一阶微分方程能写成
\begin{equation}
  \label{可分离变量的微分方程}
  g(y)dy = f(x)dx
\end{equation}
的形式,就是说,能把微分方程写成一端只含$y$的函数和$dy$,另一端只含$x$的函数和$dx$,那么原方程就称为\uwave{可分离变量的微分方程}。

\paragraph{}
\textbf{证~~}假定方程\eqref{可分离变量的微分方程}中的函数$g(y)$和$f(x)$是连续的。设$y=\varphi(x)$是方程\eqref{可分离变量的微分方程}的解,将它代入\eqref{可分离变量的微分方程}中得到恒等式
\begin{equation}
  g[\varphi(x)]\varphi'(x)dx = f(x)dx.
\end{equation}
将上式两端积分,并由$y=\varphi(x)$引进变量$y$,得
\begin{equation}
  \int{g(y)dy} = \int{f(x)dx}.
\end{equation}

设$G(y)$及$F(x)$依次为$g(y)$及$f(x)$的原函数,于是有
\begin{equation}
  \label{可分离变量的微分方程的通解}
  G(y) = F(x) + C.
\end{equation}
因此,方程\eqref{可分离变量的微分方程}的解满足关系式\eqref{可分离变量的微分方程的通解}。

\paragraph{}
反之,如果$y=\Phi(x)$是由关系式\eqref{可分离变量的微分方程的通解}所确定的隐函数,那么在$g(y) \neq 0$的条件下,$y=\Phi(x)$也是方程\eqref{可分离变量的微分方程}的解,事实上,由隐函数的求导法可知,当$g(y) \neq 0$时,

\begin{equation}
  \phi'(x) = \frac{F'(x)}{G'(y)} = \frac{f(x)}{g(y)},
\end{equation}

这就表示函数$y=\Phi(x)$满足方程\eqref{可分离变量的微分方程}。所以,如果已分离变量的方程\eqref{可分离变量的微分方程}中,$g(y)$和$f(x)$是连续的,且$g(y) \neq 0$,那么\eqref{可分离变量的微分方程}式两端积分后得到的关系式\eqref{可分离变量的微分方程的通解},就用隐式给出了方程\eqref{可分离变量的微分方程}的解,\eqref{可分离变量的微分方程的通解}式就叫做微分方程\eqref{可分离变量的微分方程}的\uwave{隐式解}。

\paragraph{}
又由于关系式\eqref{可分离变量的微分方程的通解}中含有任意常数,因此\eqref{可分离变量的微分方程的通解}式所确定的隐函数是方程\eqref{可分离变量的微分方程}的通解,所以\eqref{可分离变量的微分方程的通解}式叫做微分方程\eqref{可分离变量的微分方程}的\uwave{隐式通解}(当$f(x)\neq0$时,\eqref{可分离变量的微分方程的通解}式所确定的隐函数$x=\Psi(y)$也可认为是方程\eqref{可分离变量的微分方程}的解)。

  \section{齐次方程}
    \subsection{齐次方程}
\paragraph{}
如果一阶微分方程可化成
\begin{equation}
  \label{齐次方程形式}
  \frac{dy}{dx} = \varphi(\frac{y}{x})
\end{equation}
的形式,那么就称这方程为\uwave{齐次方程},例如
\begin{equation*}
  (xy-y^2)dx - (x^2-2xy)dy = 0
\end{equation*}
是齐次方程,因为它可化成
\begin{align*}
  \frac{dy}{dx} \;=&\; \frac{xy-y^2}{x^2-2xy}, \\
  \;=&\; \frac{\frac{y}{x} - (\frac{y}{x})^2}{1 - 2(\frac{y}{x})}.
\end{align*}

\paragraph{}
在齐次方程
\begin{equation*}
  \frac{dy}{dx} = \varphi(\frac{y}{x})
\end{equation*}
中,引进新的未知函数
\begin{equation}
  \label{齐次方程换元函数}
  u = \frac{y}{x},
\end{equation}
就可把它化为可分离变量的方程。因为由\eqref{齐次方程换元函数}有

\begin{equation*}
  y = ux, \; \text{两边对$x$求导得} \; \frac{dy}{dx} = u + x\frac{du}{dx},
\end{equation*}
代入方程\eqref{齐次方程形式},便得方程
\begin{equation*}
  u + x\frac{du}{dx} = \varphi(u),
\end{equation*}
即
\begin{equation*}
    x\frac{du}{dx} = \varphi(u) - u.
\end{equation*}
分离变量,得
\begin{equation*}
  \frac{du}{\varphi(u) - u} = \frac{dx}{x}.
\end{equation*}
两端积分,得
\begin{equation*}
  \int\frac{du}{\varphi(u) - u} = \int\frac{dx}{x}.
\end{equation*}
求出积分后,再以$\displaystyle \frac{y}{x}$代替$u$,便得所给齐次方程的通解。

\subsubsection{可化为齐次的方程}
\paragraph{}
方程
\begin{equation}
  \label{可化为齐次的方程通式}
  \frac{dy}{dx} = \frac{ax+by+c}{a_1x+b_1y+c_1}
\end{equation}

当$c=c_1=0$时是齐次的,否则不是齐次的。在非齐次的情形,可用下列变换把它化为齐次方程:令
\begin{equation*}
  x = X + h, y = Y + k,
\end{equation*}
其中$h$及$k$是待定的常数,于是
\begin{equation*}
  dx = dX, dy = dY,
\end{equation*}
从而方程\eqref{可化为齐次的方程通式}成为
\begin{equation*}
  \frac{dY}{dX} = \frac{aX+bY+ah+bk+c}{a_1X+b_1Y+a_1h+b_1k+c_1}.
\end{equation*}

\paragraph{}
如果方程组
\begin{align*}
  \left\{\begin{array}{l}
    ah + bk + c = 0, \\
    a_1h+b_1k+c_1 = 0
  \end{array} \right.
\end{align*}
的系数行列式\vspace{0.5\baselineskip}$\displaystyle \left|\begin{array}{ll}
  a & b \\ a_1 & b_1
\end{array} \right| \neq 0$,即$\displaystyle \frac{a_1}{a} \neq \frac{b_1}{b}$,那么可以定出$h$及$k$使它们满足上述方程组。这样,方程\eqref{可化为齐次的方程通式}便化为齐次方程
\begin{equation*}
  \frac{dY}{dX} = \frac{aX+bY}{a_1X+b_1Y}.
\end{equation*}
求出这齐次方程的通解后,在通解中以$x-h$代$X$,$y-k$代$Y$,便得方程\eqref{可化为齐次的方程通式}的通解。

\paragraph{}
当$\displaystyle \frac{a_1}{a}=\frac{b_1}{b}$时,$h$及$k$无法求得,因此上述方法不能应用。但这时令$\displaystyle \frac{a_1}{a}=\frac{b_1}{b}=\lambda$,从而方程\eqref{可化为齐次的方程通式}可写成

\begin{equation*}
  \frac{dy}{dx} = \frac{ax+by+c}{\lambda(ax+by)+c_1}.
\end{equation*}
引入新变量$v=ax+by$,两边对$x$求导,则
\begin{equation*}
  \frac{dv}{dx} = a + b\frac{dy}{dx} \;\text{或}\; \frac{dy}{dx} = \frac{1}{b}(\frac{dv}{dx} - a).
\end{equation*}
于是方程\eqref{可化为齐次的方程通式}成为
\begin{equation*}
  \frac{1}{b}(\frac{dv}{dx} - a) = \frac{v+c}{\lambda v + c_1},
\end{equation*}
这是可分离变量的方程。

\paragraph{}
以上所介绍的方法可以应用于更一般的方程
\begin{equation*}
  \frac{dy}{dx} = f(\frac{ax+by+c}{a_1x+b_1y+c_1}).
\end{equation*}

  \section{一阶线性微分方程}
    \subsection{线性方程}
\paragraph{}
方程
\begin{equation}
  \label{一阶线性微分方程}
  \frac{dy}{dx} + P(x)y = Q(x)
\end{equation}
叫做\uwave{一阶线性微分方程},因为它对于未知函数$y$及齐导数是一次方程。如果$Q(x) \equiv 0$,则方程\eqref{一阶线性微分方程}称为\uwave{齐次}的;如果$Q(x) \not\equiv 0$,则方程\eqref{一阶线性微分方程}称为\uwave{非齐次}的。

\paragraph{}
设\eqref{一阶线性微分方程}为非齐次线性方程。为了求出非齐次线性方程\eqref{一阶线性微分方程}的解,我们先把$Q(x)$换成零而写出方程
\begin{equation}
  \label{一阶齐次线性微分方程}
  \frac{dy}{dx} + P(x)y = 0.
\end{equation}
方程\eqref{一阶齐次线性微分方程}叫做对应于非齐次线性方程\eqref{一阶线性微分方程}的\uwave{齐次线性方程}。方程\eqref{一阶齐次线性微分方程}是可分离变量的。分离变量后得
\begin{equation*}
  \frac{dy}{y} = -P(x)dx,
\end{equation*}
两端积分,得
\begin{equation*}
  \ln|y|=-\int P(x)dx+C_1,
\end{equation*}
或
\begin{equation*}
  y = Ce^{-\int P(x)dx} \; (C=\pm e^{C_1}),
\end{equation*}
这是对应的齐次线性方程\eqref{一阶齐次线性微分方程}的通解。

\paragraph{}
现在我们使用\href{https://www.cnblogs.com/lookof/archive/2009/01/06/1370065.html}{\color{blue}\uwave{常数变易法}}来求非齐次线性方程\eqref{一阶线性微分方程}的通解。这方法是把\eqref{一阶齐次线性微分方程}的通解中的$C$换成$x$的未知函数$u(x)$,即作变换

\begin{equation}
  \label{常数变易法变换1}
  y = ue^{-\int P(x)dx},
\end{equation}
于是
\begin{equation}
  \label{常数变易法变换2}
  \frac{dy}{dx} = u'e^{-\int P(x)dx} - uP(x)e^{-\int P(x)dx}.
\end{equation}
将\eqref{常数变易法变换1}和\eqref{常数变易法变换2}代入方程\eqref{一阶线性微分方程}得
\begin{equation*}
  u'e^{-\int P(x)dx} - uP(x)e^{-\int P(x)dx} + P(x)ue^{-\int P(x)dx} = Q(x),
\end{equation*}
即
\begin{equation*}
  u'e^{-\int P(x)dx} = Q(x), \;\; u' = Q(x)e^{\int P(x)dx}.
\end{equation*}
两端积分,得
\begin{equation*}
  u=\int Q(x)e^{\int P(x)dx}dx + C.
\end{equation*}
把上式代入\eqref{常数变易法变换1},便得非齐次线性方程\eqref{一阶线性微分方程}的通解
\begin{equation}
  \label{常数变易法变换3}
  y = e^{-\int P(x)dx}\big( \int Q(x)e^{P(x)dx}dx + C \big),
\end{equation}

\paragraph{}
将\eqref{常数变易法变换3}式改写成两项之和
\begin{equation*}
  y = Ce^{-\int P(x)dx} + e^{-\int P(x)dx}\int Q(x)e^{\int P(x)dx}dx,
\end{equation*}
上式右端第一项是对应的齐次线性方程\eqref{一阶齐次线性微分方程}的通解,第二项是非齐次线性方程\eqref{一阶线性微分方程}的一个特解(在\eqref{一阶线性微分方程}的通解\eqref{常数变易法变换3}中取$C=0$便得到这个特解)。由此可见,一阶非齐次线性方程的通解等于对应的齐次方程的通解与非齐次方程的一个特解之和。

\subsection{伯努利方程}
\paragraph{}
方程
\begin{equation}
  \label{伯努利方程}
  \frac{dy}{dx} + P(x)y = Q(x)y^n \;\; (n \neq 0,1)
\end{equation}
叫做\uwave{伯努利(Bernoulli)方程}。当$n=0$或$n=1$时,这是线性微分方程。当$n\neq 0, n\neq 1$时,这方程不是线性的,但是通过变量的代换,便可把它化为线性的。事实上,以$y^n$除方程\eqref{伯努利方程}的两端,得

\begin{equation}
  \label{伯努利方程转换1}
  y^{-n}\frac{dy}{dx} + P(x)y^{1-n}=Q(x).
\end{equation}
容易看出,上式左端第一项与$\displaystyle\frac{d}{dx}(y^{1-n})$只差一个常数因子$1-n$,因此我们引入新的因变量
\begin{equation*}
  z=y^{1-n},
\end{equation*}
那么
\begin{equation*}
  \frac{dz}{dx} = (1-n)y^{-n}\frac{dy}{dx}.
\end{equation*}
用$(1-n)$乘方程\eqref{伯努利方程转换1}的两端,再通过上述代换便得线性方程
\begin{equation*}
  \frac{dz}{dx} + (1-n)P(x)z = (1-n)Q(x).
\end{equation*}
求出这方程的通解后,以$y^{1-n}$代$z$便得到伯努利方程的通解。

  \section{可降解的高阶微分方程}
    二阶及二阶以上的微分方程,即所谓\uwave{高阶微分方程}。通过代换将它化成较低阶的方程来求解。

\subsection[y^(n)=f(x) 型的微分方程]{$y^{(n)}=f(x)$型的微分方程}
\paragraph{}
微分方程
\begin{equation}
  \label{y^(n)=f(x)型的微分方程}
  y^{(n)}=f(x)
\end{equation}
的右端仅含有自变量$x$。容易看出,只要把$y^{(n-1)}$作为新的未知函数,那么\eqref{y^(n)=f(x)型的微分方程}式就是新未知函数的一阶微分方程。两边积分,就得到一个$n-1$阶的微分方程
\begin{equation*}
  y^{(n-1)} = \int f(x)dx + C_1.
\end{equation*}

\paragraph{}
同理可得
\begin{equation*}
  y^{(n-2)} = \int\big[ \int f(x)dx + C_1 \big]dx + C_2.
\end{equation*}
依此法继续进行,接连积分$n$次,便得方程\eqref{y^(n)=f(x)型的微分方程}的含有$n$个任意常数的通解。

\subsection{$y''=f(x,y')$型的微分方程}
\paragraph{}
方程
\begin{equation}
  \label{y''=f(x,y')型的微分方程}
  y'' = f(x,y')
\end{equation}
的右端不显含未知函数$y$。如果我们设$y'=p$,那么
\begin{equation*}
  y'' = \frac{dp}{dx} = p',
\end{equation*}
而方程\eqref{y''=f(x,y')型的微分方程}就成为
\begin{equation*}
  p' = f(x,p).
\end{equation*}
这是一个关于变量$x, p$的一阶微分方程。设其通解为
\begin{equation*}
  p = \varphi(x,C_1),
\end{equation*}
但是$\displaystyle p = \frac{dy}{dx}$,因此又得到一个一阶微分方程
\begin{equation*}
  \frac{dy}{dx} = \varphi(x,C_1).
\end{equation*}
对它进行积分,便得方程\eqref{y''=f(x,y')型的微分方程}的通解为
\begin{equation*}
  y = \int\varphi(x,C_1)dx + C_2.
\end{equation*}

\subsection{$y''=f(y,y')$型的微分方程}
\paragraph{}
方程
\begin{equation}
  \label{y''=f(y,y')型的微分方程}
  y'' = f(y,y')
\end{equation}
中不明显地含自变量$x$。为了求出它的解。我们令$y'=p$,并利用复合函数的求导法则,把$y''$化为对$y$的导数,即
\begin{equation*}
  y'' = \frac{dp}{dx} = \frac{dp}{dy} \bigcdot \frac{dy}{dx} = p\frac{dp}{dy}.
\end{equation*}
这样,方程\eqref{y''=f(y,y')型的微分方程}就成为
\begin{equation*}
  p\frac{dp}{dy} = f(y,p).
\end{equation*}
这是一个关于变量$y, p$的一阶微分方程。设它的通解为
\begin{equation*}
  y' = p = \varphi(y,C_1),
\end{equation*}
分离变量并积分,便得方程\eqref{y''=f(y,y')型的微分方程}的通解为
\begin{equation*}
  \int\frac{dy}{\varphi(y,C_1)} = x + C_2.
\end{equation*}

  \section{高阶线性微分方程}
    % \subsection{二阶线性微分方程举例}
\subsection{线性微分方程的解的结构}
\paragraph{}
二阶齐次线性方程
\begin{equation}
  \label{二阶齐次线性方程}
  y'' + P(x)y' + Q(x)y = 0.
\end{equation}

\paragraph{}
\textbf{定理1~~} 如果函数$y_1(x)$与$y_2(x)$是方程\eqref{二阶齐次线性方程}的两个解,那么
\begin{equation}
  y = C_1y_1(x) + C_2y_2(x)
\end{equation}
也是\eqref{二阶齐次线性方程}的解,其中$C_1, C_2$是任意常数。

\paragraph{}
\textbf{定理2~~}如果$y_1(x)$与$y_2(x)$是方程\eqref{二阶齐次线性方程}的两个线性无关的特解,那么
\begin{equation*}
  y = C_1y_1(x) + C_2y_2(x) ~~ (C_1, C_2\text{是任意常数})
\end{equation*}
就是方程\eqref{二阶齐次线性方程}的通解。

\paragraph{}
\textbf{推论~~}如果$y_1(x), y_2(x), \cdots, y_n(x)$是$n$阶齐次线性方程
\begin{equation*}
  y^{(n)}+a_1(x)y^{(n-1)}+\cdots+a_{n-1}(x)y' + a_n(x)y = 0
\end{equation*}
的$n$个线性无关的解,那么,此方程的通解为
\begin{equation*}
  y=C_1y_1(x) + C_2y_2(x) + \cdots + C_ny_n(x),
\end{equation*}
其中$C_1,C_2,\cdots,C_n$为任意常数。

\paragraph{}
\textbf{定理3~~}设$y^*(x)$是二阶非齐次线性方程
\begin{equation}
  \label{定理3,二阶非齐次线性方程}
  y''+P(x)y'+Q(x)y = f(x)
\end{equation}
的一个特解。$Y(x)$是与\eqref{定理3,二阶非齐次线性方程}对应的齐次方程\eqref{二阶齐次线性方程}的通解,那么
\begin{equation}
  y = Y(x) + y^*(x)
\end{equation}
是二阶非齐次线性微分方程\eqref{定理3,二阶非齐次线性方程}的通解。

\paragraph{}
\textbf{定理4~~}设非齐次线性方程\eqref{定理3,二阶非齐次线性方程}的右端$f(x)$是两个函数之和,即
\begin{equation}
  y'' + P(x)y' + Q(x)y = f_1(x) + f_2(x),
\end{equation}
而$y_1^*(x)$与$y_2^*(x)$分别是方程
\begin{equation*}
  y'' + P(x)y' + Q(x)y = f_1(x)
\end{equation*}
与
\begin{equation*}
  y'' + P(x)y' + Q(x)y = f_2(x)
\end{equation*}
的特解,那么$y_1^*(x)+y_2^*(x)$就是原方程的特解。
\paragraph{}
这一定理通常称为线性微分方程的解的\uwave{叠加原理}。

\subsection{常数变易法}
\paragraph{}

  \section{常系数齐次线性微分方程}
    \input{7_constant_coefficient-homogeneous-linear}
  \section{常系数非齐次线性微分方程}
    \input{8_constant_coefficient-inhomogeneous-linear}
  \section{欧拉方程}
    \input{9_euler‘s_formula}
  \section{常系数线性微分方程组解法举例}
    \input{10_solving_examples}
\end{document}
