\paragraph{}
一阶微分方程$y'=f(x,y)$有时也写成如下的对称形式:
\begin{equation}
  \label{微分方程的对称形式}
  P(x,y)dx + Q(x,y)dy = 0.
\end{equation}

在方程\eqref{微分方程的对称形式}中,变量$x$与$y$对称,它既可看作是以$x$为自变量、$y$为因变量的方程
\begin{equation}
  \frac{dy}{dx} = -\frac{P(x,y)}{Q(x,y)}, \; (Q(x,y) \neq 0)
\end{equation}
也可看作是以$y$为自变量、$x$为因变量的方程
\begin{equation}
  \frac{dx}{dy} = -\frac{Q(x,y)}{P(x,y)}, \; (P(x,y) \neq 0)
\end{equation}

\paragraph{}
在第一节的例1中,一阶微分方程$\displaystyle \frac{dy}{dx}=2x$,或$dy=2xdx$,可以把两端都积分就得到方程的通解$y=x^2+C$。但是并不是所有的一阶微分方程都能这样求解。例如,对于一阶微分方程

\begin{equation}
  \label{二阶微分方程例子1}
  \frac{dy}{dx}=2xy^2
\end{equation}
就不能像上面那样用直接对两端积分的方法求出它的通解。原因是方程\eqref{二阶微分方程例子1}的右端含有与$x$存在函数关系的变量$y$,积分
\begin{equation}
  \int{2xy^2dx}
\end{equation}
求不出来。

\paragraph{}
为了解决这个困难,在方程\eqref{二阶微分方程例子1}的两端同时乘以$\displaystyle \frac{dx}{y^2}$,使方程\eqref{二阶微分方程例子1}变为
\begin{equation}
  \frac{dy}{y^2}=2xdx,
\end{equation}
这样,变量$x$与$y$已分离在等式的两端,然后两端积分得
\begin{align}
  -\frac{1}{y} \;=&\; x^2 + C, \\
  y \;=&\; -\frac{1}{x^2+C}
\end{align}
其中$C$是任意常数。

\paragraph{}
一般的,如果一个一阶微分方程能写成
\begin{equation}
  \label{可分离变量的微分方程}
  g(y)dy = f(x)dx
\end{equation}
的形式,就是说,能把微分方程写成一端只含$y$的函数和$dy$,另一端只含$x$的函数和$dx$,那么原方程就称为\uwave{可分离变量的微分方程}。

\paragraph{}
\textbf{证~~}假定方程\eqref{可分离变量的微分方程}中的函数$g(y)$和$f(x)$是连续的。设$y=\varphi(x)$是方程\eqref{可分离变量的微分方程}的解,将它代入\eqref{可分离变量的微分方程}中得到恒等式
\begin{equation}
  g[\varphi(x)]\varphi'(x)dx = f(x)dx.
\end{equation}
将上式两端积分,并由$y=\varphi(x)$引进变量$y$,得
\begin{equation}
  \int{g(y)dy} = \int{f(x)dx}.
\end{equation}

设$G(y)$及$F(x)$依次为$g(y)$及$f(x)$的原函数,于是有
\begin{equation}
  \label{可分离变量的微分方程的通解}
  G(y) = F(x) + C.
\end{equation}
因此,方程\eqref{可分离变量的微分方程}的解满足关系式\eqref{可分离变量的微分方程的通解}。

\paragraph{}
反之,如果$y=\Phi(x)$是由关系式\eqref{可分离变量的微分方程的通解}所确定的隐函数,那么在$g(y) \neq 0$的条件下,$y=\Phi(x)$也是方程\eqref{可分离变量的微分方程}的解,事实上,由隐函数的求导法可知,当$g(y) \neq 0$时,

\begin{equation}
  \phi'(x) = \frac{F'(x)}{G'(y)} = \frac{f(x)}{g(y)},
\end{equation}

这就表示函数$y=\Phi(x)$满足方程\eqref{可分离变量的微分方程}。所以,如果已分离变量的方程\eqref{可分离变量的微分方程}中,$g(y)$和$f(x)$是连续的,且$g(y) \neq 0$,那么\eqref{可分离变量的微分方程}式两端积分后得到的关系式\eqref{可分离变量的微分方程的通解},就用隐式给出了方程\eqref{可分离变量的微分方程}的解,\eqref{可分离变量的微分方程的通解}式就叫做微分方程\eqref{可分离变量的微分方程}的\uwave{隐式解}。

\paragraph{}
又由于关系式\eqref{可分离变量的微分方程的通解}中含有任意常数,因此\eqref{可分离变量的微分方程的通解}式所确定的隐函数是方程\eqref{可分离变量的微分方程}的通解,所以\eqref{可分离变量的微分方程的通解}式叫做微分方程\eqref{可分离变量的微分方程}的\uwave{隐式通解}(当$f(x)\neq0$时,\eqref{可分离变量的微分方程的通解}式所确定的隐函数$x=\Psi(y)$也可认为是方程\eqref{可分离变量的微分方程}的解)。
