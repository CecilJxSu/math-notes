二阶及二阶以上的微分方程,即所谓\uwave{高阶微分方程}。通过代换将它化成较低阶的方程来求解。

\subsection[y^(n)=f(x) 型的微分方程]{$y^{(n)}=f(x)$型的微分方程}
\paragraph{}
微分方程
\begin{equation}
  \label{y^(n)=f(x)型的微分方程}
  y^{(n)}=f(x)
\end{equation}
的右端仅含有自变量$x$。容易看出,只要把$y^{(n-1)}$作为新的未知函数,那么\eqref{y^(n)=f(x)型的微分方程}式就是新未知函数的一阶微分方程。两边积分,就得到一个$n-1$阶的微分方程
\begin{equation*}
  y^{(n-1)} = \int f(x)dx + C_1.
\end{equation*}

\paragraph{}
同理可得
\begin{equation*}
  y^{(n-2)} = \int\big[ \int f(x)dx + C_1 \big]dx + C_2.
\end{equation*}
依此法继续进行,接连积分$n$次,便得方程\eqref{y^(n)=f(x)型的微分方程}的含有$n$个任意常数的通解。

\subsection{$y''=f(x,y')$型的微分方程}
\paragraph{}
方程
\begin{equation}
  \label{y''=f(x,y')型的微分方程}
  y'' = f(x,y')
\end{equation}
的右端不显含未知函数$y$。如果我们设$y'=p$,那么
\begin{equation*}
  y'' = \frac{dp}{dx} = p',
\end{equation*}
而方程\eqref{y''=f(x,y')型的微分方程}就成为
\begin{equation*}
  p' = f(x,p).
\end{equation*}
这是一个关于变量$x, p$的一阶微分方程。设其通解为
\begin{equation*}
  p = \varphi(x,C_1),
\end{equation*}
但是$\displaystyle p = \frac{dy}{dx}$,因此又得到一个一阶微分方程
\begin{equation*}
  \frac{dy}{dx} = \varphi(x,C_1).
\end{equation*}
对它进行积分,便得方程\eqref{y''=f(x,y')型的微分方程}的通解为
\begin{equation*}
  y = \int\varphi(x,C_1)dx + C_2.
\end{equation*}

\subsection{$y''=f(y,y')$型的微分方程}
\paragraph{}
方程
\begin{equation}
  \label{y''=f(y,y')型的微分方程}
  y'' = f(y,y')
\end{equation}
中不明显地含自变量$x$。为了求出它的解。我们令$y'=p$,并利用复合函数的求导法则,把$y''$化为对$y$的导数,即
\begin{equation*}
  y'' = \frac{dp}{dx} = \frac{dp}{dy} \bigcdot \frac{dy}{dx} = p\frac{dp}{dy}.
\end{equation*}
这样,方程\eqref{y''=f(y,y')型的微分方程}就成为
\begin{equation*}
  p\frac{dp}{dy} = f(y,p).
\end{equation*}
这是一个关于变量$y, p$的一阶微分方程。设它的通解为
\begin{equation*}
  y' = p = \varphi(y,C_1),
\end{equation*}
分离变量并积分,便得方程\eqref{y''=f(y,y')型的微分方程}的通解为
\begin{equation*}
  \int\frac{dy}{\varphi(y,C_1)} = x + C_2.
\end{equation*}
