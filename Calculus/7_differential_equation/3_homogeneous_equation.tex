\subsection{齐次方程}
\paragraph{}
如果一阶微分方程可化成
\begin{equation}
  \label{齐次方程形式}
  \frac{dy}{dx} = \varphi(\frac{y}{x})
\end{equation}
的形式,那么就称这方程为\uwave{齐次方程},例如
\begin{equation*}
  (xy-y^2)dx - (x^2-2xy)dy = 0
\end{equation*}
是齐次方程,因为它可化成
\begin{align*}
  \frac{dy}{dx} \;=&\; \frac{xy-y^2}{x^2-2xy}, \\
  \;=&\; \frac{\frac{y}{x} - (\frac{y}{x})^2}{1 - 2(\frac{y}{x})}.
\end{align*}

\paragraph{}
在齐次方程
\begin{equation*}
  \frac{dy}{dx} = \varphi(\frac{y}{x})
\end{equation*}
中,引进新的未知函数
\begin{equation}
  \label{齐次方程换元函数}
  u = \frac{y}{x},
\end{equation}
就可把它化为可分离变量的方程。因为由\eqref{齐次方程换元函数}有

\begin{equation*}
  y = ux, \; \text{两边对$x$求导得} \; \frac{dy}{dx} = u + x\frac{du}{dx},
\end{equation*}
代入方程\eqref{齐次方程形式},便得方程
\begin{equation*}
  u + x\frac{du}{dx} = \varphi(u),
\end{equation*}
即
\begin{equation*}
    x\frac{du}{dx} = \varphi(u) - u.
\end{equation*}
分离变量,得
\begin{equation*}
  \frac{du}{\varphi(u) - u} = \frac{dx}{x}.
\end{equation*}
两端积分,得
\begin{equation*}
  \int\frac{du}{\varphi(u) - u} = \int\frac{dx}{x}.
\end{equation*}
求出积分后,再以$\displaystyle \frac{y}{x}$代替$u$,便得所给齐次方程的通解。

\subsubsection{可化为齐次的方程}
\paragraph{}
方程
\begin{equation}
  \label{可化为齐次的方程通式}
  \frac{dy}{dx} = \frac{ax+by+c}{a_1x+b_1y+c_1}
\end{equation}

当$c=c_1=0$时是齐次的,否则不是齐次的。在非齐次的情形,可用下列变换把它化为齐次方程:令
\begin{equation*}
  x = X + h, y = Y + k,
\end{equation*}
其中$h$及$k$是待定的常数,于是
\begin{equation*}
  dx = dX, dy = dY,
\end{equation*}
从而方程\eqref{可化为齐次的方程通式}成为
\begin{equation*}
  \frac{dY}{dX} = \frac{aX+bY+ah+bk+c}{a_1X+b_1Y+a_1h+b_1k+c_1}.
\end{equation*}

\paragraph{}
\setlength{\baselineskip}{30pt}
如果方程组
\begin{align*}
  \left\{\begin{array}{l}
    ah + bk + c = 0, \\
    a_1h+b_1k+c_1 = 0
  \end{array} \right.
\end{align*}
的系数行列式$\displaystyle \left|\begin{array}{ll}
  a & b \\ a_1 & b_1
\end{array} \right| \neq 0$,即$\displaystyle \frac{a_1}{a} \neq \frac{b_1}{b}$,那么可以定出$h$及$k$使它们满足上述方程组。这样,方程\eqref{可化为齐次的方程通式}便化为齐次方程
\begin{equation*}
  \frac{dY}{dX} = \frac{aX+bY}{a_1X+b_1Y}.
\end{equation*}
求出这齐次方程的通解后,在通解中以$x-h$代$X$,$y-k$代$Y$,便得方程\eqref{可化为齐次的方程通式}的通解。

\paragraph{}
当$\displaystyle \frac{a_1}{a}=\frac{b_1}{b}$时,$h$及$k$无法求得,因此上述方法不能应用。但这时令$\displaystyle \frac{a_1}{a}=\frac{b_1}{b}=\lambda$,从而方程\eqref{可化为齐次的方程通式}可写成

\begin{equation*}
  \frac{dy}{dx} = \frac{ax+by+c}{\lambda(ax+by)+c_1}.
\end{equation*}
引入新变量$v=ax+by$,两边对$x$求导,则
\begin{equation*}
  \frac{dv}{dx} = a + b\frac{dy}{dx} \;\text{或}\; \frac{dy}{dx} = \frac{1}{b}(\frac{dv}{dx} - a).
\end{equation*}
于是方程\eqref{可化为齐次的方程通式}成为
\begin{equation*}
  \frac{1}{b}(\frac{dv}{dx} - a) = \frac{v+c}{\lambda v + c_1},
\end{equation*}
这是可分离变量的方程。

\paragraph{}
以上所介绍的方法可以应用于更一般的方程
\begin{equation*}
  \frac{dy}{dx} = f(\frac{ax+by+c}{a_1x+b_1y+c_1}).
\end{equation*}
