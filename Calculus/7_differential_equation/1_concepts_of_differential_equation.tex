\paragraph{}
函数是客观事物的内部联系在数量方面的反映,利用函数关系又可以对客观事物的规律性进行研究。因此如何寻求函数关系,在实践中具有重要意义。

\paragraph{}
在许多问题中,往往不能直接找出所需要的函数关系,但是根据问题所提供的情况,有时可以列出含有要找的函数及其导数的关系式。这样的关系式就是所谓\uwave{微分方程}。

\paragraph{}
微分方程建立后,对它进行研究,找出未知函数来,这就是\uwave{解微分方程}。

\subsection{例子}
\subsubsection{例1}
\paragraph{}
\textbf{例1\;}一曲线通过点$(1,2)$,且在该曲线上任一点$M(x,y)$处的切线的斜率为$2x$,求这曲线的方程。

\paragraph{}
\textbf{解\;}设所求曲线的方程为$y=\varphi(x)$。根据导数的几何意义,可知未知函数$y=\varphi(x)$应满足关系式

\begin{equation}
  \label{微分方程概念例1公式1}
  \frac{dy}{dx} = 2x.
\end{equation}

此外,未知函数$y=\varphi(x)$还应满足下列条件:

\begin{equation}
  \label{微分方程概念例1公式2}
  x = 1 \text{时}, y = 2.
\end{equation}

\paragraph{}
把\eqref{微分方程概念例1公式1}式两端积分,得

\begin{equation}
  \label{微分方程概念例1公式3}
  y = \int 2xdx \; \text{即} \; y=x^2 + C.
\end{equation}

其中$C$是任意常数。

\paragraph{}
把条件“$x=1$时$, y=2$”代入\eqref{微分方程概念例1公式3}式,得

\begin{equation}
  \label{微分方程概念例1公式4}
  2 = 1^2 + C,
\end{equation}

由此得出$C=1$。把$C=1$代入\eqref{微分方程概念例1公式3}式,即得所求曲线方程

\begin{equation}
  \label{微分方程概念例1公式5}
  y = x^2 + 1.
\end{equation}

\subsubsection{例2}
\paragraph{}
\textbf{例2\;}列车在平直线路上以$20m/s$(相当于$72 km/h$)的速度行驶;当制动时列车获得加速度$-0.4m/s^2$。问开始制动后多少时间列车才能停住以及列车在这段时间里行驶了多少路程?

\paragraph{}
\textbf{解\;}设列车在开始制动后$t s$时行驶了$s m$。根据题意,反映制动阶段列车运动规律的函数$s=s(t)$应满足关系式
\begin{equation}
  \label{微分方程概念例2公式1}
  \frac{d^2s}{dt^2} = -0.4.
\end{equation}
此外,未知函数$s=s(t)$还应满足下列条件:
\begin{equation}
  \label{微分方程概念例2公式2}
  t = 0 \text{时}, s = 0, v = \frac{ds}{dt} = 20.
\end{equation}

\paragraph{}
把\eqref{微分方程概念例2公式1}式两端积分一次,得
\begin{equation}
  \label{微分方程概念例2公式3}
  v = \frac{ds}{dt} = -0.4t + C_1;
\end{equation}
再积分一次,得
\begin{equation}
  \label{微分方程概念例2公式4}
  s = -0.2t^2 + C_1t + C_2,
\end{equation}
这里$C_1, C_2$都是任意常数。

\paragraph{}
把条件“$t = 0$时$,v=20$”代入\eqref{微分方程概念例2公式3}式,得
\begin{equation}
  \label{微分方程概念例2公式5}
  20 = C_1;
\end{equation}

\paragraph{}
把条件“$t = 0$时$,s=0$”代入\eqref{微分方程概念例2公式4}式,得
\begin{equation}
  \label{微分方程概念例2公式6}
  0 = C_2;
\end{equation}
把$C_1,C_2$的值代入\eqref{微分方程概念例2公式3}及\eqref{微分方程概念例2公式4},得
\begin{align}
    \label{微分方程概念例2公式7}
    v \;=&\; -0.4t + 20, \\
    \label{微分方程概念例2公式8}
    s \;=&\; -0.2t^2 + 20t.
\end{align}

\paragraph{}
在\eqref{微分方程概念例2公式7}式中令$v=0$,得到列车从开始制动到完全停住所需的时间
\begin{equation}
  \label{微分方程概念例2公式9}
  t = \frac{20}{0.4} = 50(s).
\end{equation}
再把$t=50$代入\eqref{微分方程概念例2公式8}式,得到列车在制动阶段行驶的路程
\begin{equation}
\label{微分方程概念例2公式10}
s = -0.2\times50^2+20\times50=500(m).
\end{equation}

\subsection{基本概念}
\subsubsection{微分方程和它的阶}
\paragraph{}
上述两个例子中的关系式\eqref{微分方程概念例1公式1}和\eqref{微分方程概念例2公式1}都含有未知函数的导数,它们都是微分方程。

\paragraph{}
一般的,凡表示未知函数、未知函数的导数与自变量之间的关系的方程,叫做\uwave{微分方程},有时也简称\uwave{方程}。

\paragraph{}
微分方程中所出现的未知函数的最高阶导数的阶数,叫做\uwave{微分方程的阶}。例如,方程\eqref{微分方程概念例1公式1}是一阶微分方程;\eqref{微分方程概念例2公式1}是二阶微分方程。又如,方程
\begin{equation}
x^3y''' + x^2y'' - 4xy' = 3x^2,
\end{equation}
是三阶微分方程;方程
\begin{equation}
y^{(4)} - 4y''' + 10y'' - 12y' + 5y = \sin{2x},
\end{equation}
是四阶微分方程。

\subsubsection{$n$阶微分方程}
\paragraph{}
一般的,$n$阶微分方程的形式是

\begin{equation}
\label{n阶微分方程的形式}
F(x,y,y',\cdots,y^{(n)})=0.
\end{equation}

这里必须指出,在方程\eqref{n阶微分方程的形式}中,$y^{(n)}$是必须出现的,而$x,y,y',\cdots,y^{(n-1)}$等变量则可以不出现。例如$n$阶微分方程

\begin{equation}
y^{(n)} + 1 = 0,
\end{equation}

中,除$y^{(n)}$外,其它变量都是没有出现。

\paragraph{}
如果能从方程\eqref{n阶微分方程的形式}中解出最高阶导数,则可得微分方程
\begin{equation}
  y^{(n)} = f(x,y,y',\cdots,y^{(n-1)}).
\end{equation}

\subsubsection{微分方程的解}
\paragraph{}
在研究某些实际问题时,首先要建立微分方程,然后找出满足微分方程的函数(解微分方程),就是说,找出这样的函数,把这函数代入微分方程能使该方程成为恒等式。这个函数就叫做该\uwave{微分方程的解}。

\paragraph{}
确切地说,设函数$y=\varphi(x)$在区间$I$上有$n$阶连续导数,如果在区间$I$上,
\begin{equation}
  F[x,\varphi(x),\varphi'(x),\cdots,\varphi^{(n)}(x) \equiv 0],
\end{equation}
那么函数$y=\varphi(x)$就叫做微分方程\eqref{n阶微分方程的形式}在区间$I$上的解;

\paragraph{}
如果微分方程的解中含有任意常数,且任意常数的个数与微分方程的阶数相同,这样的解叫做\uwave{微分方程的通解}。

\paragraph{}
例如,函数\eqref{微分方程概念例1公式3}是方程\eqref{微分方程概念例1公式1}的解,它含有一个任意常数,而方程\eqref{微分方程概念例1公式1}是一阶的,所以函数\eqref{微分方程概念例1公式3}是方程\eqref{微分方程概念例1公式1}的通解。

\paragraph{}
又如,函数\eqref{微分方程概念例2公式4}是方程\eqref{微分方程概念例2公式1}的解,它含有两个任意常数,而方程\eqref{微分方程概念例2公式1}是二阶的,所以函数\eqref{微分方程概念例2公式4}是方程\eqref{微分方程概念例2公式1}的通解。

\subsubsection{初始条件和特解}
\paragraph{}
由于通解中含有任意常数,所以它还不能完全确定地反映某一客观事物的规律性。要完全确定地反映客观事物的规律性,必须确定这些常数的值。

\paragraph{}
设微分方程中的未知函数为$y=\varphi(x)$,如果微分方程是一阶的,通常用来确定任意常数的条件是

\begin{equation}
  x = x_0 \text{时}, y = y_0,
\end{equation}
或写成
\begin{equation}
  y|_{x=x_0} = y_0,
\end{equation}

其中$x_0, y_0$都是给定的值;如果微分方程是二阶的,通常用来确定任意常数的条件是
\begin{equation}
  x=x_0 \text{时}, y=y_0, y'=y'_0,
\end{equation}
或写成
\begin{equation}
  y|_{x=x_0} = y_0, y'|_{x=x_0} = y'_0,
\end{equation}
其中$x_0, y_0$和$y'_0$都是给定的值。上述这种条件叫做\uwave{初始条件}。

\paragraph{}
确定了通解中的任意常数以后,就得到\uwave{微分方程的特解}。例如\eqref{微分方程概念例1公式5}式是方程\eqref{微分方程概念例1公式1}满足条件\eqref{微分方程概念例1公式2}的特解;\eqref{微分方程概念例2公式8}式是方程\eqref{微分方程概念例2公式1}满足条件\eqref{微分方程概念例2公式2}的特解

\subsubsection{初值问题和积分曲线}
\paragraph{}
求微分方程$y'=f(x,y)$满足初始条件$y|_{x=x_0}=y_0$的特解这样一个问题,叫做一阶微分方程的\uwave{初值问题},记作

\begin{align}
  \label{一阶初值问题}
  \left\{\begin{array}{l}
    y'=f(x,y), \\
    y|_{x=x_0} = y_0.
  \end{array} \right.
\end{align}

\paragraph{}
微分方程的解的图形是一条曲线,叫做\uwave{微分方程的积分曲线}。初值问题\eqref{一阶初值问题}的几何意义,就是求微分方程通过点$(x_0,y_0)$的那条积分曲线。二阶微分方程的初值问题

\begin{align}
  \label{二阶初值问题}
  \left\{\begin{array}{l}
    y''=f(x,y,y'), \\
    y|_{x=x_0} = y_0, y'|_{x=x_0} = y'_0
  \end{array} \right.
\end{align}
的几何意义,是求微分方程通过点$(x_0,y_0)$且在该点处的切线斜率为$y'_0$的那条积分曲线。
