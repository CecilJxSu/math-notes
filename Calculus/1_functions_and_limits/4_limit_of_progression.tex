\subsection{数列}
\paragraph{}
如果按照某一法则,对每个$n \in N^+$,对应着一个确定的实数$x_n$,这些实数$x_n$按照下标$n$ 从小到大排序得到的一个序列,

\begin{equation}
x_1, x_2, x_3, ..., x_n, ...
\end{equation}

\paragraph{}
就叫数列,简记为数列 $\{x_n\}$。

\paragraph{}
$x_n$ 为一般项。

\subsection{数列极限}
\paragraph{}
设 $\{x_n\}$ 为一数列,如果存在常数 $a$,对于任意给定的正整数 $\varepsilon$(无论它多么小),总存在正整数 $N$ ,使得当 $n > N$ 时,不等式

\begin{equation}
|x_n - a| < \varepsilon,
\end{equation}

\paragraph{}
都成立,那么就称常数 $a$ 是数列 $\{x_n\}$ 的极限,或者称数列 $\{x_n\}$ 收敛于 $a$,记为

\begin{equation}
\lim_{n \to \infty} x_n = a,
\end{equation}

\paragraph{}
或

\begin{equation}
x_n \to a (n \to \infty).
\end{equation}

\paragraph{}
如果不存在这样的常数 $a$,就说数列 $\{x_n\}$ 是发散的。

\subsection{收敛数列的性质}
\subsubsection{极限的唯一性}
\paragraph{}
如果数列 $\{x_n\}$ 收敛,那么它的极限值唯一。

\subsubsection{收敛数列的有界性}
\paragraph{}
如果数列 $\{x_n\}$ 收敛,那么数列 $\{x_n\}$ 一定有界。

\subsubsection{收敛数列的保号性}
\paragraph{}
如果 $\displaystyle{\lim_{n \to \infty}} x_n = a$,且 $a > 0$(或 $a < 0$),那么存在正整数 $N > 0$,当 $n > N$ 时,都有 $x_n > 0$(或 $x_n < 0$)。

\subsubsection{收敛数列与其子数列间的关系}
\paragraph{}
如果数列 $\{x_n\}$ 收敛于 $a$,那么它的任一子数列也收敛,且极限也是 $a$。
