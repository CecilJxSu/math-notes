\subsection{连续性}
\subsubsection{增量}
\paragraph{}
设变量 $u$ 从它的一个初值 $u_1$ 变到终值 $u_2$,终值与初值的差 $u_2 - u_1$ 就叫做变量 $u$ 的增量,记作 $\Delta u$,即

\begin{equation}
\Delta u = u_2 - u_1
\end{equation}

\paragraph{}
增量可以正,也可以负

\paragraph{}
$x$ 从 $x_0$ 变到 $x_0 + \Delta x$ 时,函数 $y$ 相应的从 $f(x_0)$ 变到 $f(x_0 + \Delta x)$,因此函数 $y$ 的增量为:

\begin{equation}
\Delta y = f(x_0 + \Delta x) - f(x_0)
\end{equation}

\subsubsection{连续性定义}
\paragraph{}
\textbf{定义:}设函数 $y = f(x)$ 在点 $x_0$ 的某一邻域内有定义,如果
\begin{gather}
\lim_{\Delta x \to 0}\Delta y = \lim_{\Delta x \to 0}[f(x_0 + \Delta x) - f(x_0)] = 0, \\
\text{或} \lim_{x \to x_0} f(x) = f(x_0).
\end{gather}

\paragraph{}
那么就称函数 $y = f(x)$ 在点 $x_0$ 连续。

\paragraph{}
\textbf{左连续:}如果 $\lim_{x \to x_0^-} f(x) = f(x_0^-)$ 存在且等于 $f(x_0)$,即

\begin{equation}
f(x_0^-) = f(x_0)
\end{equation}

\paragraph{}
\textbf{右连续:}如果 $\lim_{x \to x_0^+}f(x) = f(x_0^+)$ 存在且等于 $f(x_0)$,即

\begin{equation}
f(x_0^+) = f(x_0)
\end{equation}

\paragraph{}
在区间上每一点都连续的函数,叫做函数在该区间上连续。

\subsection{间断点}
\paragraph{}
设函数 $f(x)$ 在点 $x_0$ 的某去心邻域内有定义。在此前提,如果函数 $f(x)$ 有下列三种情形之一:

\begin{enumerate}
  \item 在 $x = x_0$ 没有定义
  \item 虽在 $x = x_0$ 有定义,但 $\lim_{x \to x_0}f(x)$ 不存在
  \item 前 2 点都成立,但 $\lim_{x \to x_0}f(x) \neq f(x_0)$
\end{enumerate}

\paragraph{}
则函数 $f(x)$ 在点 $x_0$ 为不连续,而点 $x_0$ 称为函数 $f(x)$ 的间断点。
