\subsection{有界性与最大值最小值定理}
\paragraph{}
对于在区间 $I$ 上有定义的函数 $f(x)$,如果有 $x_0 \in I$,使得对于任一 $x \in I$ 都有

\begin{equation}
f(x) \leq f(x_0) (f(x) \geq f(x_0)),
\end{equation}

\paragraph{}
则称 $f(x_0)$ 是函数 $f(x)$ 在区间 $I$ 上的\textbf{最大值}(\textbf{最小值})。

\paragraph{}
\textbf{有界性与最大值最小值定理\;} 在闭区间上连续的函数在该区间上有界且一定能取得它的最大值和最小值。

\subsection{零点定理与介值定理}
\paragraph{}
\textbf{零点定理\;} 设函数 $f(x)$ 在闭区间 $[a,b]$ 上连续,且 $f(a)$ 与 $f(b)$ 异号(即 $f(a) \cdot f(b) < 0$),那么在开区间 $(a,b)$ 内至少有一点 $\xi$,使

\begin{equation}
f(\xi) = 0.
\end{equation}

\paragraph{}
\textbf{介值定理\;} 设函数 $f(x)$ 在闭区间 $[a,b]$ 上连续,且在这区间的端点取不同的函数值 $f(a) = A$ 及 $f(b) = B$,那么,对于 $A$ 与 $B$ 之间的任意一个数 $C$,在开区间 $(a,b)$ 内至少有一点 $\xi$,使得

\begin{equation}
f(\xi) = C (a < \xi < b).
\end{equation}

\paragraph{}
\textbf{推论\;} 在闭区间上连续的函数必取得介于最大值 $M$ 与最小值 $m$ 之间的任何值。

\subsection{一致连续性}
\paragraph{}
\textbf{定义\;} 设函数 $f(x)$ 在区间 $I$ 上有定义。如果对于任意给定的正数 $\varepsilon$,总存在着正数 $\delta$,使得对于区间 $I$ 上的任意两点 $x_1, x_2$,当 $|x_1 - x_2| < \delta$ 时,就有

\begin{equation}
|f(x_1) - f(x_2)| < \varepsilon,
\end{equation}

\paragraph{}
那么称函数 $f(x)$ 在区间 $I$ 上是一致性连续的。

\paragraph{}
\textbf{一致连续性定理\;} 如果函数 $f(x)$ 在闭区间 $[a,b]$ 上连续,那么它在该区间上一致连续。
