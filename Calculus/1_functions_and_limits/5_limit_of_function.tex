\subsection{自变量趋于有限值时函数的极限}
\paragraph{}
设函数 $f(x)$ 在点 $x_0$ 的某一去心邻域内有定义,如果存在常数 $A$,对于任意给定的正数 $\varepsilon$(无论它多么小),总存在正数 $\delta$,使得当 $x$ 满足不等式 $0 < |x - x_0| < \delta$ 时,对应的函数值 $f(x)$ 都满足不等式

\begin{equation}
|f(x) - A| < \varepsilon,
\end{equation}

\paragraph{}
那么常数 $A$ 就叫做函数 $f(x)$ 当 $x \to x_0$ 时的极限,记作

\begin{equation}
\lim_{x \to x_0} f(x) = A,
\end{equation}

\paragraph{}
或

\begin{equation}
f(x) \to A(\text{当~~} x \to x_0).
\end{equation}

\subsubsection{单侧极限}
\paragraph{}
从 $x_0$ 的左侧或右侧趋近,即 $x_0 - \delta < x < x_0$ 或 $x_0 < x < x_0 + \delta$:

\begin{enumerate}
  \item 左极限:$\displaystyle{\lim_{x \to {x_0}^-}} f(x) = A \text{~~或~~} f({x_0}^-) = A$
  \item 右极限:$\displaystyle{\lim_{x \to {x_0}^+}} f(x) = A \text{~~或~~} f({x_0}^+) = A$
\end{enumerate}

\subsection{自变量趋于无穷大时函数的极限}
\paragraph{}
设函数 $f(x)$ 当 $|x|$ 大于某一正数时有定义,如果存在常数 $A$,对于任意给定的正整数 $\varepsilon$(无论它多么小),总存在着正数 $X$,使得当 $x$ 满足不等式 $|x| > X$ 时,对应的函数值 $f(x)$ 都满足不等式

\begin{equation}
|f(x) - A| < \varepsilon,
\end{equation}

\paragraph{}
那么常数 $A$ 就叫做函数 $f(x)$ 当 $x \to \infty$ 时的极限,记作

\begin{equation}
\lim_{x \to \infty} f(x) = A \text{~~或~~} f(x) \to A (\text{当~~} x \to \infty).
\end{equation}

\subsection{函数极限的性质}
\paragraph{}
下面的性质仅以 $\displaystyle{\lim_{x \to x_0}}f(x)$ 形式为代表给出的性质。

\subsubsection{函数极限的唯一性}
\paragraph{}
如果 $\displaystyle{\lim_{x \to x_0}}f(x)$ 存在,那么这极限唯一。

\subsubsection{函数极限的局部有界性}
\paragraph{}
如果 $\displaystyle{\lim_{x \to x_0}}f(x) = A$,那么存在常数 $M > 0$ 和 $\delta > 0$,使得当 $0 < |x - x_0| < \delta$ 时,有 $|f(x)| \leq M$.

\subsubsection{函数极限的局部保号性}
\paragraph{}
如果 $\displaystyle{\lim_{x \to x_0}}f(x) = A$,且 $A > 0 (\text{或~~} A < 0)$,那么存在常数 $\delta > 0$,使得当 $0 < |x - x_0| < \delta$ 时,有 $f(x) > 0 (\text{或~~} f(x) < 0)$。

\subsubsection{函数极限与数列极限的关系}
\paragraph{}
如果极限 $\displaystyle{\lim_{x \to x_0}}f(x)$ 存在,$\{x_n\}$ 为函数 $f(x)$ 的定义域内任一收敛于 $x_0$ 的数列,且满足 $x_n \neq x_0(n \in N^+)$,那么相应的函数值数列 $\{f(x_n)\}$ 必收敛,且 $\displaystyle{\lim_{n \to \infty}f(x_n) = \lim_{x \to x_0}f(x)}$。
