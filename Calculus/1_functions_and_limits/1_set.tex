\subsection{表示法}
\begin{enumerate}
  \item 列举法:$A = \{a_1, a_2, ..., a_n\}$
  \item 描述法:$M = \{x | \text{x具有性质P} \}$
\end{enumerate}

\subsection{数集}
\paragraph{}
在字母右上角标 “*”,表示排除 0 元素的数集;
\paragraph{}
在字母右上角标 “+”,表示排除 0 元素与负数的数集。
\paragraph{}
例子:自然数集 N 和整数集 Z

\begin{gather}
N^* = \{1, 2, ..., n, ...\} \\
Z^+ = \{1, 2, ..., n, ...\}
\end{gather}

\subsection{集合关系}
\paragraph{}
子集、真子集、相等、空集(是任何集合的子集)

\subsection{集合运算}
\paragraph{}
并、交、差(\textbackslash)
\paragraph{}
差运算;如果 I 是全集(基本集),也相当于求 A 的补集(余集):

\begin{equation}
I \backslash A = A^c = \{x | x \in I, x \notin A\}
\end{equation}

\paragraph{}
$A^c$ 是补集

\subsubsection{法则}
\begin{enumerate}
  \item 交换律:
    \begin{enumerate}
      \item $A \cup B = B \cup A$
      \item $A \cap B = B \cap A$
    \end{enumerate}
  \item 结合律:
    \begin{enumerate}
      \item $(A \cup B) \cup C = A \cup (B \cup C)$
      \item $(A \cap B) \cap C = A \cap (B \cap C)$
    \end{enumerate}
  \item 分配律:
    \begin{enumerate}
      \item $(A \cup B) \cap C = (A \cap C) \cup (B \cap C)$
      \item $(A \cap B) \cup C = (A \cup C) \cap (B \cup C)$
    \end{enumerate}
  \item 对偶律:
    \begin{enumerate}
      \item ${(A \cup B)}^c = A^c \cap B^c$
      \item ${(A \cap B)}^c = A^c \cup B^c$
    \end{enumerate}
\end{enumerate}

\subsubsection{直积}
\begin{equation}
A \times B = \{(x, y) | x \in A, y \in B\}
\end{equation}

\subsection{区间和邻域}

\subsubsection{区间}
\paragraph{}
(a, b)、[a, b]、[a, b)、(a, b]
\subsubsection{邻域}

\paragraph{}
邻域:
\begin{equation}
U(a, \delta) = \{x | a - \delta < x < a + \delta\}
\end{equation}

\paragraph{}
去心邻域:
\begin{equation}
\mathring{U}(a, \delta) = \{x | 0 < |x - a| < \delta\}
\end{equation}
