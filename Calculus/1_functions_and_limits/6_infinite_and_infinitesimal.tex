\subsection{无穷小}
\paragraph{}
如果函数 $f(x)$ 当 $x \to x_0(\text{或~~} x \to \infty)$ 时的极限为零,那么称函数 $f(x)$ 为当 $x \to x_0(\text{或~~} x \to \infty)$ 时的无穷小。

\paragraph{}
无穷小与函数极限的关系:

\paragraph{}
\textbf{定理1} 在自变量的同一变化过程 $x \to x_0(\text{或~~} x \to \infty)$ 中,函数 $f(x)$ 具有极限 $A$ 的充分必要条件是 $f(x) = A + a$,其中 $a$ 是无穷小.

\subsection{无穷大}
\paragraph{}
设函数 $f(x )$ 在 $x_0$ 的某一去心邻域内有定义(或 $|x|$ 大于某一正数时有定义),如果对于任意给定的正数 $M$(无论它多么大),总存在正数 $\delta$(或正数 $X$),只要 $x$ 适合不等式 $0 < |x - x_0| < \delta (\text{或} |x| > X)$,对应的函数值 $f(x)$ 总满足不等式

\begin{equation}
|f(x)| > M,
\end{equation}

\paragraph{}
则称函数 $f(x)$ 为当 $x \to x_0(\text{或} x \to \infty)$ 时的无穷大,记作

\begin{gather}
\lim_{x \to x_0}f(x) = \infty \\
(\text{或} \lim_{x \to \infty}f(x) = \infty).
\end{gather}

\paragraph{}
正无穷大和负无穷大:

\begin{gather}
f(x) > M, \lim_{x \to x_0 (\text{或}x \to \infty)} f(x) = + \infty; \\
f(x) < - M, \lim_{x \to x_0 (\text{或}x \to \infty)} f(x) = - \infty.
\end{gather}

\paragraph{}
\textbf{定理2} 在自变量的同一变化过程中,如果 $f(x)$ 为无穷大,则 $\frac{1}{f(x)}$ 为无穷小;反之,如果 $f(x)$ 为无穷小,且 $f(x) \neq 0$,则 $\frac{1}{f(x)}$ 为无穷大.
