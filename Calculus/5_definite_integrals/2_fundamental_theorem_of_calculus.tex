\subsection{变速直线运动中位置函数与速度函数之间的联系}
\paragraph{}
有一物体在一直线上运动。在这直线上取定原点、正向及长度单位,使它成一数轴。设时刻$t$时物体所在位置为$s(t)$,速度为$v(t)$(为了讨论方便起见,可以设$v(t) \geq 0$)。

\paragraph{}
从第一节知道:物体在时间间隔$[T_1,T_2]$内经过的路程可以用速度函数$v(t)$在$[T_1,T_2]$上的定积分

\begin{equation*}
  \int_{T_1}^{T_2}v(t)dt
\end{equation*}
来表达;另一方面,这段路程又可以通过位置函数$s(t)$在区间$[T_1,T_2]$上的增量

\begin{equation*}
  s(T_2) - s(T_1)
\end{equation*}

来表达。由此可见,位置函数$s(t)$与速度函数$v(t)$之间有如下关系:
\begin{equation}
  \label{变速直线运动速度定积分与路程关系}
  \int_{T_1}^{T_2}v(t)dt = s(T_2) - s(T_1).
\end{equation}

\paragraph{}
因为$s'(t)=v(t)$,即位置函数$s(t)$是速度函数$v(t)$的原函数,所以关系式\eqref{变速直线运动速度定积分与路程关系}表示,速度函数$v(t)$在区间$[T_1,T_2]$上的定积分等于$v(t)$的原函数$s(t)$在区间$[T_1,T_2]$上的增量
\begin{equation*}
  s(T_2) - s(T_1).
\end{equation*}

\paragraph{}
上述讨论的问题具有特殊性。因此,在\secref{sec:牛顿-莱布尼茨公式}下面进一步论证普遍性的函数$f(x)$在区间$[a,b]$上的积分等于原函数在区间$[a,b]$上的增量

\begin{equation*}
  F(b) - F(a).
\end{equation*}

\subsection{积分上限的函数及其导数}
\paragraph{}
设函数$f(x)$在区间$[a,b]$上连续,并且设$x$为$[a,b]$上的一点。我们考察$f(x)$在部分区间$[a,x]$上的定积分
\begin{equation*}
  \int_a^xf(x)dx.
\end{equation*}

\paragraph{}
由于$f(x)$在$[a,x]$上仍旧连续,因此这个定积分存在。这里,$x$即表示定积分的上限,又表示积分变量。因为定积分与积分变量的记法无关,所以,为了不混淆起见,可以把积分变量改用其它符号,例如用$t$表示,则上面的定积分可以写成

\begin{equation*}
  \int_a^xf(t)dt.
\end{equation*}

\paragraph{}
如果上限$x$在区间$[a,b]$上任意变动,则对于每一个取定的$x$值,定积分有一个对应值,所以它在$[a,b]$上定义了一个函数,记作$\Phi(x)$:
\begin{equation*}
  \Phi(x) = \int_a^xf(t)dt \; (a\leq x \leq b).
\end{equation*}

\paragraph{}
\textbf{定理1\;}如果函数$f(x)$在区间$[a,b]$上连续,则积分上限的函数
\begin{equation}
  \Phi(x) = \int_a^xf(t)dt
\end{equation}
在$[a,b]$上可导,并且它的导数
\begin{equation}
  \Phi'(x) = \frac{d}{dx}\int_a^xf(t)dt = f(x) \; (a \leq x \leq b).
\end{equation}

\paragraph{}
\textbf{证\;}若$x\in(a,b)$,设$x$获得增量$\Delta x$,其绝对值足够地小,使得$x+\Delta x\in(a,b)$,则$\Phi(x)$在$x+\Delta x$处的函数值为

\begin{equation*}
  \Phi(x+\Delta x) = \int_a^{x+\Delta x}f(t)dt.
\end{equation*}

\begin{figure}[H]
\centering
  % 积分上限的函数证明
\begin{tikzpicture}[scale=0.8]
  \begin{axis}[clip=false,xmin=0, xmax=13,ymin=0,ymax=13, grid=none,
    xtick=\empty, ytick=\empty, axis lines=middle,
    smooth, xlabel={$x$}, ylabel={$y$}]

    % 曲线:python, (x/3.5-1)**3-2*(x/3.5-1)**2+6
    \addplot[draw=red,domain=1:12] {(x/3.5-1)^3-2*(x/3.5-1)^2+8};

    \addplot[domain=1:7,pattern=north east lines,draw opacity=0] {(x/3.5-1)^3-2*(x/3.5-1)^2+8} \closedcycle;

    \node [above] at (3.4,7.956) {$y=f(x)$};

    \draw (1,0) -- (1,6.615);
    \draw (12,0) -- (12,10.53);
    \draw (7,0) -- (7,7);
    \draw [dashed] (8.5,0) -- (8.5,6.83);
    \draw (10,0) -- (10,7.51);

    \node [below] at (1,0) {$a$};
    \node [below] at (12,0) {$b$};
    \node [below] at (7,0) {$x$};
    \node [below] at (8.5,0) {$\xi$};
    \node [below] at (10.2,0) {$x+\Delta x$};
    \node [fill=white] at (8.5,3.3) {$f(\xi)$};
    \node [fill=white] at (4,3.3) {$\Phi(x)$};

    % 原点
    \node [below left] at (0,0) {$O$};
  \end{axis}
\end{tikzpicture}

  \caption{积分上限的函数证明}
  \label{积分上限的函数证明}
\end{figure}

由此得函数的增量
\begin{align*}
\Delta\Phi \;=&\; \Phi(x+\Delta x) - \Phi(x) \\
  =&\; \int_a^{x+\Delta x}f(t)dt - \int_a^xf(t)dt \\
  =&\; \int_a^xf(t)dt + \int_x^{x+\Delta x}f(t)dt - \int_a^xf(t)dt \\
  =&\; \int_x^{x+\Delta x}f(t)dt.
\end{align*}
再应用积分中值定理,即有等式
\begin{equation*}
  \Delta\Phi = f(\xi)\Delta x,
\end{equation*}
这里,$\xi$在$x$与$x+\Delta x$之间,把上式两端各除以$\Delta x$,得函数增量与自变量增量的比值
\begin{equation*}
  \frac{\Delta\Phi}{\Delta x} = f(\xi).
\end{equation*}

\paragraph{}
由于假设$f(x)$在$[a,b]$上连续,而$\Delta x \to 0$时,$\xi \to x$,因此$\displaystyle\lim_{\Delta x \to 0}=f(x)$。于是,令$\Delta x \to 0$对上式两端取极限时,左端的极限也应该存在且等于$f(x)$。这就是说,函数$\Phi(x)$的导数存在,并且
\begin{equation*}
  \Phi'(x) = f(x).
\end{equation*}

\paragraph{}
若$x=a$,取$\Delta x > 0$,则同理可证$\Phi'_+(a) = f(a)$;若$x=b$,取$\Delta x < 0$,则同理可证$\Phi'_-(b)=f(b)$。

\paragraph{}
\label{积分上限的函数}
\textbf{定理2\;}如果函数$f(x)$在区间$[a,b]$上连续,则函数
\begin{equation}
  \label{积分上限函数的定义式}
  \Phi(x)=\int_a^xf(t)dt
\end{equation}
就是$f(x)$在$[a,b]$上的一个原函数。

\subsection{牛顿-莱布尼茨公式}
\label{sec:牛顿-莱布尼茨公式}
\paragraph{}
\textbf{定理3\;}如果函数$F(x)$是连续函数$f(x)$在区间$[a,b]$上的一个原函数,则
\begin{equation}
  \label{微积分基本公式}
  \int_a^bf(x)dx = F(b) - F(a).
\end{equation}

\paragraph{}
\textbf{证\;}已知函数$F(x)$是连续函数$f(x)$的一个原函数,又根据\linkref[积分上限的函数]{定理2}知道,积分上限的函数
\begin{equation}
  \Phi(x) = \int_a^xf(t)dt
\end{equation}
也是$f(x)$的一个原函数。于是这两个原函数之差$F(x) - \Phi(x)$在$[a,b]$上必定是某一个常数$C$,即
\begin{equation}
  \label{定积分原函数之差}
  F(x) - \Phi(x) = C \; (a \leq x \leq b).
\end{equation}
在上式中令$x=a$,得$F(a) - \Phi(a) = C$。又由$\Phi(x)$的定义式\eqref{积分上限函数的定义式}可知$\Phi(a) = 0$,因此,$C=F(a)$。以$F(a)$代入\eqref{定积分原函数之差}式中的$C$,以$\displaystyle\int_a^xf(t)dt$代入\eqref{定积分原函数之差}式中的$\Phi(x)$,可得

\begin{equation}
  \int_a^xf(t)dt = F(x) - F(a).
\end{equation}

在上式中令$x=b$,就得到所要证明的公式\eqref{微积分基本公式}。

\paragraph{}
\eqref{微积分基本公式}式对$a>b$的情形同样成立。为了方便起见,以后把$F(b) - F(a)$记成$[F(x)]_a^b$,于是\eqref{微积分基本公式}式又可写成
\begin{equation}
  \int_a^bf(x)dx = [F(x)]_a^b.
\end{equation}

\paragraph{}
公式\eqref{微积分基本公式}叫做\uwave{牛顿(Newton)-莱布尼茨(Leibniz)公式},也通常叫做\uwave{微积分基本公式}。
