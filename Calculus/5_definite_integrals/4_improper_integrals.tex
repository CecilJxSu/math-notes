\paragraph{}
在一些实际问题中,常会遇到积分区间为无穷区间,或者被积函数为无界函数的积分,它们已经不属于前面所说的定积分了。因此,我们对定积分作如下两种推广,从而形成\uwave{反常积分}的概念。

\subsection{无穷限的反常积分}
\paragraph{}
\textbf{定义1\;}设函数$f(x)$在区间$[a,+\infty)$上连续,取$t>a$,如果极限
\begin{equation}
  \lim_{t \to +\infty} \int_a^tf(x)dx
\end{equation}
存在,则称此极限为\uwave{函数{$f(x)$}在无穷区间{$[a,+\infty)$}上的反常积分},记作$\displaystyle\int_a^{+\infty}f(x)dx$,即

\begin{equation}
  \int_a^{+\infty}f(x)dx = \lim_{t \to +\infty}\int_a^tf(x)dx,
\end{equation}
这时也称\uwave{反常积分}$\displaystyle\int_a^{+\infty}f(x)dx$\uwave{收敛};如果上述极限不存在,则函数$f(x)$在无穷区间$[a,+\infty)$上的反常积分$\displaystyle\int_a^{+\infty}f(x)dx$就没有意义,习惯上称为\uwave{反常积分}$\displaystyle\int_a^{+\infty}f(x)dx$\uwave{发散},这时记号$\displaystyle\int_a^{+\infty}f(x)dx$不再表示数值了。

\paragraph{}
类似地,函数$f(x)$在区间$(-\infty,b]$或$(-\infty,+\infty)$上也存在相同概念。

\paragraph{}
上述反常积分统称为\uwave{无穷限的反常积分}。

\subsection{无界函数的反常积分}
\paragraph{}
如果函数$f(x)$在点$a$的任一领域内都无界,那么点$a$称为函数$f(x)$的\uwave{瑕点}(也称为无界间断点)。无界函数的反常积分又称为\uwave{瑕积分}。

\textbf{定义2\;}设函数$f(x)$在$(a,b]$上连续,点$a$为$f(x)$的瑕点。取$t>a$,如果极限
\begin{equation}
  \lim_{t \to a^+}\int_t^bf(x)dx
\end{equation}
存在,则称此极限为\uwave{函数{$f(x)$}在{$(a,b]$}上的反常积分},仍然记作$\displaystyle\int_a^bf(x)dx$,即
\begin{equation}
  \int_a^bf(x)dx = \lim_{t \to a^+}\int_t^bf(x)dx.
\end{equation}
这时也称\uwave{反常积分}$\displaystyle\int_a^bf(x)dx$\uwave{收敛}。如果上述极限不存在,则称\uwave{反常积分}$\displaystyle\int_a^bf(x)dx$\uwave{发散}。

\paragraph{}
类似地,可得到函数$f(x)$在$[a,b)$上的反常积分定义。

\paragraph{}
设函数$f(x)$在$[a,b]$上除点$c \; (a<c<b)$外连续,点$c$为$f(x)$的瑕点。如果两个反常积分

\begin{equation}
  \int_a^cf(x)dx \;\text{与}\; \int_c^bf(x)dx
\end{equation}
都收敛,则定义

\begin{align}
\begin{split}
  \int_a^bf(x)dx \;=&\; \int_a^cf(x)dx + \int_c^bf(x)dx \\
  =&\; \lim_{t \to c^-}\int_a^tf(x)dx + \lim_{t\to c^+}\int_t^bf(x)dx;
\end{split}
\end{align}
否则,就称\uwave{反常积分}$\displaystyle\int_a^bf(x)dx$\uwave{发散}。
