\subsection{定积分的换元法}
\paragraph{}
\textbf{定理\;}假设函数$f(x)$在区间$[a,b]$上连续,函数$x=\varphi(t)$满足条件:
\begin{enumerate}
  \item $\varphi(\alpha) = a, \varphi(\beta) = b$;
  \item $\varphi(t)$在$[\alpha,\beta]$(或$[\beta,\alpha]$)上具有连续导数,且其值域$R_\varphi=[a,b]$,则有
  \begin{equation}
    \label{换元公式}
    \int_a^bf(x)dx = \int_\alpha^\beta f[\varphi(t)]\varphi'(t)dt.
  \end{equation}
\end{enumerate}
公式\eqref{换元公式}叫做定积分的\uwave{换元公式}。

\paragraph{}
\textbf{证\;}由假设可以知道,上式两边的被积函数都是连续的,因此不仅上式两边的定积分都存在,而且由上节的\linkref[积分上限函数的定义式]{定理2}知道,被积函数的原函数也都存在。所以,\eqref{换元公式}式两边的定积分都可应用牛顿-莱布尼茨公式。假设$F(x)$是$f(x)$的一个原函数,则
\begin{equation*}
  \int_a^bf(x)dx = F(b) - F(a).
\end{equation*}
另一方面,记$\Phi(t)=F[\varphi(t)]$,它是由$F(x)$与$x=\varphi(t)$复合而成的函数。由复合函数求导法则,得
\begin{equation*}
  \Phi'(t) = \frac{dFdx}{dxdt} = f(x)\varphi'(t) = f[\varphi(t)]\varphi'(t).
\end{equation*}
这表明$\Phi(t)$是$f[\varphi(t)]\varphi'(t)$的一个原函数。因此有
\begin{equation*}
  \int_\alpha^\beta f[\varphi(t)]\varphi'(t)dt = \Phi(\beta) - \Phi(\alpha).
\end{equation*}
又由$\Phi(t)=F[\varphi(t)]$及$\varphi(\alpha) = a, \varphi(\beta) = b$可知
\begin{equation*}
  \Phi(\beta) - \Phi(\alpha) = F[\varphi(\beta)] - F[\varphi(\alpha)] = F(b) - F(a).
\end{equation*}
所以
\begin{align*}
  \int_a^bf(x)dx \;=&\; F(b) - F(a) = \Phi(\beta) - \Phi(\alpha) \\
  =&\; \int_\alpha^\beta f[\varphi(t)]\varphi'(t)dt.
\end{align*}
这就证明了换元公式。

\paragraph{}
换元公式也可反过来使用。为使用方便起见,把换元公式中左右两边对调位置,同时把$t$改记为$x$,而$x$改记为$t$,得
\begin{equation}
  \int_a^bf[\varphi(x)]\varphi'(x)dx = \int_\alpha^\beta f(t)dt.
\end{equation}
这样,我们可用$t=\varphi(x)$来引入新变量$t$,而$\alpha = \varphi(a), \beta = \varphi(b)$。

\subsection{定积分的分部积分法}
\paragraph{}
依据不定积分的分部积分法,可得
\begin{align}
\begin{split}
  \int_a^bu(x)v'(x)dx \;=&\; \big[ \int u(x)v'(x)dx \big]_a^b \\
    =&\; \big[ u(x)v(x) - \int v(x)u'(x)dx \big]_a^b \\
    =&\; \big[ u(x)v(x) \big]_a^b - \int_a^b v(x)u'(x)dx,
\end{split}
\end{align}
简记作
\begin{equation}
  \int_a^buv'dx = [uv]_a^b - \int_a^bvu'dx,
\end{equation}
或
\begin{equation}
  \int_a^budv = [uv]_a^b - \int_a^bvdu.
\end{equation}
这就是\uwave{定积分的分部积分公式}。公式表明原函数已经积出的部分可以先用上、下限代入。
