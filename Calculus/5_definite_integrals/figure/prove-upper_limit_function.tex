% 积分上限的函数证明
\begin{tikzpicture}[scale=0.8]
  \begin{axis}[clip=false,xmin=0, xmax=13,ymin=0,ymax=13, grid=none,
    xtick=\empty, ytick=\empty, axis lines=middle,
    smooth, xlabel={$x$}, ylabel={$y$}]

    % 曲线:python, (x/3.5-1)**3-2*(x/3.5-1)**2+6
    \addplot[draw=red,domain=1:12] {(x/3.5-1)^3-2*(x/3.5-1)^2+8};

    \addplot[domain=1:7,pattern=north east lines,draw opacity=0] {(x/3.5-1)^3-2*(x/3.5-1)^2+8} \closedcycle;

    \node [above] at (3.4,7.956) {$y=f(x)$};

    \draw (1,0) -- (1,6.615);
    \draw (12,0) -- (12,10.53);
    \draw (7,0) -- (7,7);
    \draw [dashed] (8.5,0) -- (8.5,6.83);
    \draw (10,0) -- (10,7.51);

    \node [below] at (1,0) {$a$};
    \node [below] at (12,0) {$b$};
    \node [below] at (7,0) {$x$};
    \node [below] at (8.5,0) {$\xi$};
    \node [below] at (10.2,0) {$x+\Delta x$};
    \node [fill=white] at (8.5,3.3) {$f(\xi)$};
    \node [fill=white] at (4,3.3) {$\Phi(x)$};

    % 原点
    \node [below left] at (0,0) {$O$};
  \end{axis}
\end{tikzpicture}
