\subsection{第一类换元法}
\paragraph{}
设$f(u)$具有原函数$F(u)$,即
\begin{equation}
  F'(u) = f(u), \; \int{f(u)du} = F(u) + C.
\end{equation}
如果$u$是中间变量:$u=\varphi(x)$,且设$\varphi(x)$可微,那么,根据复合函数微分法,有
\begin{equation}
  dF[\varphi(x)] = f[\varphi(x)]\varphi'(x)dx,
\end{equation}
从而根据不定积分的定义就得
\begin{equation}
 \int f[\varphi(x)]\varphi'dx=F[\varphi(x)] + C = \big[\int f(u)du\big]_{u=\varphi(x)}.
\end{equation}
因此:

\paragraph{}
\textbf{定理\;}设$f(u)$具有原函数,$u=\varphi(x)$可导,则有\textbf{换元公式}
\begin{equation}
  \int f[\varphi(x)]\varphi'(x)dx = \big[\int f(u)du\big]_{u=\varphi(x)}.
\end{equation}

\subsubsection{例子1}
\paragraph{}
被积函数中含有三角函数,需要用到一些三角恒等式。

\paragraph{}
\textbf{例子1\;}求$\displaystyle\int{\sin^3xdx}$

\paragraph{}
\textbf{解\;}三角恒等变换后,令$u=\cos{x}$作变换。
\begin{align*}
\begin{split}
  \int\sin^3xdx \;=&\; \int\sin^2x\sin{x}dx \\
  =&\; -\int(1-\cos^2x)d(\cos{x}) \\
  =&\; -\cos{x} + \frac{1}{3}\cos^3{x} + C.
\end{split}
\end{align*}

\subsubsection{例子2}
\paragraph{}
对于$\sin^{2k+1}x\cos^nx$或$\sin^nx\cos^{2k+1}x$(其中$k\in N$)型函数的积分,总可依次作变换$u=\cos{x}$或$u=\sin{x}$,求得结果。

\paragraph{}
\textbf{例子2\;}求$\displaystyle\int{\sin^2x\cos^5xdx}$

\paragraph{}
\textbf{解\;}三角恒等变换后,令$u=\sin{x}$作变换。
\begin{align*}
\begin{split}
  \int\sin^2x\cos^5xdx \;=&\; \int\sin^2x\cos^4x\cos{x}dx \\
    =&\; \int\sin^2x(1-\sin^2x)^2d(\sin{x})  \\
    =&\; \int(\sin^2x-2\sin^4x + \sin^6x)d(\sin{x}) \\
    =&\; \frac{1}{3}\sin^3x-\frac{2}{5}\sin^5x+\frac{1}{7}\sin^7x + C.
\end{split}
\end{align*}

\subsubsection{例子3}
\label{第一类换元法例子3}
\paragraph{}
对于$\sin^{2k}x\cos^{2t}x \; (k,l \in N)$型的函数,总可利用三角恒等式:$\displaystyle\sin^2x=\frac{1}{2}(1-\cos{2x}),$ \\ $ \cos^2x = \frac{1}{2}(1+\cos{2x})$化成$\cos{2x}$的多项式。

\paragraph{}
\textbf{例子3\;}求$\displaystyle\int\sin^2x\cos^4xdx$

\paragraph{}
\textbf{解\;}三角恒等变换化简后,对多项式进行求积分
\begin{align*}
\begin{split}
  \int\sin^2x\cos^4xdx \;=&\; \frac{1}{8}\int(1-\cos{2x})(1+\cos{2x})^2dx \\
  =&\; \frac{1}{8} \int (1+\cos{2x} - \cos^2{2x} - \cos^3{2x})dx \\
  =&\; \frac{1}{8} \int (\cos{2x} - \cos^3{2x})dx + \frac{1}{8} \int (1 - \cos^2{2x})dx \\
  =&\; \frac{1}{8} \int \cos{2x}(1 - \cos^2{2x})dx + \frac{1}{8} \int (1 - \cos^2{2x})dx \\
  =&\; \frac{1}{8} \int \sin^2{2x} \bigcdot \frac{1}{2}d(\sin{2x}) + \frac{1}{8} \int \frac{1}{2} (1-\cos{4x}dx) \\
  =&\; \frac{1}{48} \sin^3{2x} + \frac{x}{16} - \frac{1}{64} \sin{4x} + C.
\end{split}
\end{align*}

\subsubsection{例子4}
\paragraph{}
对于$\tan^nx\sec^{2k}x$或$\tan^{2k-1}x\sec^nx \; (k \in N^+)$型函数的积分,可依次变换$u=\tan{x}$或$u=\sec{x}$,求得结果。

\paragraph{}
\textbf{例子4\;}求$\displaystyle\int\tan^5x\sec^3xdx$

\paragraph{}
\textbf{解\;}利用$\tan^2x = \sec^2x-1; \; (\sec{x})' = \sec{x}\tan{x}$
\begin{align*}
\begin{split}
  \int\tan^5x\sec^3xdx \;=&\; \int\tan^4x\sec^2x\sec{x}\tan{x}dx \\
  =&\; \int(\sec^2x-1)^2\sec^2xd(\sec{x}) \\
  =&\; \int(\sec^6x-2\sec^4x+\sec^2x)d(\sec{x}) \\
  =&\; \frac{1}{7}\sec^7x - \frac{2}{5}\sec^5x + \frac{1}{3}\sec^3x + C
\end{split}
\end{align*}

\subsection{第二类换元法}
\paragraph{}
第一类换元法是通过变量代换\uwave{$u=\varphi(x)$},将积分$\displaystyle\int{f[\varphi(x)]\varphi'(x)dx}$化为积分$\displaystyle\int{f(u)du}$.

\paragraph{}
第二类换元法是:选择适当的变量代换\uwave{$x=\psi(t)$},将积分$\displaystyle\int{f(x)dx}$化为积分 \\ $\displaystyle\int{f[\psi(t)]\psi'(t)dt}$。与第一类换元法的区别是,变量代换不同,公式和思路方向相反。

\paragraph{}
\textbf{定理2\;}设$x=\psi(t)$是单调的、可导的函数,并且$\psi'(t)\neq 0$,又设$f[\psi(t)]\psi'(t)$具有原函数,则有换元公式
\begin{equation}
  \int f(x)dx = \big[\int f[\psi(t)]\psi'(t)dt\big]_{t=\psi^-1(x)},
\end{equation}
其中$\psi^{-1}(x)$是$x=\psi(t)$的反函数。

\subsubsection{例子1}
\paragraph{}
有根式$\sqrt{a^2 - x^2}$,可以利用三角公式$\sin^2t+\cos^2t=1$来化去根式。

\paragraph{}
\textbf{例子1\;}求$\displaystyle \int\sqrt{a^2 - x^2}dx \; (a > 0)$。

\paragraph{}
\textbf{解\;}设$x=a\sin{t}, \; -\frac{\pi}{2} < t < \frac{\pi}{2}$,那么$\sqrt{a^2-x^2}=\sqrt{a^2-a^2\sin^2t} = a\cos{t}, dx = a\cos{t}dt$,于是根式化成了三角式,所求积分化为
\begin{equation}
  \int\sqrt{a^2-x^2}dx = \int a\cos{t} \bigcdot a\cos{t}dt = a^2\int \cos^2tdt.
\end{equation}
利用第一类换元法\linkref[第一类换元法例子3]{例子3}里的方法,得到
\begin{align*}
\begin{split}
  \int\sqrt{a^2 - x^2}dx \;=&\; a^2(\frac{t}{2}+\frac{\sin{2t}}{4}) + C \\
  =&\; \frac{a^2}{2}t + \frac{a^2}{2}\sin{t}\cos{t} + C.
\end{split}
\end{align*}
由于$x=a\sin{t},\; -\frac{\pi}{2} < t < \frac{\pi}{2}$,所以
\begin{align*}
  t \;=&\; \arcsin\frac{x}{a} \\
  \cos{t} = \sqrt{1-\sin^2t} \;=&\; \sqrt{1-(\frac{x}{a})^2} = \frac{\sqrt{a^2-x^2}}{a},
\end{align*}
于是所求积分为
\begin{equation*}
  \int\sqrt{a^2-x^2}dx = \frac{a^2}{2}\arcsin\frac{x}{a} + \frac{1}{2}x\sqrt{a^2-x^2} + C.
\end{equation*}

\subsubsection{例子2}
\paragraph{}
有根式$\sqrt{a^2 + x^2}$,可以利用三角公式$1 + \tan^2t = \sec^2t$来化去根式。

\subsubsection{例子3}
\paragraph{}
有根式$\sqrt{x^2 - a^2}$,可以利用三角公式$\sec^2t - 1 = \tan^2t$来化去根式。

\subsubsection{例子4}
\paragraph{}
上面的根式$\sqrt{x^2 \pm a^2}$,也可利用公式$\ch^2t - \sh^2t = 1$,采用代换$x=a\sh{t}$或$x=\pm a\ch{t}$来化去根式。

\paragraph{}
新增双曲函数积分公式:
\begin{enumerate}
  \item $\displaystyle\int\sh{x}dx = \ch{x} + C$,
  \item $\displaystyle\int\ch{x}dx = \sh{x} + C$。
\end{enumerate}

\subsection{倒代换}
\paragraph{}
利用倒代换,可消去被积函数的分母中的变量因子$x$。

\paragraph{}
\textbf{例子\;}求$\displaystyle\int\frac{\sqrt{a^2-x^2}}{x^4}dx$。

\paragraph{}
\textbf{解\;}设$\displaystyle x=\frac{1}{t}$,那么$\displaystyle dx = -\frac{dt}{t^2}$,于是
\begin{align*}
\begin{split}
  \int\frac{\sqrt{a^2-x^2}}{x^4}dx \;=&\; \int\frac{\sqrt{a^2-\frac{1}{t^2}} \bigcdot (-\frac{dt}{t^2})}{\frac{1}{t^4}} \\
  =&\; -\int(a^2t^2-1)^{\frac{1}{2}}|t|dt,
\end{split}
\end{align*}
当$x>0$时,有
\begin{align*}
\begin{split}
  \int\frac{\sqrt{a^2-x^2}}{x^4}dx \;=&\; -\frac{1}{2a^2}\int(a^2t^2-1)^{\frac{1}{2}}d(a^2t^2-1) \\
  =&\; -\frac{(a^2t^2-1)^{\frac{3}{2}}}{3a^2} + C \\
  =&\; -\frac{(a^2-x^2)^{\frac{3}{2}}}{3a^2x^3} + C,
\end{split}
\end{align*}
当$x<0$时,有相同的结果。
