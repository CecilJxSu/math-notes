\subsection{原函数与不定积分的概念}
\paragraph{}
\textbf{定义1\;}如果在区间$I$上,可导函数$F(x)$的导函数为$f(x)$,即对任一$x\in I$,都有
\begin{equation}
  F'(x) = f(x) \;\textbf{或}\; dF(x) = f(x)dx,
\end{equation}
那么函数$F(x)$就称为$f(x)$(或$f(x)dx$)在区间$I$上的\uwave{原函数}。

\paragraph{}
\textbf{原函数存在定理\;}如果函数$f(x)$在区间$I$上连续,那么在区间$I$上存在可导函数$F(x)$,使对任一$x\in I$都有
\begin{equation*}
  F'(x) = f(x).
\end{equation*}
即\textbf{连续函数一定有原函数}。

\paragraph{}
\textbf{定义2\;}在区间$I$上,函数$f(x)$的带有任意常数项的原函数称为$f(x)$(或$f(x)dx$)在区间$I$上的\uwave{不定积分},记作
\begin{equation}
  \int{f(x)dx},
\end{equation}
其中记号$\int$称为\uwave{积分号},$f(x)$称为\uwave{被积函数},$f(x)dx$称为\uwave{被积表达式},$x$称为\uwave{积分变量}。

\paragraph{}
由此定义及前面的说明可知,如果$F(x)$是$f(x)$在区间$I$上的一个原函数,那么$F(x)+C$就是$f(x)$的不定积分,即
\begin{equation}
  \int{f(x)dx} = F(x) + C.
\end{equation}
因而不定积分$\int{f(x)dx}$可以表示$f(x)$的任意一个原函数。

\subsection{基本积分表}
\paragraph{}

\bgroup
\def\arraystretch{3}
\setlength\tabcolsep{0.8cm}
\begin{table}[H]
\centering
  \caption{基本积分表}
  \begin{tabular}{l|l}
    \hline
    $\displaystyle\int{kdx}=kx+C(k\text{是常数})$ &
    $\displaystyle\int{x^\mu dx}=\frac{x^{\mu+1}}{\mu+1} + C(\mu \neq -1)$ \\
    \hline
    $\displaystyle\int{\frac{dx}{x}} = \ln{|x|} + C$ &
    $\displaystyle\int{\frac{dx}{1+x^2}} = \arctan{x} + C$ \\
    \hline
    $\displaystyle\int{\frac{dx}{\sqrt{1-x^2}}} = \arcsin{x} + C$ &
    $\displaystyle\int{\cos{x}dx} = \sin{x} + C$ \\
    \hline
    $\displaystyle\int{\sin{x}dx} = -\cos{x} + C$ &
    $\displaystyle\int{\frac{dx}{\cos^2x}} = \int{\sec^2xdx} = \tan{x} + C$ \\
    \hline
    $\displaystyle\int{\frac{dx}{\sin^2x}} = \int{\csc^2xdx} = -\cot{x} + C$ &
    $\displaystyle\int{\sec{x}\tan{x}dx} = \sec{x} + C$ \\
    \hline
    $\displaystyle\int{\csc{x}\cot{x}dx} = -\csc{x} + C$ &
    $\displaystyle\int{e^xdx} = e^x + C$ \\
    \hline
    $\displaystyle\int{a^xdx} = \frac{a^x}{\ln{a}} + C$ & \\
    \hline
  \end{tabular}
\end{table}
\egroup

\subsection{不定积分的性质}
\paragraph{}
\textbf{性质1\;}设函数$f(x)$及$g(x)$的原函数存在,则
\begin{equation}
  \int{[f(x)+g(x)]dx} = \int{f(x)dx} + \int{g(x)}dx.
\end{equation}

\paragraph{}
\textbf{性质2\;}设函数$f(x)$的原函数存在,$k$为非零常数,则
\begin{equation}
  \int{kf(x)dx}=k\int{f(x)dx}.
\end{equation}
